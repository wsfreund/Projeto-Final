\chapter{Resultados e Discussão}
\label{cap:resultados}

Este capítulo é dedicado à apresentação dos resultados e a discussão dos estudos
realizados. A primeira abordagem (Seção~\ref{sec:norm}) tratou a otimização do
algoritmo através da normalização mais indicada para o \gls{hltringer}, 
a versão do algoritmo proposto no \glsdesc{l2}, que teve os resultados 
posteriormente confirmados nos estudos realizados
por \cite{tese_torres}. Em seguida, serão apresentados os resultados para a
versão do algoritmo no \gls{sr} (Seção~\ref{sec:efic_egcalo}), o ambiente de
reconstrução da física a posteriori.


\section{Otimização do \emph{HLT\_Ringer}: Normalização}
\label{sec:norm}

Este estudo foi dedicado a escolha da normalização a ser utilizada nos
dados antes da propagação dos anéis para a \gls{rna}. Foi utilizada a
implementação do algoritmo proposto para o \gls{sf}, pois na época ainda não
havia sido requisitado pela Colaboração a implementação da versão para o \gls{sr}. Uma descrição das
normalizações testadas foi realizada no Tópico~\ref{sssec:preproc_norm}. Foram
utilizadas simulações de \gls{mc} de 2008, contendo elétrons isolados (não sendo
estudado, assim, o efeito de Empilhamento) para o conjunto de sinal e jatos com 
\gls{Et} máximo de 17 GeV para o conjunto de ruído. As redes foram treinadas com
os parâmetros especificados no Tópico~\ref{sssec:rna}, todas contendo 10
neurônios na camada escondida. Uma exceção, foi realizado apenas uma única 
inicialização e treinamento da \gls{rna}, de forma que os resultados
podem ser influênciados por flutuação estatística. Contudo, os resultados 
foram reproduzidos por \cite{tese_torres} para as normalizações que se ressaltaram 
neste estudo utilizando 10 inicializações, obtendo resultados semelhates de 
forma que a flutuação não afetou significativamente os resultados obtidos. O
estudo também considerou os dois critérios utilizados durante o treinamento da
\gls{rna} como figura de mérito, de modo a verificar se a utilização do \gls{sp}
como critério de parada ao invés do \gls{mse} realmente melhoraria a perfomance
do \gls{rna}.

\begin{table}[htb]
\centering
\resizebox{\textwidth}{!}{
\begin{tabular}{
>{\global\let\currentrowstyle\relax}p{4.5cm}
>{\centering\arraybackslash\currentrowstyle}p{1.5cm}
>{\centering\arraybackslash\currentrowstyle}p{1.5cm}
>{\centering\arraybackslash\currentrowstyle}p{1.5cm}
>{\sl\centering\arraybackslash\currentrowstyle}p{1.5cm}
>{\centering\arraybackslash\currentrowstyle}p{1.5cm}
>{\centering\arraybackslash\currentrowstyle}p{1.5cm}
>{\centering\arraybackslash\currentrowstyle}p{1.5cm}}
\hline \hline
\textbf{Normalização}&\textbf{Épocas}&\textbf{\gls{mse}\linebreak
TRN}& \textbf{{Critério\linebreak VAL}}&\textbf{\gls{sp}\linebreak
TST}&\textbf{\gls{det}\linebreak TST}&\textbf{\gls{fa}\linebreak
TST}&\textbf{Limiar}\\
\hline \hline
Fixa \gls{sp}                   &503&0.0418&98.7320&98.7498&99.2925&1.7914&0.028\\\hline
\rowstyle{\bfseries}%
Fixa Seção \gls{sp}             &339&0.0415&98.7370&98.7544&99.3257&1.8152&-0.016\\\hline
Fixa Seção \gls{sp} \gls{mod}   &329&0.0445&98.7168&98.7349&99.2721&1.8009&0.036\\\hline
Fixa Camada \gls{sp}            &295&0.0424&98.7218&98.7309&99.2938&1.8304&-0.028\\\hline
\rowstyle{\bfseries}%
Norma 1 \gls{sp}                 &325&0.0427&98.6154&98.6087&99.2798&2.0601&-0.052\\\hline
Norma 2 \gls{sp}                 &101&0.0444&98.6357&98.6399&99.0827&1.8019&0.224\\\hline
Norma 2 Seção \gls{sp}           &477&0.0546&98.3972&98.4319&99.1426&2.2764&0.028\\\hline
Norma 2 Camada \gls{sp}          &584&0.0615&98.1071&98.0881&98.7069&2.5289&0.052\\\hline
Sequencial \gls{sp}             &837&0.0610&98.0918&98.1530&98.7242&2.4165&0.088\\\hline
MinMax \gls{sp}                 &63 &0.8899&92.7524&92.8777&94.1037&8.3404&0.172\\\hline
\rowstyle{\bfseries}%
Esferização \gls{sp}            &237&0.0421&98.7472&98.7690&99.2792&1.7399&0.112\\\hline
\rowstyle{\bfseries}%
Esferização \gls{sp} \gls{mod}  &158&0.0434&98.7674&98.7748&99.3723&1.8209&0.004\\\hline
Fixa \gls{mse}                  &210&0.0465&0.0396 &98.7254&99.1879&1.7361&0.156\\\hline
Fixa Camada \gls{mse}           &131&0.0427&0.0400 &98.7042&99.3927&1.9820&-0.088\\\hline
Fixa Seção \gls{mse}            &253&0.0438&0.0393 &98.7192&99.3040&1.8638&0.036\\\hline
Fixa Seção \gls{mse} \gls{mod}  &175&0.0422&0.0394 &98.7218&99.3110&1.8657&0.020\\\hline
Norma 1 \gls{mse}                &117&0.0453&0.0428 &98.6006&99.2473&2.0439&0.000\\\hline
Norma 2 \gls{mse}                &105&0.0469&0.0431 &98.6262&99.1758&1.9219&0.144\\\hline
Norma 2 Seção \gls{mse}          &242&0.0541&0.0497 &98.4041&99.0208&2.2106&0.172\\\hline
Norma 2 Camada \gls{mse}         &522&0.0602&0.0585 &98.0544&98.8237&2.7118&-0.052\\\hline
Sequencial \gls{mse}            &538&0.0703&0.0636 &97.9352&98.6929&2.8195&0.088\\\hline
MinMax \gls{mse}                &331&0.1458&0.1385 &95.1713&96.3269&5.9773&-0.032\\\hline
Esferização \gls{mse}           &279&0.0410&0.0383&98.7580&99.2390&1.7218&0.168\\\hline
Esferização \gls{mse} \gls{mod} &243&0.0411&0.0372 &98.7662&99.3704&1.8362&0.000\\
\hline \hline
\end{tabular}
}
\caption[Resultados do estudo de pré-processamento: Normalização]{Resultados do
estudo de pré-processamento.}
\label{tab:res_norm}
\end{table}

Na Tabela~\ref{tab:res_norm} estão os resultados obtidos, como os resultados
para o de treinamento, como o número de épocas, o \gls{mse} do \gls{trn} em 
relação aos valores especificados, e o valor da figura de mérito
utilizada como critério de parada no conjunto de \gls{val}. O valor de
eficiência da \gls{rna} utilizando o \gls{sp} está destacado, pois esse valor foi
escolhido como a figura de mérito da eficiência da rede, e em seguida estão os
respectivos \gls{det}, e o \gls{fa} utilizados para o seu calculo e o limiar de
decisão da \gls{rna}.

Pode-se observar que a Esferização e sua versão
modificada obtiveram as melhores eficiências. Logo em seguida estão as
normalizações fixas, onde se destaca a versão por Seção da mesma. As
normalizações por norma obtiveram resultados similares, e não muito distantes
dos melhores resultados anteriormente indicados, o que pode ser uma vantagem em
alguns casos devido a sua praticidade de aplicação, aonde não é necessário a
descoberta do desvio padrão e média dos dados, ainda mais interessante para o
caso da Norma 1, onde simplesmente se soma a energia de todos os anéis para
então dividí-los por esse valor. A normalização sequencial, apesar de todo o
conhecimento especialista aplicado para a otimização da Norma 1, obteve
resultados bem inferiores aos demais, sendo talvez necessário a alteração de
seus parâmetros para uma melhor eficiência. A utilização de normalização por
Mínimo e Máximo obteve o pior resultado. Finalmente, avaliou-se que o
treinamento com o critério de parada por \gls{sp} obtém valores maiores de
eficiência que quando treinando por \gls{mse}, como o esperado.


\section[Estudo de Eficiência do \emph{eGamma Calorimeter Ringer}]{Estudo 
de Eficiência do $e/\gamma$ \emph{Calorimeter Ringer}}
\label{sec:efic_egcalo}

Para o estudo de eficiência do \gls{egcaloringer} foram utilizados três
conjuntos de dados contendo elétrons e o respectivo ruído esperado
(o nome dos conjuntos de dados utilizados estão em \cite{portal_caloringer}):
\begin{itemize}
\item \textbf{Singlepart\_e $\times$ J2} (Subseção~\ref{ssec:single_e}): originados através de simulações de
\gls{mc}:
\begin{itemize}
\item Conjunto de sinal: formado por elétrons isolados;
\item Conjunto de ruído: contém jatos hadrônicos.
\end{itemize}
\item \textbf{\text{J}/$\Psi \times$ Minbias} (Subseção~\ref{ssec:jpsi}): novamente original de simulações
de \gls{mc}:
\begin{itemize}
\item Conjunto de sinal: contém decaimentos de $\text{J}/\Psi$ em elétrons adicionados de eventos de
\gls{mbias} para simular o fenômeno de Empilhamento;
\item Conjunto de ruído: composto por eventos de \gls{mbias}.
\end{itemize}
\item \textbf{Z$\rightarrow$ee $\times$ JetTauEtmiss} (Subseção~\ref{ssec:Zee}): composto por dados de
colisões reais da temporada 167776, com pico de luminosidade de
$1,8\times10^{30}cm^{-2}s^{-1}$ (todas as informações da temporada estão
disponíveis em \cite{info_run}):
\begin{itemize}
\item Conjunto de sinal: Contém candidatos a Zee filtrados apenas pelo \gls{l1},
estando assim altamente contaminado por ruído físico;
\item Conjunto de ruído: Contém candidatos a jatos, taóns que também podem
formar jatos e \gls{Etmiss} (neutrinos).
\end{itemize}
\end{itemize}

\subsection{Metodologia} %?
\label{ssec:metodologia}

Para cada um dos casos foi utilizado os critérios de treinamento e separando os 
conjuntos de dados nas porcentagens dependendo da quantidade de estatística
disponível, indicadas no Tópico~\ref{sssec:rna}. Se realizou o processo de
inicialização e treinamento 3 vezes seguidas para cada quantidade de neurônios na camada
escondida testadas, utilizando uma variação entre $5-20$. O baixo valor de
inicializações se deu ao curto prazo para a apresentação dos resultados da
implementação e eficiência do algoritmo. 
Utilizou-se a normalização por Norma 1,
devido aos resultados obtidos na Seção~\ref{sec:norm} e em \cite{tese_torres},
onde a eficiência dessa normalização é ligeiramente mais baixa que aquelas de
melhor desempenho, mas é de simples implementação. A \gls{rna} foi treinada utilizando os
dados do conjunto de sinal que fossem aceitos pelo critério \emph{Medium} do
Algoritmo Padrão e rejeitasse as partículas do conjunto de ruído que não fossem
aceitos pelo critério \emph{Loose} desse algoritmo, para que fosse possível uma
boa detecção da parcela detectada por essas partículas, e ao mesmo tempo,
rejeitar uma quantidade superior que a do Algoritmo Padrão, procurando um ganho em
eficiência nesse quisito. Após o treinamento de uma rede para os casos a seguir,
foi realizado um estudo de eficiência, e uma busca por correlação
entre as variáveis físicas utilizadas pelo Algoritmo Padrão, para encontrar relações entre as
variáveis. A busca por regiões de interceção das partículas aceitas pelos os
requerimentos da \gls{rna}, \emph{\gls{rna} Loose}, \emph{\gls{rna} Medium} e
\emph{\gls{rna} Tight}, com as partículas aceitas pelos requerimentos do
Algoritmo Padrão, teve como o objetivo provar o ganho de eficiência. Nos estudos
de \gls{mc}, a verdade de \gls{mc}, contendo as informações sobre a partícula
gerada, sua energia, e outros parâmetros podem ser utilizados para aprimorar a
análise, de forma que essa informação disponível foi utilizada para esses
conjuntos.


\FloatBarrier

\subsection{\texorpdfstring{Singlepart\_e $\times$ J2}{Singlepart\_e x J2}}
\label{ssec:single_e}

Neste conjunto de dados há sobreposição dos dados para a faixa de energia por
volta dos $\sim10-35$ GeV, praticamente não havendo jatos na região de maiores
energias, como indicado na Figura~\ref{fig:singlexj2_distenergia}. Os conjuntos de
dados são limpos, contendo apenas elétrons no caso do Singlepart\_e, e jatos
para o J2.

\begin{figure}[ht]
\centering
\includegraphics[width=.7\textwidth]{imagens/CaloRinger_Analysis_ElectronVsJet/EnergyDistribution/CaloRinger_Analysis_ElectronVsJetdata_energy_dist.pdf}
\label{fig:singlexj2_distenergia}
\caption{Distribuição de energia para o conjunto Singlepart\_e x J2.}
\end{figure}

A Tabela~\ref{tab:singlexj2_efic} contém a eficiência dos algoritmos padrão e
\gls{egcaloringer}. O último teve suas taxas fixadas para obter as mesmas taxas
de \gls{fa} que aquelas de cada um dos requisitos do Algoritmo Padrão. Os
valores do algoritmo proposto foi bastante superior para os três requisitos, com
ganhos de 9,26\%, 25,18\% e 36,5\% de detecção para os critérios \emph{Loose},
\emph{Medium} e \emph{Tight}, respectivamente. As eficiências de
detecção para os algoritmos com seus requerimentos estão na
Tabela~\ref{tab:singlexj2_efic_det}, e as de falso alarme
na Tabela~\ref{tab:singlexj2_fa_det}. Os valores contidos no interior dessas
tabelas são as respectivas parcelas na qual o \gls{egcaloringer} tem em comum
com o Algoritmo Padrão. Por exemplo, com o critério \emph{\gls{rna} Medium} o
\gls{egcaloringer} é capaz de detectar 94,98\% dos elétrons, enquanto o
Algoritmo Padrão irá obter 71,82\% dessas partículas, onde o conjunto de
interseçã de partículas identificadas para ambos algoritmos com esses critérios
(ambos \emph{Medium}) será de 71,59\% das total de partículas no conjunto. Ao
mesmo tempo, 57,35\% das partículas do conjunto de sinal são identificadas por um critério
\emph{Tight} no Algoritmo Padrão e pelo critério \emph{\gls{rna} Tight},
mostrando que a \gls{rna} é capaz de identificar com alta fidelidade os elétrons
\emph{Medium} e \emph{Tight}. Mais do que alta fidelidade com esses requisitos,
o algoritmo implementado acabou superarando a taxa de deteção do Algoritmo
Padrão. Seguindo o mesmo raciocínio para a deteção de elétrons, as taxas de falso
alarme do Algoritmo Padrão são de 5,90 e 0,94, para os requerimentos \emph{Medium}
e \emph{Tight}, enquanto o \gls{egcaloringer} consegue valores menores, 1,16 e
0,67, de modo que além da melhor capacidade de identificação de elétrons, o
algoritmo tem melhor capacidade de rejeição de jatos.

\begin{table}[htb]
\centering
\begin{tabular}{cccc}
\hline
\hline
 & 
\multicolumn{2}{c}{DET (\%) para algoritmo} & 
\\
\cline{2-3}
\multirow{-2}{*}{Req. Do Alg. Padrão} & 
Alg. Padrão & 
EgCaloRinger & 
\multirow{-2}{*}{para FA (\%)} \\
\hline
Loose & 89,16 & 98,45 & 28,53 \\
Medium & 71,82 & 97,00 & 5,90 \\
Tight & 57,68 & 94,18 & 0,94 \\
\hline
\hline
\end{tabular}
\caption{Eficiências do algoritmos para FA fixos. Conjunto Singlepart\_e x J2.}
\label{tab:singlexj2_efic}
\end{table}

\begin{table}[htb]
\centering
\begin{tabular}{l cccc}
\hline
\hline
DET (\%)& Todo o Conj. & Loose & Medium & Tight \\
\hline
Todo o Conj. &  - & 89,16 & 71,82 & 57,68 \\
\hline
\gls{rna} Loose & 95,34 & 86,49/89,16 & 71,70/71,82 & 57,62/57,68 \\
\hline
\gls{rna} Medium & 94,58  & 86,11/89,16 & 71,59/71,82 & 57,55/57,68 \\
\hline
\gls{rna} Tight &  93,37 & 85,51/89,16 &  71,31/71,82 & 57,35/57,68 \\
\hline
\hline
\end{tabular}
\caption{Taxa de detecção (\%) dos algoritmos e parcela na qual o EgCaloRinger
identifica em comum ao algoritmo padrão. Conjunto Singlepart\_e x J2.}
\label{tab:singlexj2_efic_det}
\end{table}

\begin{table}[htb]
\centering
\begin{tabular}{l cccc}
\hline
\hline
FA (\%)& Todo o Conj. & Loose & Medium & Tight \\
\hline
Todo o Conj. & - & 28.57 & 5.90 &  0.94 \\
\gls{rna} Loose  & 1.88  & 1.49/28.57 & 0.32/5.90 & 0.11/0.94 \\
\gls{rna} Medium & 1.16  & 0.96/28.57 & 0.21/5.90 & 0.08/0.94 \\
\gls{rna} Tight  & 0.67  & 0.58/28.57 & 0.13/5.90 & 0.05/0.94 \\
\hline
\hline
\end{tabular}
\caption{Taxa de falso alarme (\%) dos algoritmos e parcela na qual o EgCaloRinger
identifica em comum ao algoritmo padrão. Conjunto Singlepart\_e x J2.}
\label{tab:singlexj2_fa_det}
\end{table}

A correlação da saída neural com os requerimentos do Algoritmo Padrão está
disposta na Figura~\ref{fig:singlexj2_saidaneural}. O degradê em tom de azul
identifica a contagem de elétrons com os valores de saída neural, e os
requerimentos aceitos pelo Algoritmo Padrão, sendo eles do tom mais escuro para
o mais claro: Sem Requerimento (todo o conjunto de dados), \emph{Loose},
\emph{Medium} e \emph{Tight}. O limiar de decisão para o corte \emph{\gls{rna}
Medium} está indicado na figura através da linha pontilhada vertical. Os valores
da saída neural para os elétrons que passaram critérios \emph{medium} e
\emph{tight} estão bastante concentrados em +1, o valor alvo para elétrons,
mostrando alta correlação entre a rede neural e esses critérios. Por outro lado,
o degradê em tom de vermelho identifica a saída neural para o conjunto de jatos,
novamente os tons mais escuros estão relacionados às partículas que
ultrapassaram critérios menos seletivos, e os tons mais claros, de maior
seletividade. É possível visualizar que jatos que foram aceitos pelo critério
mais seletivo (\emph{Tight}) do Algoritmo Padrão foram rejeitados pela rede neural, 
estando em sua maioria em -1, o valor alvo para jatos.

\begin{figure}[htb]
\centering
\resizebox{\textwidth}{!}{
\begin{tabular}{
>{\centering\arraybackslash}m{0.50\textwidth}@{\hskip 0.5cm}
>{\centering\arraybackslash}m{0.50\textwidth}@{\hskip 0.5cm}
}
\includegraphics[width=0.5\textwidth]{imagens/CaloRinger_Analysis_ElectronVsJet/NeuralOutput/CaloRinger_Analysis_ElectronVsJet_nnoutput_electrons.pdf}
&
\includegraphics[width=0.5\textwidth]{imagens/CaloRinger_Analysis_ElectronVsJet/NeuralOutput/CaloRinger_Analysis_ElectronVsJet_nnoutput_jets.pdf}
\\
(a) \textbf{Elétrons} & (b) \textbf{Jatos} \\
\end{tabular}
}
\label{fig:singlexj2_saidaneural}
\caption{Saída da rede neural e suas relações com os requerimentos do algoritmo
eGamma padrão. Conjunto Singlepart\_e x J2.}
\end{figure}

Não obstante, a curva~\gls{roc} pode ser utilizada para a escolha do limiar de
decisão ou comparação da eficiência dos algoritmos. Nela, a relação entre a taxa
de detecção e falso alarme é realizada, estando a detecção no eixo das
ordenadas, e o falso alarme nas abscissas, que são obtidas através do
deslocamento do limiar de decisão. Na Figura~\ref{fig:singlexj2_roc} está a
curva~\gls{roc} para a \gls{rna} treinada e os três pontos correspondentes aos
valores obtidos para os requerimentos do Algoritmo Padrão. A curva da rede
neural está sempre acima dos pontos do Algoritmo Padrão, de forma que o
\gls{egcaloringer} sempre terá eficiência melhor.

\begin{figure}[ht]
\centering
\includegraphics[width=.7\textwidth]{imagens/CaloRinger_Analysis_ElectronVsJet/Efficiency/CaloRinger_Analysis_ElectronVsJet_roc.pdf}
\label{fig:singlexj2_roc}
\caption{Curva ROC para o conjunto Singlepart\_e x J2.}
\end{figure}

As taxas de deteção, parte superior, e falso alarme, parte inferior, também podem ser 
observadas e comparadas para os algoritmos em função dos parâmetros \gls{Et} e \gls{eta} 
da partícula na Figura~\ref{fig:singlexj2_eficiencia_loose},
Figura~\ref{fig:singlexj2_eficiencia_medium} e Figura~\ref{fig:singlexj2_eficiencia_medium}, 
para respectivamente os critérios \emph{Loose}, \emph{Medium} e \emph{Tight}. A
distribuição assimétrica para o eixo positivo e negativo de \gls{eta} é incomum
e uma característica dos conjuntos de dados simulados de elétrons, observe que
os jatos (falso alarme), por sua vez, têm uma distribuição simétrica. O
\gls{egcaloringer} obteve uma eficiência superior para toda a região de energia
e \gls{eta} ao do Algoritmo Padrão, com excessão de algumas regiões de \gls{eta}
para a taxa de deteção no critério \emph{Loose}. Um fato importante também
observado nessas figuras é taxa de falso alarme constante para a região de fenda
no Sistema de Calorimetria (Subtópico~\ref{par:ecal_prec}), $|\gls{eta}|\sim1,45$, 
aonde operação do mesmo é degradada.

Um outro estudo possível em dados de \gls{mc} é a identificação das partículas
que se está tendo maior ou menor facilidade de se discriminar. A
Tabela~\ref{tab:singlexj2_part_j2} contém as taxas de rejeição de algumas
partículas estáveis -- partículas que irão interagir com o \gls{atlas} --, sendo
eles os fótons, píons, káons e elétrons para os requerimentos de ambos
algoritmos. O algoritmo proposto possuí um melhor potêncial de rejeição para
todas as partículas desse conjunto de dados, como os hádrons que compõem os
jatos, incluindo fótons e elétrons. Parte dos fótons contidos nesse
conjunto são de decaimentos de $\pi^0\rightarrow\gamma\gamma$, de forma que a
eliminação desses fótons é desejada. A pequena amostra de elétrons contida nesse
conjunto são compostos por elétrons não isolados que mais se assemelham com
hádrons, o que explica a maior rejeição pelo \gls{egcaloringer}.

\begin{table}[htb]
\centering
\resizebox{\textwidth}{!}{
\begin{tabular}{m{4cm} ccc ccc}
\hline
\hline
\multirow{2}{4cm}{Taxa de Rejeição (\%) para partícula estável} &
\multicolumn{3}{c}{Alg. Padrão} & \multicolumn{3}{c}{CaloRinger} \\
 & \textbf{Loose} & \textbf{Medium} & \textbf{Tight} & \textbf{\gls{rna} Loose} &
\textbf{\gls{rna} Medium} & \textbf{\gls{rna} Tight} \\
\hline
$\gamma$ & 78,38 & 94,10 & 99,34 & 98,69 & 99,24 & 99,58 \\
$\pi^+/\pi^-$ & 89,85 & 96,69 & 99,66 & 99,49 & 99,74 & 99,88 \\
$K^+/K^-$ & 91,81 & 97,17 & 99,66 & 99,50 & 99,72 & 99,86 \\
$K^0_l$ & 87,55 & 96,31 & 99,60 & 99,57 & 99,82 & 99,96 \\
$K^0_s$ & 87,70 & 96,85 & 99,57 & 99,49 & 99,70 & 99,86 \\
$e^+/e^-$ & 66,03 & 77,69 & 81,27 & 95,82 & 97,24 & 98,40 \\
\hline
\hline
\end{tabular}
}
\caption{Rejeição de partículas estáveis para o conjunto de J2.}
\label{tab:singlexj2_part_j2}
\end{table}


Finalmente, algumas das variáveis físicas foram foram traçadas em conjunto com
a saída neural em busca de correlações. As
Figuras~\ref{fig:singlexj2_rcore}-\ref{fig:singlexj2_width2} contém histogramas
de duas dimensões em escala logaritimica para o eixo z, 
a saída da rede neural está no eixo das ordenadas e a
variável física no eixo das abscissas. Essas variáveis são: \gls{reta},
\gls{eratio}, \gls{Rhad1} e \gls{weta2}; informações sobre essas as variáveis estão
contidas na Tabela~\ref{tab:cortes_em}. As linhas pontilhadas horizontais
indicam os três requerimento do \gls{egcaloringer}, sendo assim o limiar de
decisão da rede. Valores abaixo desse limiar serão considerados pela rede como
jatos, e acima como uma partícula \gls{eg}. As linhas pontilhadas verticais
apresentam uma referência para o corte da variável física, mas não devem ser
levadas como o seu valor verdadeiro uma vez que os cortes variam conforme
\gls{eta} e \gls{Et}. Um conjunto de 16 figuras estão
disponíveis para cada variável, sendo as linhas elétrons 
incidindo no barril, jatos incidindo no barril, elétrons incidindo na tampa, e
jatos incidindo na tampa. As colunas indicam os requerimentos (ou a ausência de) exigidos para as
partículas, estando presente nas figuras, no caso de elétrons, aqueles aceitos pelo 
requerimento, enquanto para jatos, aqueles que \emph{não} foram aceitos pelo
requerimento. Algumas observações podem ser realizadas, como no caso da variável
\gls{reta} para jatos localizados no
barril sem a exigência de requerimento (Sem Req.). É esperado que elétrons
tenham valores próximos a 1 nessa variável, justamente a região em que há uma
região de confusão para a saída neural, enquanto valores distantes de 1 são
raramente confundidos pela rede, sendo com frequência atribuidos a -1. Outra
variável, o \gls{eratio} também deverá ter valores próximos a 1 para elétrons,
onde novamente é concentrada a região de confusão da \gls{rna}. A variável
\gls{Rhad1} deverá se formar próximo de 0, uma vez que elétrons não devem
depositar energia no \gls{hcal}, enquanto valores negativos ou positivos muito
elevados são devido a jatos (os valores negativos se devem a deposito de energia
menor que o pedestal de ruído no \gls{ecal}), onde é formada a região de
confusão da rede neural para a região do barril, entretanto, na tampa a variável
tem uma distribuição relativamente uniforme. Finalmente, \gls{weta2} contém a
espessura do chuveiro na \gls{e2}, de forma que chuveiros muito largos deverão
ser formados por jatos, novamente sendo possível visualizar a região de confusão
formada próximo à variável e corte utilizado pelo Algoritmo Padrão.



%--------------------------------------------------
\begin{sidewaysfigure}[phb]
\centering
\resizebox{\textwidth}{!}{
\begin{tabular}{
>{\centering\arraybackslash}m{\textwidth}@{\hskip 0.01cm}
}
\includegraphics[width=0.7\textwidth,height=0.32\textheight]{imagens/CaloRinger_Analysis_ElectronVsJet/Efficiency/CaloRinger_Analysis_ElectronVsJet_comparison_loose_eff.pdf} \\
\includegraphics[width=0.7\textwidth,height=0.32\textheight]{imagens/CaloRinger_Analysis_ElectronVsJet/Efficiency/CaloRinger_Analysis_ElectronVsJet_comparison_loose_fa.pdf} \\
\end{tabular}
}
\label{fig:singlexj2_eficiencia_loose}
\caption{Comparações das eficiências em função das variáveis $\eta$ e $E_{T}$
para ambos algoritmos no requerimento \emph{Loose}. Conjunto Singlepart\_e x J2.}
\end{sidewaysfigure}

%--------------------------------------------------
\begin{sidewaysfigure}[phb]
\centering
\resizebox{\textwidth}{!}{
\begin{tabular}{
>{\centering\arraybackslash}m{\textwidth}@{\hskip 0.01cm}
}
\includegraphics[width=0.7\textwidth,height=0.32\textheight]{imagens/CaloRinger_Analysis_ElectronVsJet/Efficiency/CaloRinger_Analysis_ElectronVsJet_comparison_medium_eff.pdf} \\
\includegraphics[width=0.7\textwidth,height=0.32\textheight]{imagens/CaloRinger_Analysis_ElectronVsJet/Efficiency/CaloRinger_Analysis_ElectronVsJet_comparison_medium_fa.pdf} \\
\end{tabular}
}
\label{fig:singlexj2_eficiencia_medium}
\caption{Comparações das eficiências em função das variáveis $\eta$ e $E_{T}$
para ambos algoritmos no requerimento \emph{Medium}. Conjunto Singlepart\_e x J2.}
\end{sidewaysfigure}


%--------------------------------------------------
\begin{sidewaysfigure}[phb]
\centering
\resizebox{\textwidth}{!}{
\begin{tabular}{
>{\centering\arraybackslash}m{\textwidth}@{\hskip 0.01cm}
}
\includegraphics[width=0.7\textwidth,height=0.32\textheight]{imagens/CaloRinger_Analysis_ElectronVsJet/Efficiency/CaloRinger_Analysis_ElectronVsJet_comparison_tight_eff.pdf} \\
\includegraphics[width=0.7\textwidth,height=0.32\textheight]{imagens/CaloRinger_Analysis_ElectronVsJet/Efficiency/CaloRinger_Analysis_ElectronVsJet_comparison_tight_fa.pdf} \\
\end{tabular}
}
\label{fig:singlexj2_eficiencia_tight}
\caption{Comparações das eficiências em função das variáveis $\eta$ e $E_{T}$
para ambos algoritmos no requerimento \emph{Tight}. Conjunto Singlepart\_e x J2.}
\end{sidewaysfigure}



%--------------------------------------------------
\begin{sidewaysfigure}[phb]
\centering
\resizebox{\textwidth}{!}{
\begin{tabular}{
>{\centering\arraybackslash}m{0.10\textwidth} 
>{\centering\arraybackslash}m{0.225\textwidth}@{\hskip 0.01cm}
>{\centering\arraybackslash}m{0.225\textwidth}@{\hskip 0.01cm}
>{\centering\arraybackslash}m{0.225\textwidth}@{\hskip 0.01cm}
>{\centering\arraybackslash}m{0.225\textwidth}@{\hskip 0.01cm}
}
 & \textbf{Sem Req.} & \textbf{Loose} & \textbf{Medium} & \textbf{Tight} \\
\textbf{Single\linebreak part\_e}\linebreak (Barril) &  
\includegraphics[width=0.225\textwidth]{imagens/CaloRinger_Analysis_ElectronVsJet/CorrelationPlots_BC/Singlepart_e_noreq/CaloRinger_Analysis_ElectronVsJet_rcorexnnoutput_electron_bc_all.pdf} &
\includegraphics[width=0.225\textwidth]{imagens/CaloRinger_Analysis_ElectronVsJet/CorrelationPlots_BC/Singlepart_e_loose/CaloRinger_Analysis_ElectronVsJet_rcorexnnoutput_electron_bc_loose.pdf} &
\includegraphics[width=0.225\textwidth]{imagens/CaloRinger_Analysis_ElectronVsJet/CorrelationPlots_BC/Singlepart_e_medium/CaloRinger_Analysis_ElectronVsJet_rcorexnnoutput_electron_bc_medium.pdf} &
\includegraphics[width=0.225\textwidth]{imagens/CaloRinger_Analysis_ElectronVsJet/CorrelationPlots_BC/Singlepart_e_tight/CaloRinger_Analysis_ElectronVsJet_rcorexnnoutput_electron_bc_tight.pdf} \\
\textbf{J2} \linebreak (Barril)&  
\includegraphics[width=0.225\textwidth]{imagens/CaloRinger_Analysis_ElectronVsJet/CorrelationPlots_BC/J2_noreq/CaloRinger_Analysis_ElectronVsJet_rcorexnnoutput_jet_bc_all.pdf} &
\includegraphics[width=0.225\textwidth]{imagens/CaloRinger_Analysis_ElectronVsJet/CorrelationPlots_BC/J2_loose/CaloRinger_Analysis_ElectronVsJet_rcorexnnoutput_jet_bc_loose.pdf} &
\includegraphics[width=0.225\textwidth]{imagens/CaloRinger_Analysis_ElectronVsJet/CorrelationPlots_BC/J2_medium/CaloRinger_Analysis_ElectronVsJet_rcorexnnoutput_jet_bc_medium.pdf} &
\includegraphics[width=0.225\textwidth]{imagens/CaloRinger_Analysis_ElectronVsJet/CorrelationPlots_BC/J2_tight/CaloRinger_Analysis_ElectronVsJet_rcorexnnoutput_jet_bc_tight.pdf} \\
\textbf{Single\linebreak part\_e}\linebreak (Tampa) &  
\includegraphics[width=0.225\textwidth]{imagens/CaloRinger_Analysis_ElectronVsJet/CorrelationPlots_AC/Singlepart_e_noreq/CaloRinger_Analysis_ElectronVsJet_rcorexnnoutput_electron_ac_all.pdf} &
\includegraphics[width=0.225\textwidth]{imagens/CaloRinger_Analysis_ElectronVsJet/CorrelationPlots_AC/Singlepart_e_loose/CaloRinger_Analysis_ElectronVsJet_rcorexnnoutput_electron_ac_loose.pdf} &
\includegraphics[width=0.225\textwidth]{imagens/CaloRinger_Analysis_ElectronVsJet/CorrelationPlots_AC/Singlepart_e_medium/CaloRinger_Analysis_ElectronVsJet_rcorexnnoutput_electron_ac_medium.pdf} &
\includegraphics[width=0.225\textwidth]{imagens/CaloRinger_Analysis_ElectronVsJet/CorrelationPlots_AC/Singlepart_e_tight/CaloRinger_Analysis_ElectronVsJet_rcorexnnoutput_electron_ac_tight.pdf}
\\
\textbf{J2} \linebreak (Tampa)&  
\includegraphics[width=0.225\textwidth]{imagens/CaloRinger_Analysis_ElectronVsJet/CorrelationPlots_AC/J2_noreq/CaloRinger_Analysis_ElectronVsJet_rcorexnnoutput_jet_ac_all.pdf} &
\includegraphics[width=0.225\textwidth]{imagens/CaloRinger_Analysis_ElectronVsJet/CorrelationPlots_AC/J2_loose/CaloRinger_Analysis_ElectronVsJet_rcorexnnoutput_jet_ac_loose.pdf} &
\includegraphics[width=0.225\textwidth]{imagens/CaloRinger_Analysis_ElectronVsJet/CorrelationPlots_AC/J2_medium/CaloRinger_Analysis_ElectronVsJet_rcorexnnoutput_jet_ac_medium.pdf} &
\includegraphics[width=0.225\textwidth]{imagens/CaloRinger_Analysis_ElectronVsJet/CorrelationPlots_AC/J2_tight/CaloRinger_Analysis_ElectronVsJet_rcorexnnoutput_jet_ac_tight.pdf}
\\
\end{tabular}
}
\label{fig:singlexj2_rcore}
\caption{Correlações da saída neural para o conjunto Singlepart\_e x J2 com: rEta.}
\end{sidewaysfigure}

%--------------------------------------------------
\begin{sidewaysfigure}[phb]
\centering
\resizebox{\textwidth}{!}{
\begin{tabular}{
>{\centering\arraybackslash}m{0.10\textwidth} 
>{\centering\arraybackslash}m{0.225\textwidth}@{\hskip 0.01cm}
>{\centering\arraybackslash}m{0.225\textwidth}@{\hskip 0.01cm}
>{\centering\arraybackslash}m{0.225\textwidth}@{\hskip 0.01cm}
>{\centering\arraybackslash}m{0.225\textwidth}@{\hskip 0.01cm}
}
 & \textbf{Sem Req.} & \textbf{Loose} & \textbf{Medium} & \textbf{Tight} \\
\textbf{Single\linebreak part\_e}\linebreak (Barril) &  
\includegraphics[width=0.225\textwidth]{imagens/CaloRinger_Analysis_ElectronVsJet/CorrelationPlots_BC/Singlepart_e_noreq/CaloRinger_Analysis_ElectronVsJet_eratioxnnoutput_electron_bc_all.pdf} &
\includegraphics[width=0.225\textwidth]{imagens/CaloRinger_Analysis_ElectronVsJet/CorrelationPlots_BC/Singlepart_e_loose/CaloRinger_Analysis_ElectronVsJet_eratioxnnoutput_electron_bc_loose.pdf} &
\includegraphics[width=0.225\textwidth]{imagens/CaloRinger_Analysis_ElectronVsJet/CorrelationPlots_BC/Singlepart_e_medium/CaloRinger_Analysis_ElectronVsJet_eratioxnnoutput_electron_bc_medium.pdf} &
\includegraphics[width=0.225\textwidth]{imagens/CaloRinger_Analysis_ElectronVsJet/CorrelationPlots_BC/Singlepart_e_tight/CaloRinger_Analysis_ElectronVsJet_eratioxnnoutput_electron_bc_tight.pdf} \\
\textbf{J2} \linebreak (Barril)&  
\includegraphics[width=0.225\textwidth]{imagens/CaloRinger_Analysis_ElectronVsJet/CorrelationPlots_BC/J2_noreq/CaloRinger_Analysis_ElectronVsJet_eratioxnnoutput_jet_bc_all.pdf} &
\includegraphics[width=0.225\textwidth]{imagens/CaloRinger_Analysis_ElectronVsJet/CorrelationPlots_BC/J2_loose/CaloRinger_Analysis_ElectronVsJet_eratioxnnoutput_jet_bc_loose.pdf} &
\includegraphics[width=0.225\textwidth]{imagens/CaloRinger_Analysis_ElectronVsJet/CorrelationPlots_BC/J2_medium/CaloRinger_Analysis_ElectronVsJet_eratioxnnoutput_jet_bc_medium.pdf} &
\includegraphics[width=0.225\textwidth]{imagens/CaloRinger_Analysis_ElectronVsJet/CorrelationPlots_BC/J2_tight/CaloRinger_Analysis_ElectronVsJet_eratioxnnoutput_jet_bc_tight.pdf} \\
\textbf{Single\linebreak part\_e}\linebreak (Tampa) &  
\includegraphics[width=0.225\textwidth]{imagens/CaloRinger_Analysis_ElectronVsJet/CorrelationPlots_AC/Singlepart_e_noreq/CaloRinger_Analysis_ElectronVsJet_eratioxnnoutput_electron_ac_all.pdf} &
\includegraphics[width=0.225\textwidth]{imagens/CaloRinger_Analysis_ElectronVsJet/CorrelationPlots_AC/Singlepart_e_loose/CaloRinger_Analysis_ElectronVsJet_eratioxnnoutput_electron_ac_loose.pdf} &
\includegraphics[width=0.225\textwidth]{imagens/CaloRinger_Analysis_ElectronVsJet/CorrelationPlots_AC/Singlepart_e_medium/CaloRinger_Analysis_ElectronVsJet_eratioxnnoutput_electron_ac_medium.pdf} &
\includegraphics[width=0.225\textwidth]{imagens/CaloRinger_Analysis_ElectronVsJet/CorrelationPlots_AC/Singlepart_e_tight/CaloRinger_Analysis_ElectronVsJet_eratioxnnoutput_electron_ac_tight.pdf}
\\
\textbf{J2} \linebreak (Tampa)&  
\includegraphics[width=0.225\textwidth]{imagens/CaloRinger_Analysis_ElectronVsJet/CorrelationPlots_AC/J2_noreq/CaloRinger_Analysis_ElectronVsJet_eratioxnnoutput_jet_ac_all.pdf} &
\includegraphics[width=0.225\textwidth]{imagens/CaloRinger_Analysis_ElectronVsJet/CorrelationPlots_AC/J2_loose/CaloRinger_Analysis_ElectronVsJet_eratioxnnoutput_jet_ac_loose.pdf} &
\includegraphics[width=0.225\textwidth]{imagens/CaloRinger_Analysis_ElectronVsJet/CorrelationPlots_AC/J2_medium/CaloRinger_Analysis_ElectronVsJet_eratioxnnoutput_jet_ac_medium.pdf} &
\includegraphics[width=0.225\textwidth]{imagens/CaloRinger_Analysis_ElectronVsJet/CorrelationPlots_AC/J2_tight/CaloRinger_Analysis_ElectronVsJet_eratioxnnoutput_jet_ac_tight.pdf}
\\
\end{tabular}
}
\label{fig:singlexj2_eratio}
\caption{Correlações da saída neural para o conjunto Singlepart\_e x J2 com: eRatio.}
\end{sidewaysfigure}


%--------------------------------------------------
\begin{sidewaysfigure}[phb]
\centering
\resizebox{\textwidth}{!}{
\begin{tabular}{
>{\centering\arraybackslash}m{0.10\textwidth} 
>{\centering\arraybackslash}m{0.225\textwidth}@{\hskip 0.01cm}
>{\centering\arraybackslash}m{0.225\textwidth}@{\hskip 0.01cm}
>{\centering\arraybackslash}m{0.225\textwidth}@{\hskip 0.01cm}
>{\centering\arraybackslash}m{0.225\textwidth}@{\hskip 0.01cm}
}
 & \textbf{Sem Req.} & \textbf{Loose} & \textbf{Medium} & \textbf{Tight} \\
\textbf{Single\linebreak part\_e}\linebreak (Barril) &  
\includegraphics[width=0.225\textwidth]{imagens/CaloRinger_Analysis_ElectronVsJet/CorrelationPlots_BC/Singlepart_e_noreq/CaloRinger_Analysis_ElectronVsJet_hadleakagexnnoutput_electron_bc_all.pdf} &
\includegraphics[width=0.225\textwidth]{imagens/CaloRinger_Analysis_ElectronVsJet/CorrelationPlots_BC/Singlepart_e_loose/CaloRinger_Analysis_ElectronVsJet_hadleakagexnnoutput_electron_bc_loose.pdf} &
\includegraphics[width=0.225\textwidth]{imagens/CaloRinger_Analysis_ElectronVsJet/CorrelationPlots_BC/Singlepart_e_medium/CaloRinger_Analysis_ElectronVsJet_hadleakagexnnoutput_electron_bc_medium.pdf} &
\includegraphics[width=0.225\textwidth]{imagens/CaloRinger_Analysis_ElectronVsJet/CorrelationPlots_BC/Singlepart_e_tight/CaloRinger_Analysis_ElectronVsJet_hadleakagexnnoutput_electron_bc_tight.pdf} \\
\textbf{J2} \linebreak (Barril)&  
\includegraphics[width=0.225\textwidth]{imagens/CaloRinger_Analysis_ElectronVsJet/CorrelationPlots_BC/J2_noreq/CaloRinger_Analysis_ElectronVsJet_hadleakagexnnoutput_jet_bc_all.pdf} &
\includegraphics[width=0.225\textwidth]{imagens/CaloRinger_Analysis_ElectronVsJet/CorrelationPlots_BC/J2_loose/CaloRinger_Analysis_ElectronVsJet_hadleakagexnnoutput_jet_bc_loose.pdf} &
\includegraphics[width=0.225\textwidth]{imagens/CaloRinger_Analysis_ElectronVsJet/CorrelationPlots_BC/J2_medium/CaloRinger_Analysis_ElectronVsJet_hadleakagexnnoutput_jet_bc_medium.pdf} &
\includegraphics[width=0.225\textwidth]{imagens/CaloRinger_Analysis_ElectronVsJet/CorrelationPlots_BC/J2_tight/CaloRinger_Analysis_ElectronVsJet_hadleakagexnnoutput_jet_bc_tight.pdf} \\
\textbf{Single\linebreak part\_e}\linebreak (Tampa) &  
\includegraphics[width=0.225\textwidth]{imagens/CaloRinger_Analysis_ElectronVsJet/CorrelationPlots_AC/Singlepart_e_noreq/CaloRinger_Analysis_ElectronVsJet_hadleakagexnnoutput_electron_ac_all.pdf} &
\includegraphics[width=0.225\textwidth]{imagens/CaloRinger_Analysis_ElectronVsJet/CorrelationPlots_AC/Singlepart_e_loose/CaloRinger_Analysis_ElectronVsJet_hadleakagexnnoutput_electron_ac_loose.pdf} &
\includegraphics[width=0.225\textwidth]{imagens/CaloRinger_Analysis_ElectronVsJet/CorrelationPlots_AC/Singlepart_e_medium/CaloRinger_Analysis_ElectronVsJet_hadleakagexnnoutput_electron_ac_medium.pdf} &
\includegraphics[width=0.225\textwidth]{imagens/CaloRinger_Analysis_ElectronVsJet/CorrelationPlots_AC/Singlepart_e_tight/CaloRinger_Analysis_ElectronVsJet_hadleakagexnnoutput_electron_ac_tight.pdf}
\\
\textbf{J2} \linebreak (Tampa)&  
\includegraphics[width=0.225\textwidth]{imagens/CaloRinger_Analysis_ElectronVsJet/CorrelationPlots_AC/J2_noreq/CaloRinger_Analysis_ElectronVsJet_hadleakagexnnoutput_jet_ac_all.pdf} &
\includegraphics[width=0.225\textwidth]{imagens/CaloRinger_Analysis_ElectronVsJet/CorrelationPlots_AC/J2_loose/CaloRinger_Analysis_ElectronVsJet_hadleakagexnnoutput_jet_ac_loose.pdf} &
\includegraphics[width=0.225\textwidth]{imagens/CaloRinger_Analysis_ElectronVsJet/CorrelationPlots_AC/J2_medium/CaloRinger_Analysis_ElectronVsJet_hadleakagexnnoutput_jet_ac_medium.pdf} &
\includegraphics[width=0.225\textwidth]{imagens/CaloRinger_Analysis_ElectronVsJet/CorrelationPlots_AC/J2_tight/CaloRinger_Analysis_ElectronVsJet_hadleakagexnnoutput_jet_ac_tight.pdf}
\\
\end{tabular}
}
\label{fig:singlexj2_hadleakage}
\caption{Correlações da saída neural para o conjunto Singlepart\_e x J2 com: Rhad1.}
\end{sidewaysfigure}

%--------------------------------------------------
\begin{sidewaysfigure}[phb]
\centering
\resizebox{\textwidth}{!}{
\begin{tabular}{
>{\centering\arraybackslash}m{0.10\textwidth} 
>{\centering\arraybackslash}m{0.225\textwidth}@{\hskip 0.01cm}
>{\centering\arraybackslash}m{0.225\textwidth}@{\hskip 0.01cm}
>{\centering\arraybackslash}m{0.225\textwidth}@{\hskip 0.01cm}
>{\centering\arraybackslash}m{0.225\textwidth}@{\hskip 0.01cm}
}
 & \textbf{Sem Req.} & \textbf{Loose} & \textbf{Medium} & \textbf{Tight} \\
\textbf{Single\linebreak part\_e}\linebreak (Barril) &  
\includegraphics[width=0.225\textwidth]{imagens/CaloRinger_Analysis_ElectronVsJet/CorrelationPlots_BC/Singlepart_e_noreq/CaloRinger_Analysis_ElectronVsJet_width2xnnoutput_electron_bc_all.pdf} &
\includegraphics[width=0.225\textwidth]{imagens/CaloRinger_Analysis_ElectronVsJet/CorrelationPlots_BC/Singlepart_e_loose/CaloRinger_Analysis_ElectronVsJet_width2xnnoutput_electron_bc_loose.pdf} &
\includegraphics[width=0.225\textwidth]{imagens/CaloRinger_Analysis_ElectronVsJet/CorrelationPlots_BC/Singlepart_e_medium/CaloRinger_Analysis_ElectronVsJet_width2xnnoutput_electron_bc_medium.pdf} &
\includegraphics[width=0.225\textwidth]{imagens/CaloRinger_Analysis_ElectronVsJet/CorrelationPlots_BC/Singlepart_e_tight/CaloRinger_Analysis_ElectronVsJet_width2xnnoutput_electron_bc_tight.pdf} \\
\textbf{J2} \linebreak (Barril)&  
\includegraphics[width=0.225\textwidth]{imagens/CaloRinger_Analysis_ElectronVsJet/CorrelationPlots_BC/J2_noreq/CaloRinger_Analysis_ElectronVsJet_width2xnnoutput_jet_bc_all.pdf} &
\includegraphics[width=0.225\textwidth]{imagens/CaloRinger_Analysis_ElectronVsJet/CorrelationPlots_BC/J2_loose/CaloRinger_Analysis_ElectronVsJet_width2xnnoutput_jet_bc_loose.pdf} &
\includegraphics[width=0.225\textwidth]{imagens/CaloRinger_Analysis_ElectronVsJet/CorrelationPlots_BC/J2_medium/CaloRinger_Analysis_ElectronVsJet_width2xnnoutput_jet_bc_medium.pdf} &
\includegraphics[width=0.225\textwidth]{imagens/CaloRinger_Analysis_ElectronVsJet/CorrelationPlots_BC/J2_tight/CaloRinger_Analysis_ElectronVsJet_width2xnnoutput_jet_bc_tight.pdf} \\
\textbf{Single\linebreak part\_e}\linebreak (Tampa) &  
\includegraphics[width=0.225\textwidth]{imagens/CaloRinger_Analysis_ElectronVsJet/CorrelationPlots_AC/Singlepart_e_noreq/CaloRinger_Analysis_ElectronVsJet_width2xnnoutput_electron_ac_all.pdf} &
\includegraphics[width=0.225\textwidth]{imagens/CaloRinger_Analysis_ElectronVsJet/CorrelationPlots_AC/Singlepart_e_loose/CaloRinger_Analysis_ElectronVsJet_width2xnnoutput_electron_ac_loose.pdf} &
\includegraphics[width=0.225\textwidth]{imagens/CaloRinger_Analysis_ElectronVsJet/CorrelationPlots_AC/Singlepart_e_medium/CaloRinger_Analysis_ElectronVsJet_width2xnnoutput_electron_ac_medium.pdf} &
\includegraphics[width=0.225\textwidth]{imagens/CaloRinger_Analysis_ElectronVsJet/CorrelationPlots_AC/Singlepart_e_tight/CaloRinger_Analysis_ElectronVsJet_width2xnnoutput_electron_ac_tight.pdf}
\\
\textbf{J2} \linebreak (Tampa)&  
\includegraphics[width=0.225\textwidth]{imagens/CaloRinger_Analysis_ElectronVsJet/CorrelationPlots_AC/J2_noreq/CaloRinger_Analysis_ElectronVsJet_width2xnnoutput_jet_ac_all.pdf} &
\includegraphics[width=0.225\textwidth]{imagens/CaloRinger_Analysis_ElectronVsJet/CorrelationPlots_AC/J2_loose/CaloRinger_Analysis_ElectronVsJet_width2xnnoutput_jet_ac_loose.pdf} &
\includegraphics[width=0.225\textwidth]{imagens/CaloRinger_Analysis_ElectronVsJet/CorrelationPlots_AC/J2_medium/CaloRinger_Analysis_ElectronVsJet_width2xnnoutput_jet_ac_medium.pdf} &
\includegraphics[width=0.225\textwidth]{imagens/CaloRinger_Analysis_ElectronVsJet/CorrelationPlots_AC/J2_tight/CaloRinger_Analysis_ElectronVsJet_width2xnnoutput_jet_ac_tight.pdf}
\\
\end{tabular}
}
\label{fig:singlexj2_width2}
\caption{Correlações da saída neural para o conjunto Singlepart\_e x J2 com: wEta2.}
\end{sidewaysfigure}

\FloatBarrier

\subsection{\texorpdfstring{$\text{J}/\Psi \times$ Minbias}{JPsi x Minbias}}
\label{ssec:jpsi}

Deve-se levantar algumas diferenças entre o conjunto de Singlepart\_e x J2 para
o conjunto atual. Primeiro, a sobreposição da faixa de energia nos dados
acontece para toda ela, como pode ser visto na
Figura~\ref{fig:jpsixminb_dist_energia}. Além disso, agora o conjunto de sinal
não contém apenas elétrons, mas também simulações de \gls{mbias}, de forma que o
conjunto está contaminado por partículas constituíntes de ruído. Também, não é
garantido que os elétrons estarão isolados, podendo ter parte da sua deposição
de energia contaminada por outra partícula, dificultando o processo de
discriminação uma vez que o outro deposito irá fazer partículas \gls{em}
perderem suas características deposição de energia em regiões mais estreitas.
Quanto ao conjunto de ruído composto por \gls{mbias}, diferente do conjunto de
J2, onde todas as partículas são compostas por jatos de seus decaimentos, podem
estar contidos nesse conjunto elétrons legítmos, de forma que ambos os conjuntos
podem conter partículas que, teoricamente, eram para pertencer ao conjunto
oposto. O procedimento é o mesmo que aquele realizado para o conjunto anterior, entretanto
essas diferenças serão importantes para determinar as diferenças encontradas
durante a análise.

\begin{figure}[ht]
\centering
\includegraphics[width=.7\textwidth]{imagens/CaloRinger_Analysis_JpsiVsMinBias/EnergyDistribution/CaloRinger_Analysis_JpsiVsMinBiasdata_energy_dist.pdf}
\label{fig:jpsixminb_dist_energia}
\caption{Distribuição de energia para o conjunto JPsi x Minbias.}
\end{figure}

\begin{table}[htb]
\centering
\begin{tabular}{cccc}
\hline
\hline
 & 
\multicolumn{2}{c}{DET (\%) para algoritmo} & 
\\
\cline{2-3}
\multirow{-2}{*}{Req. Do Alg. Padrão} & 
Alg. Padrão & 
EgCaloRinger & 
\multirow{-2}{*}{para FA (\%)} \\
\hline
Loose  & 62.89 & 86.52 & 54.22\\
Medium & 43.89 & 77.51 & 29.17\\
Tight  & 33.70 & 34.51 & 4.48 \\
\hline
\hline
\end{tabular}
\caption{Eficiências do algoritmos para FA fixos para o conjunto JPsi x Minbias.}
\label{tab:jpsixminb_efic}
\end{table}


A Tabela~\ref{tab:jpsixminb_efic} contém as taxas de deteção de ambos os
algoritmos para falso alarme fixo nos valores do Algoritmo Padrão. O
\gls{egcaloringer} obtém valores consideravelmente maiores que os do Algoritmo
Padrão para os cortes \emph{Loose}, \emph{Medium}, contudo os algoritmos
obtiveram eficiência equivalente para o \emph{Tight}. Todavia, esses valores não
devem ser considerados sem um estudo mais profundo, uma vez que não se sabe,
ainda, quais são as partículas que o \gls{egcaloringer} está considerando como
sinal, enquanto o Algoritmo Padrão rejeita. As taxas de detecção e falso alarme 
para os algoritmos e seus requerimentos estão na Tabelas~\ref{tab:jpsixminb_efic_det} 
e \ref{tab:jpsixminb_efic_fa}. Diferente do conjunto de elétrons isolados, há
uma quantidade de elétrons \emph{Tight} do Algoritmo Padrão que estão sendo
rejeitados pelo requerimento \emph{\gls{rna} Tight}. Por outro lado o
\emph{\gls{rna} Medium} está considerando boa parte dos elétrons selecionados
pelo critério \emph{Medium} e \emph{Tight} do Algoritmo padrão. Por sua vez, as
taxas de rejeição do algoritmo proposto são menores para os requerimentos
\emph{\gls{rna} Loose} e \emph{\gls{rna} Medium} quando em comparação com os
respectivos requerimentos do Algoritmo Padrão, enquanto o oposto ocorre para o
\emph{\gls{rna} Tight}.


\begin{table}[htb]
\centering
\begin{tabular}{l cccc}
\hline
\hline
DET (\%)& Todo o Conj. & Loose & Medium & Tight \\
\hline
Todo o Conj.      &     -  &  62.89      & 43.89       &  33.70      \\
\hline
\gls{rna} Loose   &  80.00 & 57.49/62.89 & 42.61/43.89 & 33.08/33.70 \\
\hline
\gls{rna} Medium  &  70.39 & 51.74/62.89 & 40.04/43.89 & 31.36/33.70 \\
\hline
\gls{rna} Tight   &  52.72 & 39.99/62.89 & 32.71/43.89 & 26.14/33.70 \\
\hline
\hline
\end{tabular}
\caption{Taxa de detecção (\%) dos algoritmos e parcela na qual o EgCaloRinger
identifica em comum ao algoritmo padrão. Conjunto JPsi x Minbias.}
\label{tab:jpsixminb_efic_det}
\end{table}


\begin{table}[htb]
\centering
\begin{tabular}{l cccc}
\hline
\hline
FA (\%)& Todo o Conj. & Loose & Medium & Tight \\
\hline
Todo o Conj.    &  -    & 54.23       & 29.18       & 4.48      \\
\gls{rna} Loose & 34.32 & 27.54/54.23 & 14.18/29.18 & 2.70/4.48 \\
\gls{rna} Medium& 19.67 & 16.85/54.23 & 8.34/29.18  & 1.82/4.48 \\
\gls{rna} Tight & 9.64  & 8.53/54.23  & 4.08/29.18  & 1.06/4.48 \\
\hline
\hline
\end{tabular}
\caption{Taxa de falso alarme (\%) dos algoritmos e parcela na qual o EgCaloRinger
identifica em comum ao algoritmo padrão. Conjunto JPsi x Minbias.}
\label{tab:jpsixminb_efic_fa}
\end{table}

Quanto a Figura~\ref{fig:jpsixminb_saidaneural}, pode-se reparar que a saída da
rede neural já não está mais tão concentrada nos extremos, como quando no
conjunto de Singlepart\_e x J2, mostrando o impacto da contaminação dos dados no
ajuste da rede neural, que irá deslocar os picos de concentração de saídas para
uma região ligeraimente mais central. Contudo, a rede ainda foi capaz de
concentrar grande parte dos elétrons \emph{Medium} e \emph{Tight} do Algoritmo
Padrão na região do pico correspondente a elétrons na saída neural. Por outro
lado, quando observando a saída neural para o conjunto de Minbias, observa-se
uma distribuição praticamente uniforme para os Minbias que passaram o critério
\emph{Tight}. A distribuição gradualmente se concentra no pico destinado a jatos
da saída neural conforme os requerimentos do Algoritmo Padrão são relaxados,
até o momento que se chega aos jatos sem requerimento, que nitidamente se agrupam 
no pico destinados a jatos.

\begin{figure}[ht]
\centering
\resizebox{\textwidth}{!}{
\begin{tabular}{
>{\centering\arraybackslash}m{0.50\textwidth}@{\hskip 0.5cm}
>{\centering\arraybackslash}m{0.50\textwidth}@{\hskip 0.5cm}
}
\includegraphics[width=0.5\textwidth]{imagens/CaloRinger_Analysis_JpsiVsMinBias/NeuralOutput/CaloRinger_Analysis_JpsiVsMinBias_nnoutput_jpsi.pdf}
&
\includegraphics[width=0.5\textwidth]{imagens/CaloRinger_Analysis_JpsiVsMinBias/NeuralOutput/CaloRinger_Analysis_JpsiVsMinBias_nnoutput_minbias.pdf}
\\
(a) \textbf{Elétrons} & (b) \textbf{Jatos} \\
\end{tabular}
}
\label{fig:jpsixminb_saidaneural}
\caption{Saída da rede neural e suas relações com os requerimentos do algoritmo
eGamma padrão. Conjunto JPsi x Minbias.}
\end{figure}

\begin{figure}[ht]
\centering
\includegraphics[width=.7\textwidth]{imagens/CaloRinger_Analysis_JpsiVsMinBias/Efficiency/CaloRinger_Analysis_JpsiVsMinBias_roc.pdf}
\label{fig:jpsixminb_roc}
\caption{Curva ROC para o conjunto JPsi x Minbias.}
\end{figure}

Na curva~\gls{roc} fica nítido a informação contida na
Tabela~\ref{tab:jpsixminb_efic}, aonde a eficiência da rede para a taxa de falso
alarme do requerimento \emph{\gls{rna} Tight} é praticamente a mesma daquela do Algoritmo
Padrão no seu respetivo requerimento, enquanto os critérios \emph{Medium} e
\emph{Loose} obtém valores mais elevados, entretanto, sem se saber realmente
quais são as partículas que estão sendo selecionadas. A
Tabela~\ref{tab:jpsixminb_part_jpsi} está preenchida com as taxas de rejeição de
algumas partículas estáveis contidas no conjunto de dados de JPsi. A rede neural
está rejeitando menos elétrons em todos os cortes, que na lógica invertida, 
significaria uma melhor eficiência de detecção, demonstrando que as maiores taxas
de detecção obtidas contêm, sim, uma maior quantidade de elétrons que aqueles
selecionados pelo Algoritmo Padrão. Por outro lado, as taxas de rejeição de
hádrons e fótons -- que aqui novamente devem ser resultados de decaimentos da
partícula $\pi^0$ -- são menores que aquelas do Algoritmo Padrão. Esse fator
pode ser corrigido ao se filtrar essas partículas que estão contaminando o
conjunto de sinal, alterando o alvo das mesmas para jatos, mas, ainda assim é
interessante notar que a rede neural foi capaz de rejeitar grande parte
dessas partículas mesmo que o alvo delas tenha sido como de elétrons,
demonstrando que os dados contidos no conjunto de \gls{mbias} deslocaram essas
partículas para a região de ruído por conterem características similares. O
mesmo estudo foi realizado para o conjunto de \gls{mbias}, estando na
Tabela~\ref{tab:jpsixminb_part_minb}. Observa-se que o critério \emph{\gls{rna}
Tight} não obteve o mesmo poder de rejeição que o critério \emph{Tight} do
Algoritmo Padrão, mas que pode ser compensado pelo ganho obtido na taxa de
detecção. Por outro lado, os critérios \emph{\gls{rna} Loose} e \emph{\gls{rna} Medium} 
foram capazes de rejeitar uma maior porção dos hádrons. Ao mesmo tempo, foram
perdidos elétrons contidos nesse conjunto, que compõe uma quantidade porção
pequena das partículas nesse conjunto, podendo ser corrigido através da
alteração do alvo dessas partículas.



\begin{table}[htb]
\centering
\resizebox{\textwidth}{!}{
\begin{tabular}{m{4cm} ccc ccc}
\hline
\hline
\multirow{2}{4cm}{Taxa de Rejeição (\%) para partícula estável} &
\multicolumn{3}{c}{Alg. Padrão} & \multicolumn{3}{c}{CaloRinger} \\
 & \textbf{Loose} & \textbf{Medium} & \textbf{Tight} & \textbf{gls{rna} Loose} &
\textbf{gls{rna} Medium} & \textbf{gls{rna} Tight} \\
\hline
$\gamma$      &64.52&91.63&95.39&67.37&79.32&88.86\\
$\pi^+/\pi^-$ &73.91&89.36&94.18&79.27&86.25&91.46\\
$K^+/K^-$     &78.37&91.78&95.76&82.57&89.45&93.74\\
$K^0_l$       &69.17&87.97&93.23&75.19&81.95&89.47\\
$K^0_s$       &71.45&91.00&93.61&79.66&86.31&92.57\\
$e^+/e^-$     &30.02&42.65&48.92&4.92 &13.82&33.95\\
\hline
\hline
\end{tabular}
}
\caption{Rejeição de partículas estáveis para o conjunto de JPsi.}
\label{tab:jpsixminb_part_jpsi}
\end{table}

\begin{table}[htb]
\centering
\resizebox{\textwidth}{!}{
\begin{tabular}{m{4cm} ccc ccc}
\hline
\hline
\multirow{2}{4cm}{Taxa de Rejeição (\%) para partícula estável} &
\multicolumn{3}{c}{Alg. Padrão} & \multicolumn{3}{c}{CaloRinger} \\
 & \textbf{Loose} & \textbf{Medium} & \textbf{Tight} & \textbf{\gls{rna} Loose} &
\textbf{\gls{rna} Medium} & \textbf{\gls{rna} Tight} \\
\hline
$\gamma$      &50.44 &77.66 &97.09 &61.85 &76.89 &88.45  \\
$\pi^+/\pi^-$ &53.47 &66.21 &95.63 &79.02 &90.10 &96.04  \\
$K^+/K^-$     &58.51 &69.84 &96.84 &83.05 &91.82 &96.83  \\
$K^0_l$       &63.24 &79.41 &95.34 &82.11 &91.67 &96.57  \\
$K^0_s$       &64.70 &82.41 &98.11 &84.53 &93.07 &97.44  \\
$e^+/e^-$     &20.07 &25.69 &35.82 &25.95 &41.70 &61.24  \\
\hline
\hline
\end{tabular}
}
\caption{Rejeição de partículas estáveis para o conjunto de Minbias.}
\label{tab:jpsixminb_part_minb}
\end{table}


As eficiências de taxa de deteção e falso alarme em função das variáveis estão
dispostas nas Figuras~\ref{fig:jpsixminb_eficiencia_tight},
\ref{fig:jpsixminb_eficiencia_medium} e \ref{fig:jpsixminb_eficiencia_loose},
para os cortes \emph{Tight}, \emph{Medium} e \emph{Loose}. Os valores do
\gls{egcaloringer} flutuam seguindo as tendências do Algoritmo Padrão, obtendo
eficiências melhores ou piores para cada requerimento dependendo do valor médio de 
eficiência disposto nas Tabelas~\ref{tab:jpsixminb_efic_det} e
\ref{tab:jpsixminb_efic_fa}.

Os histogramas de duas dimensões para as variáveis físicas e saída neural seguem
o que foi dito para a análise realizada para o conjuto Singlepart\_e x J2, onde
a região de confusão fica relativamente próximo dos limiares de corte utilizados
pelo Algoritmo Padrão, região aonde os jatos mascaram os elétrons.
Contudo, a concentração dos dados não é na borda como no conjunto anterior, 
reflexo do deslocamento dos picos observados na Figura~\ref{fig:jpsixminb_saidaneural}.


%--------------------------------------------------
\begin{sidewaysfigure}[phb]
\centering
\resizebox{\textwidth}{!}{
\begin{tabular}{
>{\centering\arraybackslash}m{\textwidth}@{\hskip 0.01cm}
}
\includegraphics[width=0.7\textwidth,height=0.32\textheight]{imagens/CaloRinger_Analysis_JpsiVsMinBias/Efficiency/CaloRinger_Analysis_JpsiVsMinBias_comparison_loose_eff.pdf} \\
\includegraphics[width=0.7\textwidth,height=0.32\textheight]{imagens/CaloRinger_Analysis_JpsiVsMinBias/Efficiency/CaloRinger_Analysis_JpsiVsMinBias_comparison_loose_fa.pdf} \\
\end{tabular}
}
\label{fig:jpsixminb_eficiencia_tight}
\caption{Comparações das eficiências em função das variáveis $\eta$ e $E_{T}$
para ambos algoritmos no requerimento \emph{Loose}. Conjunto JPsi x Minbias.}
\end{sidewaysfigure}

%--------------------------------------------------
\begin{sidewaysfigure}[phb]
\centering
\resizebox{\textwidth}{!}{
\begin{tabular}{
>{\centering\arraybackslash}m{\textwidth}@{\hskip 0.01cm}
}
\includegraphics[width=0.7\textwidth,height=0.32\textheight]{imagens/CaloRinger_Analysis_JpsiVsMinBias/Efficiency/CaloRinger_Analysis_JpsiVsMinBias_comparison_medium_eff.pdf} \\
\includegraphics[width=0.7\textwidth,height=0.32\textheight]{imagens/CaloRinger_Analysis_JpsiVsMinBias/Efficiency/CaloRinger_Analysis_JpsiVsMinBias_comparison_medium_fa.pdf} \\
\end{tabular}
}
\label{fig:jpsixminb_eficiencia_medium}
\caption{Comparações das eficiências em função das variáveis $\eta$ e $E_{T}$
para ambos algoritmos no requerimento \emph{Medium}. Conjunto JPsi x Minbias.}
\end{sidewaysfigure}

%--------------------------------------------------
\begin{sidewaysfigure}[phb]
\centering
\resizebox{\textwidth}{!}{
\begin{tabular}{
>{\centering\arraybackslash}m{\textwidth}@{\hskip 0.01cm}
}
\includegraphics[width=0.7\textwidth,height=0.32\textheight]{imagens/CaloRinger_Analysis_JpsiVsMinBias/Efficiency/CaloRinger_Analysis_JpsiVsMinBias_comparison_tight_eff.pdf} \\
\includegraphics[width=0.7\textwidth,height=0.32\textheight]{imagens/CaloRinger_Analysis_JpsiVsMinBias/Efficiency/CaloRinger_Analysis_JpsiVsMinBias_comparison_tight_fa.pdf} \\
\end{tabular}
}
\label{fig:jpsixminb_eficiencia_loose}
\caption{Comparações das eficiências em função das variáveis $\eta$ e $E_{T}$
para ambos algoritmos no requerimento \emph{Medium}. Conjunto JPsi x Minbias.}
\end{sidewaysfigure}

%--------------------------------------------------
\begin{sidewaysfigure}[phb]
\centering
\resizebox{\textwidth}{!}{
\begin{tabular}{
>{\centering\arraybackslash}m{0.10\textwidth} 
>{\centering\arraybackslash}m{0.225\textwidth}@{\hskip 0.01cm}
>{\centering\arraybackslash}m{0.225\textwidth}@{\hskip 0.01cm}
>{\centering\arraybackslash}m{0.225\textwidth}@{\hskip 0.01cm}
>{\centering\arraybackslash}m{0.225\textwidth}@{\hskip 0.01cm}
}
 & \textbf{Sem Req.} & \textbf{Loose} & \textbf{Medium} & \textbf{Tight} \\
\textbf{Single\linebreak part\_e}\linebreak (Barril) &  
\includegraphics[width=0.225\textwidth]{imagens/CaloRinger_Analysis_JpsiVsMinBias/CorrelationPlots_BC/Jpsi_3e3_noreq/CaloRinger_Analysis_JpsiVsMinBias_rcorexnnoutput_jpsi_bc_all.pdf} &
\includegraphics[width=0.225\textwidth]{imagens/CaloRinger_Analysis_JpsiVsMinBias/CorrelationPlots_BC/Jpsi_3e3_loose/CaloRinger_Analysis_JpsiVsMinBias_rcorexnnoutput_jpsi_bc_loose.pdf} &
\includegraphics[width=0.225\textwidth]{imagens/CaloRinger_Analysis_JpsiVsMinBias/CorrelationPlots_BC/Jpsi_3e3_medium/CaloRinger_Analysis_JpsiVsMinBias_rcorexnnoutput_jpsi_bc_medium.pdf} &
\includegraphics[width=0.225\textwidth]{imagens/CaloRinger_Analysis_JpsiVsMinBias/CorrelationPlots_BC/Jpsi_3e3_tight/CaloRinger_Analysis_JpsiVsMinBias_rcorexnnoutput_jpsi_bc_tight.pdf} \\
\textbf{Minbias} \linebreak (Barril)&  
\includegraphics[width=0.225\textwidth]{imagens/CaloRinger_Analysis_JpsiVsMinBias/CorrelationPlots_BC/Minbias_noreq/CaloRinger_Analysis_JpsiVsMinBias_rcorexnnoutput_minb_bc_all.pdf} &
\includegraphics[width=0.225\textwidth]{imagens/CaloRinger_Analysis_JpsiVsMinBias/CorrelationPlots_BC/Minbias_loose/CaloRinger_Analysis_JpsiVsMinBias_rcorexnnoutput_minb_bc_loose.pdf} &
\includegraphics[width=0.225\textwidth]{imagens/CaloRinger_Analysis_JpsiVsMinBias/CorrelationPlots_BC/Minbias_medium/CaloRinger_Analysis_JpsiVsMinBias_rcorexnnoutput_minb_bc_medium.pdf} &
\includegraphics[width=0.225\textwidth]{imagens/CaloRinger_Analysis_JpsiVsMinBias/CorrelationPlots_BC/Minbias_tight/CaloRinger_Analysis_JpsiVsMinBias_rcorexnnoutput_minb_bc_tight.pdf} \\
\textbf{Single\linebreak part\_e}\linebreak (Tampa) &  
\includegraphics[width=0.225\textwidth]{imagens/CaloRinger_Analysis_JpsiVsMinBias/CorrelationPlots_AC/Jpsi_3e3_noreq/CaloRinger_Analysis_JpsiVsMinBias_rcorexnnoutput_jpsi_ac_all.pdf} &
\includegraphics[width=0.225\textwidth]{imagens/CaloRinger_Analysis_JpsiVsMinBias/CorrelationPlots_AC/Jpsi_3e3_loose/CaloRinger_Analysis_JpsiVsMinBias_rcorexnnoutput_jpsi_ac_loose.pdf} &
\includegraphics[width=0.225\textwidth]{imagens/CaloRinger_Analysis_JpsiVsMinBias/CorrelationPlots_AC/Jpsi_3e3_medium/CaloRinger_Analysis_JpsiVsMinBias_rcorexnnoutput_jpsi_ac_medium.pdf} &
\includegraphics[width=0.225\textwidth]{imagens/CaloRinger_Analysis_JpsiVsMinBias/CorrelationPlots_AC/Jpsi_3e3_tight/CaloRinger_Analysis_JpsiVsMinBias_rcorexnnoutput_jpsi_ac_tight.pdf}
\\
\textbf{Minbias} \linebreak (Tampa)&  
\includegraphics[width=0.225\textwidth]{imagens/CaloRinger_Analysis_JpsiVsMinBias/CorrelationPlots_AC/Minbias_noreq/CaloRinger_Analysis_JpsiVsMinBias_rcorexnnoutput_minb_ac_all.pdf} &
\includegraphics[width=0.225\textwidth]{imagens/CaloRinger_Analysis_JpsiVsMinBias/CorrelationPlots_AC/Minbias_loose/CaloRinger_Analysis_JpsiVsMinBias_rcorexnnoutput_minb_ac_loose.pdf} &
\includegraphics[width=0.225\textwidth]{imagens/CaloRinger_Analysis_JpsiVsMinBias/CorrelationPlots_AC/Minbias_medium/CaloRinger_Analysis_JpsiVsMinBias_rcorexnnoutput_minb_ac_medium.pdf} &
\includegraphics[width=0.225\textwidth]{imagens/CaloRinger_Analysis_JpsiVsMinBias/CorrelationPlots_AC/Minbias_tight/CaloRinger_Analysis_JpsiVsMinBias_rcorexnnoutput_minb_ac_tight.pdf}
\\
\end{tabular}
}
\label{fig:jpsixminb_rcore}
\caption{Correlações da saída neural para o conjunto JPsi x Minbias com:
rEta.}
\end{sidewaysfigure}

%--------------------------------------------------
\begin{sidewaysfigure}[phb]
\centering
\resizebox{\textwidth}{!}{
\begin{tabular}{
>{\centering\arraybackslash}m{0.10\textwidth} 
>{\centering\arraybackslash}m{0.225\textwidth}@{\hskip 0.01cm}
>{\centering\arraybackslash}m{0.225\textwidth}@{\hskip 0.01cm}
>{\centering\arraybackslash}m{0.225\textwidth}@{\hskip 0.01cm}
>{\centering\arraybackslash}m{0.225\textwidth}@{\hskip 0.01cm}
}
 & \textbf{Sem Req.} & \textbf{Loose} & \textbf{Medium} & \textbf{Tight} \\
\textbf{Single\linebreak part\_e}\linebreak (Barril) &  
\includegraphics[width=0.225\textwidth]{imagens/CaloRinger_Analysis_JpsiVsMinBias/CorrelationPlots_BC/Jpsi_3e3_noreq/CaloRinger_Analysis_JpsiVsMinBias_eratioxnnoutput_jpsi_bc_all.pdf} &
\includegraphics[width=0.225\textwidth]{imagens/CaloRinger_Analysis_JpsiVsMinBias/CorrelationPlots_BC/Jpsi_3e3_loose/CaloRinger_Analysis_JpsiVsMinBias_eratioxnnoutput_jpsi_bc_loose.pdf} &
\includegraphics[width=0.225\textwidth]{imagens/CaloRinger_Analysis_JpsiVsMinBias/CorrelationPlots_BC/Jpsi_3e3_medium/CaloRinger_Analysis_JpsiVsMinBias_eratioxnnoutput_jpsi_bc_medium.pdf} &
\includegraphics[width=0.225\textwidth]{imagens/CaloRinger_Analysis_JpsiVsMinBias/CorrelationPlots_BC/Jpsi_3e3_tight/CaloRinger_Analysis_JpsiVsMinBias_eratioxnnoutput_jpsi_bc_tight.pdf} \\
\textbf{Minbias} \linebreak (Barril)&  
\includegraphics[width=0.225\textwidth]{imagens/CaloRinger_Analysis_JpsiVsMinBias/CorrelationPlots_BC/Minbias_noreq/CaloRinger_Analysis_JpsiVsMinBias_eratioxnnoutput_minb_bc_all.pdf} &
\includegraphics[width=0.225\textwidth]{imagens/CaloRinger_Analysis_JpsiVsMinBias/CorrelationPlots_BC/Minbias_loose/CaloRinger_Analysis_JpsiVsMinBias_eratioxnnoutput_minb_bc_loose.pdf} &
\includegraphics[width=0.225\textwidth]{imagens/CaloRinger_Analysis_JpsiVsMinBias/CorrelationPlots_BC/Minbias_medium/CaloRinger_Analysis_JpsiVsMinBias_eratioxnnoutput_minb_bc_medium.pdf} &
\includegraphics[width=0.225\textwidth]{imagens/CaloRinger_Analysis_JpsiVsMinBias/CorrelationPlots_BC/Minbias_tight/CaloRinger_Analysis_JpsiVsMinBias_eratioxnnoutput_minb_bc_tight.pdf} \\
\textbf{Single\linebreak part\_e}\linebreak (Tampa) &  
\includegraphics[width=0.225\textwidth]{imagens/CaloRinger_Analysis_JpsiVsMinBias/CorrelationPlots_AC/Jpsi_3e3_noreq/CaloRinger_Analysis_JpsiVsMinBias_eratioxnnoutput_jpsi_ac_all.pdf} &
\includegraphics[width=0.225\textwidth]{imagens/CaloRinger_Analysis_JpsiVsMinBias/CorrelationPlots_AC/Jpsi_3e3_loose/CaloRinger_Analysis_JpsiVsMinBias_eratioxnnoutput_jpsi_ac_loose.pdf} &
\includegraphics[width=0.225\textwidth]{imagens/CaloRinger_Analysis_JpsiVsMinBias/CorrelationPlots_AC/Jpsi_3e3_medium/CaloRinger_Analysis_JpsiVsMinBias_eratioxnnoutput_jpsi_ac_medium.pdf} &
\includegraphics[width=0.225\textwidth]{imagens/CaloRinger_Analysis_JpsiVsMinBias/CorrelationPlots_AC/Jpsi_3e3_tight/CaloRinger_Analysis_JpsiVsMinBias_eratioxnnoutput_jpsi_ac_tight.pdf}
\\
\textbf{Minbias} \linebreak (Tampa)&  
\includegraphics[width=0.225\textwidth]{imagens/CaloRinger_Analysis_JpsiVsMinBias/CorrelationPlots_AC/Minbias_noreq/CaloRinger_Analysis_JpsiVsMinBias_eratioxnnoutput_minb_ac_all.pdf} &
\includegraphics[width=0.225\textwidth]{imagens/CaloRinger_Analysis_JpsiVsMinBias/CorrelationPlots_AC/Minbias_loose/CaloRinger_Analysis_JpsiVsMinBias_eratioxnnoutput_minb_ac_loose.pdf} &
\includegraphics[width=0.225\textwidth]{imagens/CaloRinger_Analysis_JpsiVsMinBias/CorrelationPlots_AC/Minbias_medium/CaloRinger_Analysis_JpsiVsMinBias_eratioxnnoutput_minb_ac_medium.pdf} &
\includegraphics[width=0.225\textwidth]{imagens/CaloRinger_Analysis_JpsiVsMinBias/CorrelationPlots_AC/Minbias_tight/CaloRinger_Analysis_JpsiVsMinBias_eratioxnnoutput_minb_ac_tight.pdf}
\\
\end{tabular}
}
\label{fig:jpsixminb_eratio}
\caption{Correlações da saída neural para o conjunto JPsi x Minbias com:
eRatio.}
\end{sidewaysfigure}

%--------------------------------------------------
\begin{sidewaysfigure}[phb]
\centering
\resizebox{\textwidth}{!}{
\begin{tabular}{
>{\centering\arraybackslash}m{0.10\textwidth} 
>{\centering\arraybackslash}m{0.225\textwidth}@{\hskip 0.01cm}
>{\centering\arraybackslash}m{0.225\textwidth}@{\hskip 0.01cm}
>{\centering\arraybackslash}m{0.225\textwidth}@{\hskip 0.01cm}
>{\centering\arraybackslash}m{0.225\textwidth}@{\hskip 0.01cm}
}
 & \textbf{Sem Req.} & \textbf{Loose} & \textbf{Medium} & \textbf{Tight} \\
\textbf{Single\linebreak part\_e}\linebreak (Barril) &  
\includegraphics[width=0.225\textwidth]{imagens/CaloRinger_Analysis_JpsiVsMinBias/CorrelationPlots_BC/Jpsi_3e3_noreq/CaloRinger_Analysis_JpsiVsMinBias_hadleakagexnnoutput_jpsi_bc_all.pdf} &
\includegraphics[width=0.225\textwidth]{imagens/CaloRinger_Analysis_JpsiVsMinBias/CorrelationPlots_BC/Jpsi_3e3_loose/CaloRinger_Analysis_JpsiVsMinBias_hadleakagexnnoutput_jpsi_bc_loose.pdf} &
\includegraphics[width=0.225\textwidth]{imagens/CaloRinger_Analysis_JpsiVsMinBias/CorrelationPlots_BC/Jpsi_3e3_medium/CaloRinger_Analysis_JpsiVsMinBias_hadleakagexnnoutput_jpsi_bc_medium.pdf} &
\includegraphics[width=0.225\textwidth]{imagens/CaloRinger_Analysis_JpsiVsMinBias/CorrelationPlots_BC/Jpsi_3e3_tight/CaloRinger_Analysis_JpsiVsMinBias_hadleakagexnnoutput_jpsi_bc_tight.pdf} \\
\textbf{Minbias} \linebreak (Barril)&  
\includegraphics[width=0.225\textwidth]{imagens/CaloRinger_Analysis_JpsiVsMinBias/CorrelationPlots_BC/Minbias_noreq/CaloRinger_Analysis_JpsiVsMinBias_hadleakagexnnoutput_minb_bc_all.pdf} &
\includegraphics[width=0.225\textwidth]{imagens/CaloRinger_Analysis_JpsiVsMinBias/CorrelationPlots_BC/Minbias_loose/CaloRinger_Analysis_JpsiVsMinBias_hadleakagexnnoutput_minb_bc_loose.pdf} &
\includegraphics[width=0.225\textwidth]{imagens/CaloRinger_Analysis_JpsiVsMinBias/CorrelationPlots_BC/Minbias_medium/CaloRinger_Analysis_JpsiVsMinBias_hadleakagexnnoutput_minb_bc_medium.pdf} &
\includegraphics[width=0.225\textwidth]{imagens/CaloRinger_Analysis_JpsiVsMinBias/CorrelationPlots_BC/Minbias_tight/CaloRinger_Analysis_JpsiVsMinBias_hadleakagexnnoutput_minb_bc_tight.pdf} \\
\textbf{Single\linebreak part\_e}\linebreak (Tampa) &  
\includegraphics[width=0.225\textwidth]{imagens/CaloRinger_Analysis_JpsiVsMinBias/CorrelationPlots_AC/Jpsi_3e3_noreq/CaloRinger_Analysis_JpsiVsMinBias_hadleakagexnnoutput_jpsi_ac_all.pdf} &
\includegraphics[width=0.225\textwidth]{imagens/CaloRinger_Analysis_JpsiVsMinBias/CorrelationPlots_AC/Jpsi_3e3_loose/CaloRinger_Analysis_JpsiVsMinBias_hadleakagexnnoutput_jpsi_ac_loose.pdf} &
\includegraphics[width=0.225\textwidth]{imagens/CaloRinger_Analysis_JpsiVsMinBias/CorrelationPlots_AC/Jpsi_3e3_medium/CaloRinger_Analysis_JpsiVsMinBias_hadleakagexnnoutput_jpsi_ac_medium.pdf} &
\includegraphics[width=0.225\textwidth]{imagens/CaloRinger_Analysis_JpsiVsMinBias/CorrelationPlots_AC/Jpsi_3e3_tight/CaloRinger_Analysis_JpsiVsMinBias_hadleakagexnnoutput_jpsi_ac_tight.pdf}
\\
\textbf{Minbias} \linebreak (Tampa)&  
\includegraphics[width=0.225\textwidth]{imagens/CaloRinger_Analysis_JpsiVsMinBias/CorrelationPlots_AC/Minbias_noreq/CaloRinger_Analysis_JpsiVsMinBias_hadleakagexnnoutput_minb_ac_all.pdf} &
\includegraphics[width=0.225\textwidth]{imagens/CaloRinger_Analysis_JpsiVsMinBias/CorrelationPlots_AC/Minbias_loose/CaloRinger_Analysis_JpsiVsMinBias_hadleakagexnnoutput_minb_ac_loose.pdf} &
\includegraphics[width=0.225\textwidth]{imagens/CaloRinger_Analysis_JpsiVsMinBias/CorrelationPlots_AC/Minbias_medium/CaloRinger_Analysis_JpsiVsMinBias_hadleakagexnnoutput_minb_ac_medium.pdf} &
\includegraphics[width=0.225\textwidth]{imagens/CaloRinger_Analysis_JpsiVsMinBias/CorrelationPlots_AC/Minbias_tight/CaloRinger_Analysis_JpsiVsMinBias_hadleakagexnnoutput_minb_ac_tight.pdf}
\\
\end{tabular}
}
\label{fig:jpsixminb_hadleakage}
\caption{Correlações da saída neural para o conjunto JPsi x Minbias com:
Rhad1.}
\end{sidewaysfigure}

%--------------------------------------------------
\begin{sidewaysfigure}[phb]
\centering
\resizebox{\textwidth}{!}{
\begin{tabular}{
>{\centering\arraybackslash}m{0.10\textwidth} 
>{\centering\arraybackslash}m{0.225\textwidth}@{\hskip 0.01cm}
>{\centering\arraybackslash}m{0.225\textwidth}@{\hskip 0.01cm}
>{\centering\arraybackslash}m{0.225\textwidth}@{\hskip 0.01cm}
>{\centering\arraybackslash}m{0.225\textwidth}@{\hskip 0.01cm}
}
 & \textbf{Sem Req.} & \textbf{Loose} & \textbf{Medium} & \textbf{Tight} \\
\textbf{Single\linebreak part\_e}\linebreak (Barril) &  
\includegraphics[width=0.225\textwidth]{imagens/CaloRinger_Analysis_JpsiVsMinBias/CorrelationPlots_BC/Jpsi_3e3_noreq/CaloRinger_Analysis_JpsiVsMinBias_width2xnnoutput_jpsi_bc_all.pdf} &
\includegraphics[width=0.225\textwidth]{imagens/CaloRinger_Analysis_JpsiVsMinBias/CorrelationPlots_BC/Jpsi_3e3_loose/CaloRinger_Analysis_JpsiVsMinBias_width2xnnoutput_jpsi_bc_loose.pdf} &
\includegraphics[width=0.225\textwidth]{imagens/CaloRinger_Analysis_JpsiVsMinBias/CorrelationPlots_BC/Jpsi_3e3_medium/CaloRinger_Analysis_JpsiVsMinBias_width2xnnoutput_jpsi_bc_medium.pdf} &
\includegraphics[width=0.225\textwidth]{imagens/CaloRinger_Analysis_JpsiVsMinBias/CorrelationPlots_BC/Jpsi_3e3_tight/CaloRinger_Analysis_JpsiVsMinBias_width2xnnoutput_jpsi_bc_tight.pdf} \\
\textbf{Minbias} \linebreak (Barril)&  
\includegraphics[width=0.225\textwidth]{imagens/CaloRinger_Analysis_JpsiVsMinBias/CorrelationPlots_BC/Minbias_noreq/CaloRinger_Analysis_JpsiVsMinBias_width2xnnoutput_minb_bc_all.pdf} &
\includegraphics[width=0.225\textwidth]{imagens/CaloRinger_Analysis_JpsiVsMinBias/CorrelationPlots_BC/Minbias_loose/CaloRinger_Analysis_JpsiVsMinBias_width2xnnoutput_minb_bc_loose.pdf} &
\includegraphics[width=0.225\textwidth]{imagens/CaloRinger_Analysis_JpsiVsMinBias/CorrelationPlots_BC/Minbias_medium/CaloRinger_Analysis_JpsiVsMinBias_width2xnnoutput_minb_bc_medium.pdf} &
\includegraphics[width=0.225\textwidth]{imagens/CaloRinger_Analysis_JpsiVsMinBias/CorrelationPlots_BC/Minbias_tight/CaloRinger_Analysis_JpsiVsMinBias_width2xnnoutput_minb_bc_tight.pdf} \\
\textbf{Single\linebreak part\_e}\linebreak (Tampa) &  
\includegraphics[width=0.225\textwidth]{imagens/CaloRinger_Analysis_JpsiVsMinBias/CorrelationPlots_AC/Jpsi_3e3_noreq/CaloRinger_Analysis_JpsiVsMinBias_width2xnnoutput_jpsi_ac_all.pdf} &
\includegraphics[width=0.225\textwidth]{imagens/CaloRinger_Analysis_JpsiVsMinBias/CorrelationPlots_AC/Jpsi_3e3_loose/CaloRinger_Analysis_JpsiVsMinBias_width2xnnoutput_jpsi_ac_loose.pdf} &
\includegraphics[width=0.225\textwidth]{imagens/CaloRinger_Analysis_JpsiVsMinBias/CorrelationPlots_AC/Jpsi_3e3_medium/CaloRinger_Analysis_JpsiVsMinBias_width2xnnoutput_jpsi_ac_medium.pdf} &
\includegraphics[width=0.225\textwidth]{imagens/CaloRinger_Analysis_JpsiVsMinBias/CorrelationPlots_AC/Jpsi_3e3_tight/CaloRinger_Analysis_JpsiVsMinBias_width2xnnoutput_jpsi_ac_tight.pdf}
\\
\textbf{Minbias} \linebreak (Tampa)&  
\includegraphics[width=0.225\textwidth]{imagens/CaloRinger_Analysis_JpsiVsMinBias/CorrelationPlots_AC/Minbias_noreq/CaloRinger_Analysis_JpsiVsMinBias_width2xnnoutput_minb_ac_all.pdf} &
\includegraphics[width=0.225\textwidth]{imagens/CaloRinger_Analysis_JpsiVsMinBias/CorrelationPlots_AC/Minbias_loose/CaloRinger_Analysis_JpsiVsMinBias_width2xnnoutput_minb_ac_loose.pdf} &
\includegraphics[width=0.225\textwidth]{imagens/CaloRinger_Analysis_JpsiVsMinBias/CorrelationPlots_AC/Minbias_medium/CaloRinger_Analysis_JpsiVsMinBias_width2xnnoutput_minb_ac_medium.pdf} &
\includegraphics[width=0.225\textwidth]{imagens/CaloRinger_Analysis_JpsiVsMinBias/CorrelationPlots_AC/Minbias_tight/CaloRinger_Analysis_JpsiVsMinBias_width2xnnoutput_minb_ac_tight.pdf}
\\
\end{tabular}
}
\label{fig:jpsixminb_width2}
\caption{Correlações da saída neural para o conjunto JPsi x Minbias com:
wEta2.}
\end{sidewaysfigure}

\FloatBarrier

\subsection{\texorpdfstring{Z$\rightarrow$ee $\times$ JetTauEtmiss}{Zee x
JetTauEtMiss}}
\label{ssec:Zee}

\begin{figure}[ht]
\centering
\includegraphics[width=.7\textwidth]{imagens/CaloRinger_Analysis_ZeexJetTauEtMiss/EnergyDistribution/CaloRinger_Analysis_ZeexJetTauEtMissdata_energy_dist.pdf}
\label{fig:zeexjet_distenergia}
\caption{Distribuição de energia para o conjunto Zee x JetTauEtMiss.}
\end{figure}

\begin{table}[htb]
\centering
\begin{tabular}{cccc}
\hline
\hline
 & 
\multicolumn{2}{c}{DET (\%) para algoritmo} & 
\\
\cline{2-3}
\multirow{-2}{*}{Req. Do Alg. Padrão} & 
Alg. Padrão & 
EgCaloRinger & 
\multirow{-2}{*}{para FA (\%)} \\
\hline
Loose &  48.44 & 54.35 & 16.52 \\
Medium & 11.70 & 36.47 & 5.61 \\
Tight & 7.47 & 20.99 &  1.83 \\
\hline
\hline
\end{tabular}
\caption{Eficiências do algoritmos para FA fixos para o conjunto Zee x JetTauEtMiss.}
\label{tab:zeexjet_efic}
\end{table}

\begin{table}[htb]
\centering
\begin{tabular}{l cccc}
\hline
\hline
DET (\%)& Todo o Conj. & Loose & Medium & Tight \\
\hline
Todo o Conj.   &    - &  48.44 &  11.70 &  7.47  \\ 
\hline
\gls{rna} Loose & 54.63 & 43.57/48.44 & 11.23/11.70 & 7.35/7.47 \\
\hline
\gls{rna} Medium & 37.14 & 34.45/48.44 & 10.05/11.70 & 6.82/7.47 \\
\hline
\gls{rna} Tight &  21.25 & 20.74/48.44 & 7.85/11.70  & 5.57/7.47 \\
\hline
\hline
\end{tabular}
\caption{Taxa de detecção (\%) dos algoritmos e parcela na qual o EgCaloRinger
identifica em comum ao algoritmo padrão. Conjunto Zee x JetTauEtMiss.}
\label{tab:zeexjet_efic_det}
\end{table}

\begin{table}[htb]
\centering
\begin{tabular}{l cccc}
\hline
\hline
FA (\%)& Todo o Conj. & Loose & Medium & Tight \\
\hline
Todo o Conj.  &  - & 16.52 & 5.61 & 1.83 \\
\hline
\gls{rna} Loose  & 16.81 & 8.94/16.52 &  3.73/5.61 & 1.51/1.83 \\
\hline                                             
\gls{rna} Medium & 5.83  & 4.31/16.52 &  2.16/5.61 & 1.01/1.83 \\
\hline
\gls{rna} Tight  & 1.86  & 1.65/16.52 &  0.93/5.61 & 0.47/1.83 \\
\hline
\hline
\end{tabular}
\caption{Taxa de falso alarme (\%) dos algoritmos e parcela na qual o EgCaloRinger
identifica em comum ao algoritmo padrão. Conjunto Zee x JetTauEtMiss.}
\label{tab:zeexjet_fa_det}
\end{table}


\begin{figure}[ht]
\centering
\resizebox{\textwidth}{!}{
\begin{tabular}{
>{\centering\arraybackslash}m{0.50\textwidth}@{\hskip 0.5cm}
>{\centering\arraybackslash}m{0.50\textwidth}@{\hskip 0.5cm}
}
\includegraphics[width=0.5\textwidth]{imagens/CaloRinger_Analysis_ZeexJetTauEtMiss/NeuralOutput/CaloRinger_Analysis_ZeexJetTauEtMiss_nnoutput_electrons.pdf}
&
\includegraphics[width=0.5\textwidth]{imagens/CaloRinger_Analysis_ZeexJetTauEtMiss/NeuralOutput/CaloRinger_Analysis_ZeexJetTauEtMiss_nnoutput_jets.pdf}
\\
(a) \textbf{Elétrons} & (b) \textbf{Jatos} \\
\end{tabular}
}
\label{fig:zeexjet_saidaneural}
\caption{Saída da rede neural e suas relações com os requerimentos do algoritmo
eGamma padrão. Conjunto Zee x JetTauEtMiss. }
\end{figure}

\begin{figure}[ht]
\centering
\includegraphics[width=.7\textwidth]{imagens/CaloRinger_Analysis_ZeexJetTauEtMiss/Efficiency/CaloRinger_Analysis_ZeexJetTauEtMiss_roc.pdf}
\label{fig:zeexjet_roc}
\caption{Curva ROC para o conjunto Zee x JetTauEtMiss.}
\end{figure}



%--------------------------------------------------
\begin{sidewaysfigure}[phb]
\centering
\resizebox{\textwidth}{!}{
\begin{tabular}{
>{\centering\arraybackslash}m{\textwidth}@{\hskip 0.01cm}
}
\includegraphics[width=.7\textwidth,height=0.32\textheight]{imagens/CaloRinger_Analysis_ZeexJetTauEtMiss/Efficiency/CaloRinger_Analysis_ZeexJetTauEtMiss_comparison_loose_eff.pdf} \\
\includegraphics[width=.7\textwidth,height=0.32\textheight]{imagens/CaloRinger_Analysis_ZeexJetTauEtMiss/Efficiency/CaloRinger_Analysis_ZeexJetTauEtMiss_comparison_loose_fa.pdf} \\
\end{tabular}
}
\label{fig:zeexjet_eficiencia_loose}
\caption{Comparações das eficiências em função das variáveis $\eta$ e $E_{T}$
para ambos algoritmos no requerimento \emph{Loose}, conjunto Zee x JetTauEtMiss.}
\end{sidewaysfigure}

%--------------------------------------------------
\begin{sidewaysfigure}[phb]
\centering
\resizebox{\textwidth}{!}{
\begin{tabular}{
>{\centering\arraybackslash}m{\textwidth}@{\hskip 0.01cm}
}
\includegraphics[width=.7\textwidth,height=0.32\textheight]{imagens/CaloRinger_Analysis_ZeexJetTauEtMiss/Efficiency/CaloRinger_Analysis_ZeexJetTauEtMiss_comparison_medium_eff.pdf} \\
\includegraphics[width=.7\textwidth,height=0.32\textheight]{imagens/CaloRinger_Analysis_ZeexJetTauEtMiss/Efficiency/CaloRinger_Analysis_ZeexJetTauEtMiss_comparison_medium_fa.pdf} \\
\end{tabular}
}
\label{fig:zeexjet_eficiencia_medium}
\caption{Comparações das eficiências em função das variáveis $\eta$ e $E_{T}$
para ambos algoritmos no requerimento \emph{Medium}, conjunto Zee x JetTauEtMiss.}
\end{sidewaysfigure}

%--------------------------------------------------
\begin{sidewaysfigure}[phb]
\centering
\resizebox{\textwidth}{!}{
\begin{tabular}{
>{\centering\arraybackslash}m{\textwidth}@{\hskip 0.01cm}
}
\includegraphics[width=.7\textwidth,height=0.32\textheight]{imagens/CaloRinger_Analysis_ZeexJetTauEtMiss/Efficiency/CaloRinger_Analysis_ZeexJetTauEtMiss_comparison_tight_eff.pdf} \\
\includegraphics[width=.7\textwidth,height=0.32\textheight]{imagens/CaloRinger_Analysis_ZeexJetTauEtMiss/Efficiency/CaloRinger_Analysis_ZeexJetTauEtMiss_comparison_tight_fa.pdf} \\
\end{tabular}
}
\label{fig:zeexjet_eficiencia_tight}
\caption{Comparações das eficiências em função das variáveis $\eta$ e $E_{T}$
para ambos algoritmos no requerimento \emph{Tight}, conjunto Zee x JetTauEtMiss.}
\end{sidewaysfigure}

%--------------------------------------------------
\begin{sidewaysfigure}[phb]
\centering
\resizebox{\textwidth}{!}{
\begin{tabular}{
>{\centering\arraybackslash}m{0.10\textwidth} 
>{\centering\arraybackslash}m{0.225\textwidth}@{\hskip 0.01cm}
>{\centering\arraybackslash}m{0.225\textwidth}@{\hskip 0.01cm}
>{\centering\arraybackslash}m{0.225\textwidth}@{\hskip 0.01cm}
>{\centering\arraybackslash}m{0.225\textwidth}@{\hskip 0.01cm}
}
 & \textbf{Sem Req.} & \textbf{Loose} & \textbf{Medium} & \textbf{Tight} \\
\textbf{Single\linebreak part\_e}\linebreak (Barril) &  
\includegraphics[width=0.225\textwidth]{imagens/CaloRinger_Analysis_ZeexJetTauEtMiss/CorrelationPlots_BC/Zee_noreq/CaloRinger_Analysis_ZeexJetTauEtMiss_rcorexnnoutput_electron_bc_all.pdf} &
\includegraphics[width=0.225\textwidth]{imagens/CaloRinger_Analysis_ZeexJetTauEtMiss/CorrelationPlots_BC/Zee_loose/CaloRinger_Analysis_ZeexJetTauEtMiss_rcorexnnoutput_electron_bc_loose.pdf} &
\includegraphics[width=0.225\textwidth]{imagens/CaloRinger_Analysis_ZeexJetTauEtMiss/CorrelationPlots_BC/Zee_medium/CaloRinger_Analysis_ZeexJetTauEtMiss_rcorexnnoutput_electron_bc_medium.pdf} &
\includegraphics[width=0.225\textwidth]{imagens/CaloRinger_Analysis_ZeexJetTauEtMiss/CorrelationPlots_BC/Zee_tight/CaloRinger_Analysis_ZeexJetTauEtMiss_rcorexnnoutput_electron_bc_tight.pdf} \\
\textbf{JetTauMiss} \linebreak (Barril)&  
\includegraphics[width=0.225\textwidth]{imagens/CaloRinger_Analysis_ZeexJetTauEtMiss/CorrelationPlots_BC/JetTauMiss_noreq/CaloRinger_Analysis_ZeexJetTauEtMiss_rcorexnnoutput_jet_bc_all.pdf} &
\includegraphics[width=0.225\textwidth]{imagens/CaloRinger_Analysis_ZeexJetTauEtMiss/CorrelationPlots_BC/JetTauMiss_loose/CaloRinger_Analysis_ZeexJetTauEtMiss_rcorexnnoutput_jet_bc_loose.pdf} &
\includegraphics[width=0.225\textwidth]{imagens/CaloRinger_Analysis_ZeexJetTauEtMiss/CorrelationPlots_BC/JetTauMiss_medium/CaloRinger_Analysis_ZeexJetTauEtMiss_rcorexnnoutput_jet_bc_medium.pdf} &
\includegraphics[width=0.225\textwidth]{imagens/CaloRinger_Analysis_ZeexJetTauEtMiss/CorrelationPlots_BC/JetTauMiss_tight/CaloRinger_Analysis_ZeexJetTauEtMiss_rcorexnnoutput_jet_bc_tight.pdf} \\
\textbf{Single\linebreak part\_e}\linebreak (Tampa) &  
\includegraphics[width=0.225\textwidth]{imagens/CaloRinger_Analysis_ZeexJetTauEtMiss/CorrelationPlots_AC/Zee_noreq/CaloRinger_Analysis_ZeexJetTauEtMiss_rcorexnnoutput_electron_ac_all.pdf} &
\includegraphics[width=0.225\textwidth]{imagens/CaloRinger_Analysis_ZeexJetTauEtMiss/CorrelationPlots_AC/Zee_loose/CaloRinger_Analysis_ZeexJetTauEtMiss_rcorexnnoutput_electron_ac_loose.pdf} &
\includegraphics[width=0.225\textwidth]{imagens/CaloRinger_Analysis_ZeexJetTauEtMiss/CorrelationPlots_AC/Zee_medium/CaloRinger_Analysis_ZeexJetTauEtMiss_rcorexnnoutput_electron_ac_medium.pdf} &
\includegraphics[width=0.225\textwidth]{imagens/CaloRinger_Analysis_ZeexJetTauEtMiss/CorrelationPlots_AC/Zee_tight/CaloRinger_Analysis_ZeexJetTauEtMiss_rcorexnnoutput_electron_ac_tight.pdf}
\\
\textbf{JetTauMiss} \linebreak (Tampa)&  
\includegraphics[width=0.225\textwidth]{imagens/CaloRinger_Analysis_ZeexJetTauEtMiss/CorrelationPlots_AC/JetTauMiss_noreq/CaloRinger_Analysis_ZeexJetTauEtMiss_rcorexnnoutput_jet_ac_all.pdf} &
\includegraphics[width=0.225\textwidth]{imagens/CaloRinger_Analysis_ZeexJetTauEtMiss/CorrelationPlots_AC/JetTauMiss_loose/CaloRinger_Analysis_ZeexJetTauEtMiss_rcorexnnoutput_jet_ac_loose.pdf} &
\includegraphics[width=0.225\textwidth]{imagens/CaloRinger_Analysis_ZeexJetTauEtMiss/CorrelationPlots_AC/JetTauMiss_medium/CaloRinger_Analysis_ZeexJetTauEtMiss_rcorexnnoutput_jet_ac_medium.pdf} &
\includegraphics[width=0.225\textwidth]{imagens/CaloRinger_Analysis_ZeexJetTauEtMiss/CorrelationPlots_AC/JetTauMiss_tight/CaloRinger_Analysis_ZeexJetTauEtMiss_rcorexnnoutput_jet_ac_tight.pdf}
\\
\end{tabular}
}
\label{fig:zeexjet_rcore}
\caption{Correlações da saída neural para o conjunto Zee x JetTauEtMiss com:
rEta.}
\end{sidewaysfigure}

%--------------------------------------------------
\begin{sidewaysfigure}[phb]
\centering
\resizebox{\textwidth}{!}{
\begin{tabular}{
>{\centering\arraybackslash}m{0.10\textwidth} 
>{\centering\arraybackslash}m{0.225\textwidth}@{\hskip 0.01cm}
>{\centering\arraybackslash}m{0.225\textwidth}@{\hskip 0.01cm}
>{\centering\arraybackslash}m{0.225\textwidth}@{\hskip 0.01cm}
>{\centering\arraybackslash}m{0.225\textwidth}@{\hskip 0.01cm}
}
 & \textbf{Sem Req.} & \textbf{Loose} & \textbf{Medium} & \textbf{Tight} \\
\textbf{Single\linebreak part\_e}\linebreak (Barril) &  
\includegraphics[width=0.225\textwidth]{imagens/CaloRinger_Analysis_ZeexJetTauEtMiss/CorrelationPlots_BC/Zee_noreq/CaloRinger_Analysis_ZeexJetTauEtMiss_eratioxnnoutput_electron_bc_all.pdf} &
\includegraphics[width=0.225\textwidth]{imagens/CaloRinger_Analysis_ZeexJetTauEtMiss/CorrelationPlots_BC/Zee_loose/CaloRinger_Analysis_ZeexJetTauEtMiss_eratioxnnoutput_electron_bc_loose.pdf} &
\includegraphics[width=0.225\textwidth]{imagens/CaloRinger_Analysis_ZeexJetTauEtMiss/CorrelationPlots_BC/Zee_medium/CaloRinger_Analysis_ZeexJetTauEtMiss_eratioxnnoutput_electron_bc_medium.pdf} &
\includegraphics[width=0.225\textwidth]{imagens/CaloRinger_Analysis_ZeexJetTauEtMiss/CorrelationPlots_BC/Zee_tight/CaloRinger_Analysis_ZeexJetTauEtMiss_eratioxnnoutput_electron_bc_tight.pdf} \\
\textbf{JetTauMiss} \linebreak (Barril)&  
\includegraphics[width=0.225\textwidth]{imagens/CaloRinger_Analysis_ZeexJetTauEtMiss/CorrelationPlots_BC/JetTauMiss_noreq/CaloRinger_Analysis_ZeexJetTauEtMiss_eratioxnnoutput_jet_bc_all.pdf} &
\includegraphics[width=0.225\textwidth]{imagens/CaloRinger_Analysis_ZeexJetTauEtMiss/CorrelationPlots_BC/JetTauMiss_loose/CaloRinger_Analysis_ZeexJetTauEtMiss_eratioxnnoutput_jet_bc_loose.pdf} &
\includegraphics[width=0.225\textwidth]{imagens/CaloRinger_Analysis_ZeexJetTauEtMiss/CorrelationPlots_BC/JetTauMiss_medium/CaloRinger_Analysis_ZeexJetTauEtMiss_eratioxnnoutput_jet_bc_medium.pdf} &
\includegraphics[width=0.225\textwidth]{imagens/CaloRinger_Analysis_ZeexJetTauEtMiss/CorrelationPlots_BC/JetTauMiss_tight/CaloRinger_Analysis_ZeexJetTauEtMiss_eratioxnnoutput_jet_bc_tight.pdf} \\
\textbf{Single\linebreak part\_e}\linebreak (Tampa) &  
\includegraphics[width=0.225\textwidth]{imagens/CaloRinger_Analysis_ZeexJetTauEtMiss/CorrelationPlots_AC/Zee_noreq/CaloRinger_Analysis_ZeexJetTauEtMiss_eratioxnnoutput_electron_ac_all.pdf} &
\includegraphics[width=0.225\textwidth]{imagens/CaloRinger_Analysis_ZeexJetTauEtMiss/CorrelationPlots_AC/Zee_loose/CaloRinger_Analysis_ZeexJetTauEtMiss_eratioxnnoutput_electron_ac_loose.pdf} &
\includegraphics[width=0.225\textwidth]{imagens/CaloRinger_Analysis_ZeexJetTauEtMiss/CorrelationPlots_AC/Zee_medium/CaloRinger_Analysis_ZeexJetTauEtMiss_eratioxnnoutput_electron_ac_medium.pdf} &
\includegraphics[width=0.225\textwidth]{imagens/CaloRinger_Analysis_ZeexJetTauEtMiss/CorrelationPlots_AC/Zee_tight/CaloRinger_Analysis_ZeexJetTauEtMiss_eratioxnnoutput_electron_ac_tight.pdf}
\\
\textbf{JetTauMiss} \linebreak (Tampa)&  
\includegraphics[width=0.225\textwidth]{imagens/CaloRinger_Analysis_ZeexJetTauEtMiss/CorrelationPlots_AC/JetTauMiss_noreq/CaloRinger_Analysis_ZeexJetTauEtMiss_eratioxnnoutput_jet_ac_all.pdf} &
\includegraphics[width=0.225\textwidth]{imagens/CaloRinger_Analysis_ZeexJetTauEtMiss/CorrelationPlots_AC/JetTauMiss_loose/CaloRinger_Analysis_ZeexJetTauEtMiss_eratioxnnoutput_jet_ac_loose.pdf} &
\includegraphics[width=0.225\textwidth]{imagens/CaloRinger_Analysis_ZeexJetTauEtMiss/CorrelationPlots_AC/JetTauMiss_medium/CaloRinger_Analysis_ZeexJetTauEtMiss_eratioxnnoutput_jet_ac_medium.pdf} &
\includegraphics[width=0.225\textwidth]{imagens/CaloRinger_Analysis_ZeexJetTauEtMiss/CorrelationPlots_AC/JetTauMiss_tight/CaloRinger_Analysis_ZeexJetTauEtMiss_eratioxnnoutput_jet_ac_tight.pdf}
\\
\end{tabular}
}
\label{fig:zeexjet_eratio}
\caption{Correlações da saída neural para o conjunto Zee x JetTauEtMiss com:
eRatio.}
\end{sidewaysfigure}

%--------------------------------------------------
\begin{sidewaysfigure}[hpb]
\centering
\resizebox{\textwidth}{!}{
\begin{tabular}{
>{\centering\arraybackslash}m{0.10\textwidth} 
>{\centering\arraybackslash}m{0.225\textwidth}@{\hskip 0.01cm}
>{\centering\arraybackslash}m{0.225\textwidth}@{\hskip 0.01cm}
>{\centering\arraybackslash}m{0.225\textwidth}@{\hskip 0.01cm}
>{\centering\arraybackslash}m{0.225\textwidth}@{\hskip 0.01cm}
}
 & \textbf{Sem Req.} & \textbf{Loose} & \textbf{Medium} & \textbf{Tight} \\
\textbf{Single\linebreak part\_e}\linebreak (Barril) &  
\includegraphics[width=0.225\textwidth]{imagens/CaloRinger_Analysis_ZeexJetTauEtMiss/CorrelationPlots_BC/Zee_noreq/CaloRinger_Analysis_ZeexJetTauEtMiss_hadleakagexnnoutput_electron_bc_all.pdf} &
\includegraphics[width=0.225\textwidth]{imagens/CaloRinger_Analysis_ZeexJetTauEtMiss/CorrelationPlots_BC/Zee_loose/CaloRinger_Analysis_ZeexJetTauEtMiss_hadleakagexnnoutput_electron_bc_loose.pdf} &
\includegraphics[width=0.225\textwidth]{imagens/CaloRinger_Analysis_ZeexJetTauEtMiss/CorrelationPlots_BC/Zee_medium/CaloRinger_Analysis_ZeexJetTauEtMiss_hadleakagexnnoutput_electron_bc_medium.pdf} &
\includegraphics[width=0.225\textwidth]{imagens/CaloRinger_Analysis_ZeexJetTauEtMiss/CorrelationPlots_BC/Zee_tight/CaloRinger_Analysis_ZeexJetTauEtMiss_hadleakagexnnoutput_electron_bc_tight.pdf} \\
\textbf{JetTauMiss} \linebreak (Barril)&  
\includegraphics[width=0.225\textwidth]{imagens/CaloRinger_Analysis_ZeexJetTauEtMiss/CorrelationPlots_BC/JetTauMiss_noreq/CaloRinger_Analysis_ZeexJetTauEtMiss_hadleakagexnnoutput_jet_bc_all.pdf} &
\includegraphics[width=0.225\textwidth]{imagens/CaloRinger_Analysis_ZeexJetTauEtMiss/CorrelationPlots_BC/JetTauMiss_loose/CaloRinger_Analysis_ZeexJetTauEtMiss_hadleakagexnnoutput_jet_bc_loose.pdf} &
\includegraphics[width=0.225\textwidth]{imagens/CaloRinger_Analysis_ZeexJetTauEtMiss/CorrelationPlots_BC/JetTauMiss_medium/CaloRinger_Analysis_ZeexJetTauEtMiss_hadleakagexnnoutput_jet_bc_medium.pdf} &
\includegraphics[width=0.225\textwidth]{imagens/CaloRinger_Analysis_ZeexJetTauEtMiss/CorrelationPlots_BC/JetTauMiss_tight/CaloRinger_Analysis_ZeexJetTauEtMiss_hadleakagexnnoutput_jet_bc_tight.pdf} \\
\textbf{Single\linebreak part\_e}\linebreak (Tampa) &  
\includegraphics[width=0.225\textwidth]{imagens/CaloRinger_Analysis_ZeexJetTauEtMiss/CorrelationPlots_AC/Zee_noreq/CaloRinger_Analysis_ZeexJetTauEtMiss_hadleakagexnnoutput_electron_ac_all.pdf} &
\includegraphics[width=0.225\textwidth]{imagens/CaloRinger_Analysis_ZeexJetTauEtMiss/CorrelationPlots_AC/Zee_loose/CaloRinger_Analysis_ZeexJetTauEtMiss_hadleakagexnnoutput_electron_ac_loose.pdf} &
\includegraphics[width=0.225\textwidth]{imagens/CaloRinger_Analysis_ZeexJetTauEtMiss/CorrelationPlots_AC/Zee_medium/CaloRinger_Analysis_ZeexJetTauEtMiss_hadleakagexnnoutput_electron_ac_medium.pdf} &
\includegraphics[width=0.225\textwidth]{imagens/CaloRinger_Analysis_ZeexJetTauEtMiss/CorrelationPlots_AC/Zee_tight/CaloRinger_Analysis_ZeexJetTauEtMiss_hadleakagexnnoutput_electron_ac_tight.pdf}
\\
\textbf{JetTauMiss} \linebreak (Tampa)&  
\includegraphics[width=0.225\textwidth]{imagens/CaloRinger_Analysis_ZeexJetTauEtMiss/CorrelationPlots_AC/JetTauMiss_noreq/CaloRinger_Analysis_ZeexJetTauEtMiss_hadleakagexnnoutput_jet_ac_all.pdf} &
\includegraphics[width=0.225\textwidth]{imagens/CaloRinger_Analysis_ZeexJetTauEtMiss/CorrelationPlots_AC/JetTauMiss_loose/CaloRinger_Analysis_ZeexJetTauEtMiss_hadleakagexnnoutput_jet_ac_loose.pdf} &
\includegraphics[width=0.225\textwidth]{imagens/CaloRinger_Analysis_ZeexJetTauEtMiss/CorrelationPlots_AC/JetTauMiss_medium/CaloRinger_Analysis_ZeexJetTauEtMiss_hadleakagexnnoutput_jet_ac_medium.pdf} &
\includegraphics[width=0.225\textwidth]{imagens/CaloRinger_Analysis_ZeexJetTauEtMiss/CorrelationPlots_AC/JetTauMiss_tight/CaloRinger_Analysis_ZeexJetTauEtMiss_hadleakagexnnoutput_jet_ac_tight.pdf}
\\
\end{tabular}
}
\label{fig:zeexjet_hadleakage}
\caption{Correlações da saída neural para o conjunto Zee x JetTauEtMiss com:
Rhad1.}
\end{sidewaysfigure}

%--------------------------------------------------
\begin{sidewaysfigure}[hp]
\centering
\resizebox{\textwidth}{!}{
\begin{tabular}{
>{\centering\arraybackslash}m{0.10\textwidth} 
>{\centering\arraybackslash}m{0.225\textwidth}@{\hskip 0.01cm}
>{\centering\arraybackslash}m{0.225\textwidth}@{\hskip 0.01cm}
>{\centering\arraybackslash}m{0.225\textwidth}@{\hskip 0.01cm}
>{\centering\arraybackslash}m{0.225\textwidth}@{\hskip 0.01cm}
}
 & \textbf{Sem Req.} & \textbf{Loose} & \textbf{Medium} & \textbf{Tight} \\
\textbf{Single\linebreak part\_e}\linebreak (Barril) &  
\includegraphics[width=0.225\textwidth]{imagens/CaloRinger_Analysis_ZeexJetTauEtMiss/CorrelationPlots_BC/Zee_noreq/CaloRinger_Analysis_ZeexJetTauEtMiss_width2xnnoutput_electron_bc_all.pdf} &
\includegraphics[width=0.225\textwidth]{imagens/CaloRinger_Analysis_ZeexJetTauEtMiss/CorrelationPlots_BC/Zee_loose/CaloRinger_Analysis_ZeexJetTauEtMiss_width2xnnoutput_electron_bc_loose.pdf} &
\includegraphics[width=0.225\textwidth]{imagens/CaloRinger_Analysis_ZeexJetTauEtMiss/CorrelationPlots_BC/Zee_medium/CaloRinger_Analysis_ZeexJetTauEtMiss_width2xnnoutput_electron_bc_medium.pdf} &
\includegraphics[width=0.225\textwidth]{imagens/CaloRinger_Analysis_ZeexJetTauEtMiss/CorrelationPlots_BC/Zee_tight/CaloRinger_Analysis_ZeexJetTauEtMiss_width2xnnoutput_electron_bc_tight.pdf} \\
\textbf{JetTauMiss} \linebreak (Barril)&  
\includegraphics[width=0.225\textwidth]{imagens/CaloRinger_Analysis_ZeexJetTauEtMiss/CorrelationPlots_BC/JetTauMiss_noreq/CaloRinger_Analysis_ZeexJetTauEtMiss_width2xnnoutput_jet_bc_all.pdf} &
\includegraphics[width=0.225\textwidth]{imagens/CaloRinger_Analysis_ZeexJetTauEtMiss/CorrelationPlots_BC/JetTauMiss_loose/CaloRinger_Analysis_ZeexJetTauEtMiss_width2xnnoutput_jet_bc_loose.pdf} &
\includegraphics[width=0.225\textwidth]{imagens/CaloRinger_Analysis_ZeexJetTauEtMiss/CorrelationPlots_BC/JetTauMiss_medium/CaloRinger_Analysis_ZeexJetTauEtMiss_width2xnnoutput_jet_bc_medium.pdf} &
\includegraphics[width=0.225\textwidth]{imagens/CaloRinger_Analysis_ZeexJetTauEtMiss/CorrelationPlots_BC/JetTauMiss_tight/CaloRinger_Analysis_ZeexJetTauEtMiss_width2xnnoutput_jet_bc_tight.pdf} \\
\textbf{Single\linebreak part\_e}\linebreak (Tampa) &  
\includegraphics[width=0.225\textwidth]{imagens/CaloRinger_Analysis_ZeexJetTauEtMiss/CorrelationPlots_AC/Zee_noreq/CaloRinger_Analysis_ZeexJetTauEtMiss_width2xnnoutput_electron_ac_all.pdf} &
\includegraphics[width=0.225\textwidth]{imagens/CaloRinger_Analysis_ZeexJetTauEtMiss/CorrelationPlots_AC/Zee_loose/CaloRinger_Analysis_ZeexJetTauEtMiss_width2xnnoutput_electron_ac_loose.pdf} &
\includegraphics[width=0.225\textwidth]{imagens/CaloRinger_Analysis_ZeexJetTauEtMiss/CorrelationPlots_AC/Zee_medium/CaloRinger_Analysis_ZeexJetTauEtMiss_width2xnnoutput_electron_ac_medium.pdf} &
\includegraphics[width=0.225\textwidth]{imagens/CaloRinger_Analysis_ZeexJetTauEtMiss/CorrelationPlots_AC/Zee_tight/CaloRinger_Analysis_ZeexJetTauEtMiss_width2xnnoutput_electron_ac_tight.pdf}
\\
\textbf{JetTauMiss} \linebreak (Tampa)&  
\includegraphics[width=0.225\textwidth]{imagens/CaloRinger_Analysis_ZeexJetTauEtMiss/CorrelationPlots_AC/JetTauMiss_noreq/CaloRinger_Analysis_ZeexJetTauEtMiss_width2xnnoutput_jet_ac_all.pdf} &
\includegraphics[width=0.225\textwidth]{imagens/CaloRinger_Analysis_ZeexJetTauEtMiss/CorrelationPlots_AC/JetTauMiss_loose/CaloRinger_Analysis_ZeexJetTauEtMiss_width2xnnoutput_jet_ac_loose.pdf} &
\includegraphics[width=0.225\textwidth]{imagens/CaloRinger_Analysis_ZeexJetTauEtMiss/CorrelationPlots_AC/JetTauMiss_medium/CaloRinger_Analysis_ZeexJetTauEtMiss_width2xnnoutput_jet_ac_medium.pdf} &
\includegraphics[width=0.225\textwidth]{imagens/CaloRinger_Analysis_ZeexJetTauEtMiss/CorrelationPlots_AC/JetTauMiss_tight/CaloRinger_Analysis_ZeexJetTauEtMiss_width2xnnoutput_jet_ac_tight.pdf}
\\
\end{tabular}
}
\label{fig:zeexjet_width2}
\caption{Correlações da saída neural para o conjunto Zee x JetTauEtMiss com:
wEta2.}
\end{sidewaysfigure}

\FloatBarrier

