\chapter{Resultados e Discussão}
\label{cap:resultados}

Este capítulo é dedicado à apresentação dos resultados e a discussão dos estudos
realizados. A primeira abordagem (Seção~\ref{sec:norm}) tratou a otimização do
algoritmo através da normalização mais indicada para o \gls{hltringer}, 
a versão do algoritmo proposto no \glsdesc{l2}, que teve os resultados 
posteriormente confirmados nos estudos realizados
por \cite{tese_torres}. Em seguida, serão apresentados os resultados para a
versão do algoritmo no \gls{sr} (Seção~\ref{sec:efic_egcalo}), o ambiente de
reconstrução da física a posteriori.


\section{Otimização do \emph{HLT\_Ringer}: Normalização}
\label{sec:norm}

Este estudo foi dedicado a escolha da normalização a ser utilizada nos
dados antes da propagação dos anéis para a \gls{rna}. Foi utilizada a
implementação do algoritmo proposto para o \gls{sf}, pois na época ainda não
havia sido requisitado pela Colaboração a implementação da versão para o \gls{sr}. Uma descrição das
normalizações testadas foi realizada no Tópico~\ref{sssec:preproc_norm}. Foram
utilizadas simulações de \gls{mc} de 2008, contendo elétrons isolados (não sendo
estudado, assim, o efeito de Empilhamento) para o conjunto de sinal e jatos com 
\gls{Et} máximo de 17 GeV para o conjunto de ruído. As redes foram treinadas com
os parâmetros especificados no Tópico~\ref{sssec:rna}, com a exceção da
utilização de apenas uma inicialização da \gls{rna}, de forma que os resultados
estavam haver flutuação estatística nos resultados indicados. Contudo, os resultados 
foram reproduzidos por \cite{tese_torres} para as normalizações que se ressaltaram 
neste estudo utilizando 10 inicializações, obtendo resultados semelhates de 
forma que a flutuação não afetou significativamente os resultados obtidos. O
estudo também considerou os dois critérios utilizados durante o treinamento da
\gls{rna} como figura de mérito, de modo a verificar se a utilização do \gls{sp}
como critério de parada ao invés do \gls{mse} realmente melhoraria a perfomance
do \gls{rna}.

\begin{table}[ht!]
\centering
\resizebox{\textwidth}{!}{
\begin{tabular}{
>{\global\let\currentrowstyle\relax}p{4.5cm}
>{\centering\arraybackslash\currentrowstyle}p{1.5cm}
>{\centering\arraybackslash\currentrowstyle}p{1.5cm}
>{\centering\arraybackslash\currentrowstyle}p{1.5cm}
>{\sl\centering\arraybackslash\currentrowstyle}p{1.5cm}
>{\centering\arraybackslash\currentrowstyle}p{1.5cm}
>{\centering\arraybackslash\currentrowstyle}p{1.5cm}
>{\centering\arraybackslash\currentrowstyle}p{1.5cm}}
\hline \hline
\textbf{Normalização}&\textbf{Épocas}&\textbf{\gls{mse}\linebreak
TRN}& \textbf{{Critério\linebreak VAL}}&\textbf{\gls{sp}\linebreak
TST}&\textbf{\gls{det}\linebreak TST}&\textbf{\gls{fa}\linebreak
TST}&\textbf{Limiar}\\
\hline \hline
Fixa \gls{sp}                   &503&0.0418&98.7320&98.7498&99.2925&1.7914&0.028\\\hline
\rowstyle{\bfseries}%
Fixa Seção \gls{sp}             &339&0.0415&98.7370&98.7544&99.3257&1.8152&-0.016\\\hline
Fixa Seção \gls{sp} \gls{mod}   &329&0.0445&98.7168&98.7349&99.2721&1.8009&0.036\\\hline
Fixa Camada \gls{sp}            &295&0.0424&98.7218&98.7309&99.2938&1.8304&-0.028\\\hline
\rowstyle{\bfseries}%
Norma 1 \gls{sp}                 &325&0.0427&98.6154&98.6087&99.2798&2.0601&-0.052\\\hline
Norma 2 \gls{sp}                 &101&0.0444&98.6357&98.6399&99.0827&1.8019&0.224\\\hline
Norma 2 Seção \gls{sp}           &477&0.0546&98.3972&98.4319&99.1426&2.2764&0.028\\\hline
Norma 2 Camada \gls{sp}          &584&0.0615&98.1071&98.0881&98.7069&2.5289&0.052\\\hline
Sequencial \gls{sp}             &837&0.0610&98.0918&98.1530&98.7242&2.4165&0.088\\\hline
MinMax \gls{sp}                 &63 &0.8899&92.7524&92.8777&94.1037&8.3404&0.172\\\hline
\rowstyle{\bfseries}%
Esferização \gls{sp}            &237&0.0421&98.7472&98.7690&99.2792&1.7399&0.112\\\hline
\rowstyle{\bfseries}%
Esferização \gls{sp} \gls{mod}  &158&0.0434&98.7674&98.7748&99.3723&1.8209&0.004\\\hline
Fixa \gls{mse}                  &210&0.0465&0.0396 &98.7254&99.1879&1.7361&0.156\\\hline
Fixa Camada \gls{mse}           &131&0.0427&0.0400 &98.7042&99.3927&1.9820&-0.088\\\hline
Fixa Seção \gls{mse}            &253&0.0438&0.0393 &98.7192&99.3040&1.8638&0.036\\\hline
Fixa Seção \gls{mse} \gls{mod}  &175&0.0422&0.0394 &98.7218&99.3110&1.8657&0.020\\\hline
Norma 1 \gls{mse}                &117&0.0453&0.0428 &98.6006&99.2473&2.0439&0.000\\\hline
Norma 2 \gls{mse}                &105&0.0469&0.0431 &98.6262&99.1758&1.9219&0.144\\\hline
Norma 2 Seção \gls{mse}          &242&0.0541&0.0497 &98.4041&99.0208&2.2106&0.172\\\hline
Norma 2 Camada \gls{mse}         &522&0.0602&0.0585 &98.0544&98.8237&2.7118&-0.052\\\hline
Sequencial \gls{mse}            &538&0.0703&0.0636 &97.9352&98.6929&2.8195&0.088\\\hline
MinMax \gls{mse}                &331&0.1458&0.1385 &95.1713&96.3269&5.9773&-0.032\\\hline
Esferização \gls{mse}           &279&0.0410&0.0383&98.7580&99.2390&1.7218&0.168\\\hline
Esferização \gls{mse} \gls{mod} &243&0.0411&0.0372 &98.7662&99.3704&1.8362&0.000\\
\hline \hline
\end{tabular}
}
\caption[Resultados do estudo de pré-processamento: Normalização]{Resultados do
estudo de pré-processamento.}
\label{tab:res_norm}
\end{table}

Na Tabela~\ref{tab:res_norm} estão os resultados obtidos, como os resultados
para o de treinamento, como o número de épocas, o \gls{mse} do \gls{trn} em 
relação aos valores especificados, e o valor da figura de mérito
utilizada como critério de parada no conjunto de \gls{val}. O valor de
eficiência da \gls{rna} utilizando o \gls{sp} está destacado, pois esse valor foi
escolhido como a figura de mérito da eficiência da rede, e em seguida estão os
respectivos \gls{det}, e o \gls{fa} utilizados para o seu calculo e o limiar de
decisão da \gls{rna}.

Pode-se observar que a Esferização e sua versão
modificada obtiveram as melhores eficiências. Logo em seguida estão as
normalizações fixas, onde se destaca a versão por Seção da mesma. As
normalizações por norma obtiveram resultados similares, e não muito distantes
dos melhores resultados anteriormente indicados, o que pode ser uma vantagem em
alguns casos devido a sua praticidade de aplicação, aonde não é necessário a
descoberta do desvio padrão e média dos dados, ainda mais interessante para o
caso da Norma 1, onde simplesmente se soma a energia de todos os anéis para
então dividí-los por esse valor. A normalização sequencial, apesar de todo o
conhecimento especialista aplicado para a otimização da Norma 1, obteve
resultados bem inferiores aos demais, sendo talvez necessário a alteração de
seus parâmetros para uma melhor eficiência. A utilização de normalização por
Mínimo e Máximo obteve o pior resultado. Finalmente, avaliou-se que o
treinamento com o critério de parada por \gls{sp} obtém valores maiores de
eficiência que quando treinando por \gls{mse}, como o esperado.


\section[Estudo de Eficiência do \emph{eGamma Calorimeter Ringer}]{Estudo 
de Eficiência do $e/\gamma$ \emph{Calorimeter Ringer}}
\label{sec:efic_egcalo}

Para o estudo de eficiência do \gls{egcaloringer} foram utilizados três
conjuntos de dados contendo elétrons e o respectivo ruído esperado
(o nome dos conjuntos de dados utilizados estão em \cite{portal_caloringer}):
\begin{itemize}
\item \textbf{Singlepart\_e $\times$ J2}: originados através de simulações de
\gls{mc}:
\begin{itemize}
\item Conjunto de sinal: formado por elétrons isolados;
\item Conjunto de ruído: contém jatos hadrônicos.
\end{itemize}
\item \textbf{\text{J}/$\Psi \times$ Minbias}: novamente original de simulações
de \gls{mc}:
\begin{itemize}
\item Conjunto de sinal: contém decaimentos de $\text{J}/\Psi$ em elétrons adicionados de eventos de
\gls{mbias} para simular o fenômeno de Empilhamento;
\item Conjunto de ruído: composto por eventos de \gls{mbias}.
\end{itemize}
\item \textbf{Z$\rightarrow$ee $\times$ JetTauEtmiss}: composto por dados de
colisões reais da temporada 167776, com pico de luminosidade de
$1,8\times10^{30}cm^{-2}s^{-1}$ (todas as informações da temporada estão
disponíveis em \cite{info_run}):
\begin{itemize}
\item Conjunto de sinal: Contém candidatos a Zee filtrados apenas pelo \gls{l1},
estando assim altamente contaminado por ruído físico;
\item Conjunto de ruído: Contém candidatos a jatos, taóns que também podem
formar jatos e \gls{Etmiss} (neutrinos).
\end{itemize}
\end{itemize}




%\subsection{Metodologia} %?
%\label{ssec:metodologia}

\subsection{\texorpdfstring{Singlepart\_e $\times$ J2}{Singlepart\_e x J2}}
\label{ssec:single_e}

\begin{table}[ht!]
\centering
\begin{tabular}{cccc}
\hline
\hline
 & 
\multicolumn{2}{c}{DET (\%) para algoritmo} & 
\\
\cline{2-3}
\multirow{-2}{*}{Req. Do Alg. Padrão} & 
Alg. Padrão & 
EgCaloRinger & 
\multirow{-2}{*}{para FA (\%)} \\
\hline
Loose & 89,16 & 98,45(+9,26) & 28,53 \\
Medium & 71,82 & 97,00(+25,18) & 5,90 \\
Tight & 57,68 & 94,18(+36,50) & 0,94 \\
\hline
\hline
\end{tabular}
\caption{Eficiências do algoritmos para FA fixos.}
\end{table}

\begin{table}[ht!]
\centering
\begin{tabular}{l cccc}
\hline
\hline
DET (\%)& Todo o Conj. & Loose & Medium & Tight \\
\hline
Todo o Conj. &  - & 89,16 & 71,82 & 57,68 \\
\hline
\gls{rna} Loose & 95,34 & 86,49/89,16 & 71,70/71,82 & 57,62/57,68 \\
\hline
\gls{rna} Medium & 94,58  & 86,11/89,16 & 71,59/71,82 & 57,55/57,68 \\
\hline
\gls{rna} Tight &  93,37 & 85,51/89,16 &  71,31/71,82 & 57,35/57,68 \\
\hline
\hline
\end{tabular}
\caption{Taxa de deteção (\%) dos algoritmos e parcela na qual o EgCaloRinger
identifica em comum ao algoritmo padrão.}
\end{table}

\begin{table}[ht!]
\centering
\begin{tabular}{l cccc}
\hline
\hline
FA (\%)& Todo o Conj. & Loose & Medium & Tight \\
\hline
Todo o Conj. & - & 28.57 & 5.90 &  0.94 \\
\gls{rna} Loose  & 1.88  & 1.49/28.57 & 0.32/5.90 & 0.11/0.94 \\
\gls{rna} Medium & 1.16  & 0.96/28.57 & 0.21/5.90 & 0.08/0.94 \\
\gls{rna} Tight  & 0.67  & 0.58/28.57 & 0.13/5.90 & 0.05/0.94 \\
\hline
\hline
\end{tabular}
\caption{Taxa de falso alarme (\%) dos algoritmos e parcela na qual o EgCaloRinger
identifica em comum ao algoritmo padrão.}
\end{table}




\begin{table}[ht!]
\centering
\resizebox{\textwidth}{!}{
\begin{tabular}{m{4cm} ccc ccc}
\hline
\hline
\multirow{2}{4cm}{Taxa de Rejeição (\%) para partícula estável} &
\multicolumn{3}{c}{Alg. Padrão} & \multicolumn{3}{c}{CaloRinger} \\
 & \textbf{Loose} & \textbf{Medium} & \textbf{Tight} & \textbf{Loose} &
\textbf{Medium} & \textbf{Tight} \\
\hline
$\gamma$ & 78,38 & 94,10 & 99,34 & 98,69 & 99,24 & 99,58 \\
$\pi^+/\pi^-$ & 89,85 & 96,69 & 99,66 & 99,49 & 99,74 & 99,88 \\
$K^+/K^-$ & 91,81 & 97,17 & 99,66 & 99,50 & 99,72 & 99,86 \\
$K^0_l$ & 87,55 & 96,31 & 99,60 & 99,57 & 99,82 & 99,96 \\
$K^0_s$ & 87,70 & 96,85 & 99,57 & 99,49 & 99,70 & 99,86 \\
$e^+/e^-$ & 66,03 & 77,69 & 81,27 & 95,82 & 97,24 & 98,40 \\
\hline
\hline
\end{tabular}
}
\caption{Rejeição de partículas estáveis para o conjunto de J2.}
\end{table}


\begin{figure}[ht]
\centering
\subfigure[Elétrons]{%
\label{fig:singlexj2_saidaneural_eletron}
\includegraphics[width=0.45\textwidth]{imagens/CaloRinger_Analysis_ElectronVsJet/NeuralOutput/CaloRinger_Analysis_ElectronVsJet_nnoutput_electrons.pdf} 
}
\subfigure[Jatos.]{%
\label{fig:singlexj2_saidaneural_jato}
\includegraphics[width=0.45\textwidth]{imagens/CaloRinger_Analysis_ElectronVsJet/NeuralOutput/CaloRinger_Analysis_ElectronVsJet_nnoutput_jets.pdf}
}
\label{fig:singlexj2_saidaneural}
\caption{Saída da rede neural e suas relações com os requerimentos do algoritmo
eGamma padrão para o conjunto Single\_e x J2.}
\end{figure}

\begin{figure}
\centering
\includegraphics[width=.7\textwidth]{imagens/CaloRinger_Analysis_ElectronVsJet/Efficiency/CaloRinger_Analysis_ElectronVsJet_roc.pdf}
\label{fig:singlexj2_saidaneural}
\caption{Curva ROC para o conjunto Single\_e x J2.}
\end{figure}


%--------------------------------------------------
\begin{sidewaysfigure}[p]
\centering
\resizebox{\textwidth}{!}{
\begin{tabular}{
>{\centering\arraybackslash}m{0.10\textwidth} 
>{\centering\arraybackslash}m{0.45\textwidth}@{\hskip 0.01cm}
>{\centering\arraybackslash}m{0.45\textwidth}@{\hskip 0.01cm}
}
 & \textbf{Taxa de Deteção} & \textbf{Taxa de Falso Alarme} \\
\textbf{Loose} &  
\includegraphics[width=0.45\textwidth,height=0.23\textheight]{imagens/CaloRinger_Analysis_ElectronVsJet/Efficiency/CaloRinger_Analysis_ElectronVsJet_comparison_loose_eff.pdf} &
\includegraphics[width=0.45\textwidth,height=0.23\textheight]{imagens/CaloRinger_Analysis_ElectronVsJet/Efficiency/CaloRinger_Analysis_ElectronVsJet_comparison_loose_fa.pdf} \\
\textbf{Medium} &  
\includegraphics[width=0.45\textwidth,height=0.23\textheight]{imagens/CaloRinger_Analysis_ElectronVsJet/Efficiency/CaloRinger_Analysis_ElectronVsJet_comparison_medium_eff.pdf} &
\includegraphics[width=0.45\textwidth,height=0.23\textheight]{imagens/CaloRinger_Analysis_ElectronVsJet/Efficiency/CaloRinger_Analysis_ElectronVsJet_comparison_medium_fa.pdf} \\
\textbf{Tight} &  
\includegraphics[width=0.45\textwidth,height=0.23\textheight]{imagens/CaloRinger_Analysis_ElectronVsJet/Efficiency/CaloRinger_Analysis_ElectronVsJet_comparison_tight_eff.pdf} &
\includegraphics[width=0.45\textwidth,height=0.23\textheight]{imagens/CaloRinger_Analysis_ElectronVsJet/Efficiency/CaloRinger_Analysis_ElectronVsJet_comparison_tight_fa.pdf} \\
\end{tabular}
}
\label{fig:singlexj2_eficiencias}
\caption{Comparações das eficiências em função das variáveis $\eta$ e $E_{T}$
para ambos algoritmos e seus requerimentos.}
\end{sidewaysfigure}


%--------------------------------------------------
\begin{sidewaysfigure}[p]
\centering
\resizebox{\textwidth}{!}{
\begin{tabular}{
>{\centering\arraybackslash}m{0.10\textwidth} 
>{\centering\arraybackslash}m{0.225\textwidth}@{\hskip 0.01cm}
>{\centering\arraybackslash}m{0.225\textwidth}@{\hskip 0.01cm}
>{\centering\arraybackslash}m{0.225\textwidth}@{\hskip 0.01cm}
>{\centering\arraybackslash}m{0.225\textwidth}@{\hskip 0.01cm}
}
 & \textbf{Sem Req.} & \textbf{Loose} & \textbf{Medium} & \textbf{Tight} \\
\textbf{Single\linebreak part\_e}\linebreak (Barril) &  
\includegraphics[width=0.225\textwidth]{imagens/CaloRinger_Analysis_ElectronVsJet/CorrelationPlots_BC/Singlepart_e_noreq/CaloRinger_Analysis_ElectronVsJet_rcorexnnoutput_electron_bc_all.pdf} &
\includegraphics[width=0.225\textwidth]{imagens/CaloRinger_Analysis_ElectronVsJet/CorrelationPlots_BC/Singlepart_e_loose/CaloRinger_Analysis_ElectronVsJet_rcorexnnoutput_electron_bc_loose.pdf} &
\includegraphics[width=0.225\textwidth]{imagens/CaloRinger_Analysis_ElectronVsJet/CorrelationPlots_BC/Singlepart_e_medium/CaloRinger_Analysis_ElectronVsJet_rcorexnnoutput_electron_bc_medium.pdf} &
\includegraphics[width=0.225\textwidth]{imagens/CaloRinger_Analysis_ElectronVsJet/CorrelationPlots_BC/Singlepart_e_tight/CaloRinger_Analysis_ElectronVsJet_rcorexnnoutput_electron_bc_tight.pdf} \\
\textbf{J2} \linebreak (Barril)&  
\includegraphics[width=0.225\textwidth]{imagens/CaloRinger_Analysis_ElectronVsJet/CorrelationPlots_BC/J2_noreq/CaloRinger_Analysis_ElectronVsJet_rcorexnnoutput_jet_bc_all.pdf} &
\includegraphics[width=0.225\textwidth]{imagens/CaloRinger_Analysis_ElectronVsJet/CorrelationPlots_BC/J2_loose/CaloRinger_Analysis_ElectronVsJet_rcorexnnoutput_jet_bc_loose.pdf} &
\includegraphics[width=0.225\textwidth]{imagens/CaloRinger_Analysis_ElectronVsJet/CorrelationPlots_BC/J2_medium/CaloRinger_Analysis_ElectronVsJet_rcorexnnoutput_jet_bc_medium.pdf} &
\includegraphics[width=0.225\textwidth]{imagens/CaloRinger_Analysis_ElectronVsJet/CorrelationPlots_BC/J2_tight/CaloRinger_Analysis_ElectronVsJet_rcorexnnoutput_jet_bc_tight.pdf} \\
\textbf{Single\linebreak part\_e}\linebreak (Tampa) &  
\includegraphics[width=0.225\textwidth]{imagens/CaloRinger_Analysis_ElectronVsJet/CorrelationPlots_AC/Singlepart_e_noreq/CaloRinger_Analysis_ElectronVsJet_rcorexnnoutput_electron_ac_all.pdf} &
\includegraphics[width=0.225\textwidth]{imagens/CaloRinger_Analysis_ElectronVsJet/CorrelationPlots_AC/Singlepart_e_loose/CaloRinger_Analysis_ElectronVsJet_rcorexnnoutput_electron_ac_loose.pdf} &
\includegraphics[width=0.225\textwidth]{imagens/CaloRinger_Analysis_ElectronVsJet/CorrelationPlots_AC/Singlepart_e_medium/CaloRinger_Analysis_ElectronVsJet_rcorexnnoutput_electron_ac_medium.pdf} &
\includegraphics[width=0.225\textwidth]{imagens/CaloRinger_Analysis_ElectronVsJet/CorrelationPlots_AC/Singlepart_e_tight/CaloRinger_Analysis_ElectronVsJet_rcorexnnoutput_electron_ac_tight.pdf}
\\
\textbf{J2} \linebreak (Tampa)&  
\includegraphics[width=0.225\textwidth]{imagens/CaloRinger_Analysis_ElectronVsJet/CorrelationPlots_AC/J2_noreq/CaloRinger_Analysis_ElectronVsJet_rcorexnnoutput_jet_ac_all.pdf} &
\includegraphics[width=0.225\textwidth]{imagens/CaloRinger_Analysis_ElectronVsJet/CorrelationPlots_AC/J2_loose/CaloRinger_Analysis_ElectronVsJet_rcorexnnoutput_jet_ac_loose.pdf} &
\includegraphics[width=0.225\textwidth]{imagens/CaloRinger_Analysis_ElectronVsJet/CorrelationPlots_AC/J2_medium/CaloRinger_Analysis_ElectronVsJet_rcorexnnoutput_jet_ac_medium.pdf} &
\includegraphics[width=0.225\textwidth]{imagens/CaloRinger_Analysis_ElectronVsJet/CorrelationPlots_AC/J2_tight/CaloRinger_Analysis_ElectronVsJet_rcorexnnoutput_jet_ac_tight.pdf}
\\
\end{tabular}
}
\label{fig:singlexj2_rcore}
\caption{Correlações da saída neural para o conjunto Single\_e x J2 com: rEta}
\end{sidewaysfigure}

%--------------------------------------------------
\begin{sidewaysfigure}[p]
\centering
\resizebox{\textwidth}{!}{
\begin{tabular}{
>{\centering\arraybackslash}m{0.10\textwidth} 
>{\centering\arraybackslash}m{0.225\textwidth}@{\hskip 0.01cm}
>{\centering\arraybackslash}m{0.225\textwidth}@{\hskip 0.01cm}
>{\centering\arraybackslash}m{0.225\textwidth}@{\hskip 0.01cm}
>{\centering\arraybackslash}m{0.225\textwidth}@{\hskip 0.01cm}
}
 & \textbf{Sem Req.} & \textbf{Loose} & \textbf{Medium} & \textbf{Tight} \\
\textbf{Single\linebreak part\_e}\linebreak (Barril) &  
\includegraphics[width=0.225\textwidth]{imagens/CaloRinger_Analysis_ElectronVsJet/CorrelationPlots_BC/Singlepart_e_noreq/CaloRinger_Analysis_ElectronVsJet_eratioxnnoutput_electron_bc_all.pdf} &
\includegraphics[width=0.225\textwidth]{imagens/CaloRinger_Analysis_ElectronVsJet/CorrelationPlots_BC/Singlepart_e_loose/CaloRinger_Analysis_ElectronVsJet_eratioxnnoutput_electron_bc_loose.pdf} &
\includegraphics[width=0.225\textwidth]{imagens/CaloRinger_Analysis_ElectronVsJet/CorrelationPlots_BC/Singlepart_e_medium/CaloRinger_Analysis_ElectronVsJet_eratioxnnoutput_electron_bc_medium.pdf} &
\includegraphics[width=0.225\textwidth]{imagens/CaloRinger_Analysis_ElectronVsJet/CorrelationPlots_BC/Singlepart_e_tight/CaloRinger_Analysis_ElectronVsJet_eratioxnnoutput_electron_bc_tight.pdf} \\
\textbf{J2} \linebreak (Barril)&  
\includegraphics[width=0.225\textwidth]{imagens/CaloRinger_Analysis_ElectronVsJet/CorrelationPlots_BC/J2_noreq/CaloRinger_Analysis_ElectronVsJet_eratioxnnoutput_jet_bc_all.pdf} &
\includegraphics[width=0.225\textwidth]{imagens/CaloRinger_Analysis_ElectronVsJet/CorrelationPlots_BC/J2_loose/CaloRinger_Analysis_ElectronVsJet_eratioxnnoutput_jet_bc_loose.pdf} &
\includegraphics[width=0.225\textwidth]{imagens/CaloRinger_Analysis_ElectronVsJet/CorrelationPlots_BC/J2_medium/CaloRinger_Analysis_ElectronVsJet_eratioxnnoutput_jet_bc_medium.pdf} &
\includegraphics[width=0.225\textwidth]{imagens/CaloRinger_Analysis_ElectronVsJet/CorrelationPlots_BC/J2_tight/CaloRinger_Analysis_ElectronVsJet_eratioxnnoutput_jet_bc_tight.pdf} \\
\textbf{Single\linebreak part\_e}\linebreak (Tampa) &  
\includegraphics[width=0.225\textwidth]{imagens/CaloRinger_Analysis_ElectronVsJet/CorrelationPlots_AC/Singlepart_e_noreq/CaloRinger_Analysis_ElectronVsJet_eratioxnnoutput_electron_ac_all.pdf} &
\includegraphics[width=0.225\textwidth]{imagens/CaloRinger_Analysis_ElectronVsJet/CorrelationPlots_AC/Singlepart_e_loose/CaloRinger_Analysis_ElectronVsJet_eratioxnnoutput_electron_ac_loose.pdf} &
\includegraphics[width=0.225\textwidth]{imagens/CaloRinger_Analysis_ElectronVsJet/CorrelationPlots_AC/Singlepart_e_medium/CaloRinger_Analysis_ElectronVsJet_eratioxnnoutput_electron_ac_medium.pdf} &
\includegraphics[width=0.225\textwidth]{imagens/CaloRinger_Analysis_ElectronVsJet/CorrelationPlots_AC/Singlepart_e_tight/CaloRinger_Analysis_ElectronVsJet_eratioxnnoutput_electron_ac_tight.pdf}
\\
\textbf{J2} \linebreak (Tampa)&  
\includegraphics[width=0.225\textwidth]{imagens/CaloRinger_Analysis_ElectronVsJet/CorrelationPlots_AC/J2_noreq/CaloRinger_Analysis_ElectronVsJet_eratioxnnoutput_jet_ac_all.pdf} &
\includegraphics[width=0.225\textwidth]{imagens/CaloRinger_Analysis_ElectronVsJet/CorrelationPlots_AC/J2_loose/CaloRinger_Analysis_ElectronVsJet_eratioxnnoutput_jet_ac_loose.pdf} &
\includegraphics[width=0.225\textwidth]{imagens/CaloRinger_Analysis_ElectronVsJet/CorrelationPlots_AC/J2_medium/CaloRinger_Analysis_ElectronVsJet_eratioxnnoutput_jet_ac_medium.pdf} &
\includegraphics[width=0.225\textwidth]{imagens/CaloRinger_Analysis_ElectronVsJet/CorrelationPlots_AC/J2_tight/CaloRinger_Analysis_ElectronVsJet_eratioxnnoutput_jet_ac_tight.pdf}
\\
\end{tabular}
}
\label{fig:singlexj2_eratio}
\caption{Correlações da saída neural para o conjunto Single\_e x J2 com: eRatio}
\end{sidewaysfigure}

%--------------------------------------------------
\begin{sidewaysfigure}[p]
\centering
\resizebox{\textwidth}{!}{
\begin{tabular}{
>{\centering\arraybackslash}m{0.10\textwidth} 
>{\centering\arraybackslash}m{0.225\textwidth}@{\hskip 0.01cm}
>{\centering\arraybackslash}m{0.225\textwidth}@{\hskip 0.01cm}
>{\centering\arraybackslash}m{0.225\textwidth}@{\hskip 0.01cm}
>{\centering\arraybackslash}m{0.225\textwidth}@{\hskip 0.01cm}
}
 & \textbf{Sem Req.} & \textbf{Loose} & \textbf{Medium} & \textbf{Tight} \\
\textbf{Single\linebreak part\_e}\linebreak (Barril) &  
\includegraphics[width=0.225\textwidth]{imagens/CaloRinger_Analysis_ElectronVsJet/CorrelationPlots_BC/Singlepart_e_noreq/CaloRinger_Analysis_ElectronVsJet_hadleakagexnnoutput_electron_bc_all.pdf} &
\includegraphics[width=0.225\textwidth]{imagens/CaloRinger_Analysis_ElectronVsJet/CorrelationPlots_BC/Singlepart_e_loose/CaloRinger_Analysis_ElectronVsJet_hadleakagexnnoutput_electron_bc_loose.pdf} &
\includegraphics[width=0.225\textwidth]{imagens/CaloRinger_Analysis_ElectronVsJet/CorrelationPlots_BC/Singlepart_e_medium/CaloRinger_Analysis_ElectronVsJet_hadleakagexnnoutput_electron_bc_medium.pdf} &
\includegraphics[width=0.225\textwidth]{imagens/CaloRinger_Analysis_ElectronVsJet/CorrelationPlots_BC/Singlepart_e_tight/CaloRinger_Analysis_ElectronVsJet_hadleakagexnnoutput_electron_bc_tight.pdf} \\
\textbf{J2} \linebreak (Barril)&  
\includegraphics[width=0.225\textwidth]{imagens/CaloRinger_Analysis_ElectronVsJet/CorrelationPlots_BC/J2_noreq/CaloRinger_Analysis_ElectronVsJet_hadleakagexnnoutput_jet_bc_all.pdf} &
\includegraphics[width=0.225\textwidth]{imagens/CaloRinger_Analysis_ElectronVsJet/CorrelationPlots_BC/J2_loose/CaloRinger_Analysis_ElectronVsJet_hadleakagexnnoutput_jet_bc_loose.pdf} &
\includegraphics[width=0.225\textwidth]{imagens/CaloRinger_Analysis_ElectronVsJet/CorrelationPlots_BC/J2_medium/CaloRinger_Analysis_ElectronVsJet_hadleakagexnnoutput_jet_bc_medium.pdf} &
\includegraphics[width=0.225\textwidth]{imagens/CaloRinger_Analysis_ElectronVsJet/CorrelationPlots_BC/J2_tight/CaloRinger_Analysis_ElectronVsJet_hadleakagexnnoutput_jet_bc_tight.pdf} \\
\textbf{Single\linebreak part\_e}\linebreak (Tampa) &  
\includegraphics[width=0.225\textwidth]{imagens/CaloRinger_Analysis_ElectronVsJet/CorrelationPlots_AC/Singlepart_e_noreq/CaloRinger_Analysis_ElectronVsJet_hadleakagexnnoutput_electron_ac_all.pdf} &
\includegraphics[width=0.225\textwidth]{imagens/CaloRinger_Analysis_ElectronVsJet/CorrelationPlots_AC/Singlepart_e_loose/CaloRinger_Analysis_ElectronVsJet_hadleakagexnnoutput_electron_ac_loose.pdf} &
\includegraphics[width=0.225\textwidth]{imagens/CaloRinger_Analysis_ElectronVsJet/CorrelationPlots_AC/Singlepart_e_medium/CaloRinger_Analysis_ElectronVsJet_hadleakagexnnoutput_electron_ac_medium.pdf} &
\includegraphics[width=0.225\textwidth]{imagens/CaloRinger_Analysis_ElectronVsJet/CorrelationPlots_AC/Singlepart_e_tight/CaloRinger_Analysis_ElectronVsJet_hadleakagexnnoutput_electron_ac_tight.pdf}
\\
\textbf{J2} \linebreak (Tampa)&  
\includegraphics[width=0.225\textwidth]{imagens/CaloRinger_Analysis_ElectronVsJet/CorrelationPlots_AC/J2_noreq/CaloRinger_Analysis_ElectronVsJet_hadleakagexnnoutput_jet_ac_all.pdf} &
\includegraphics[width=0.225\textwidth]{imagens/CaloRinger_Analysis_ElectronVsJet/CorrelationPlots_AC/J2_loose/CaloRinger_Analysis_ElectronVsJet_hadleakagexnnoutput_jet_ac_loose.pdf} &
\includegraphics[width=0.225\textwidth]{imagens/CaloRinger_Analysis_ElectronVsJet/CorrelationPlots_AC/J2_medium/CaloRinger_Analysis_ElectronVsJet_hadleakagexnnoutput_jet_ac_medium.pdf} &
\includegraphics[width=0.225\textwidth]{imagens/CaloRinger_Analysis_ElectronVsJet/CorrelationPlots_AC/J2_tight/CaloRinger_Analysis_ElectronVsJet_hadleakagexnnoutput_jet_ac_tight.pdf}
\\
\end{tabular}
}
\label{fig:singlexj2_hadleakage}
\caption{Correlações da saída neural para o conjunto Single\_e x J2 com: Rhad1}
\end{sidewaysfigure}

%--------------------------------------------------
\begin{sidewaysfigure}[p]
\centering
\resizebox{\textwidth}{!}{
\begin{tabular}{
>{\centering\arraybackslash}m{0.10\textwidth} 
>{\centering\arraybackslash}m{0.225\textwidth}@{\hskip 0.01cm}
>{\centering\arraybackslash}m{0.225\textwidth}@{\hskip 0.01cm}
>{\centering\arraybackslash}m{0.225\textwidth}@{\hskip 0.01cm}
>{\centering\arraybackslash}m{0.225\textwidth}@{\hskip 0.01cm}
}
 & \textbf{Sem Req.} & \textbf{Loose} & \textbf{Medium} & \textbf{Tight} \\
\textbf{Single\linebreak part\_e}\linebreak (Barril) &  
\includegraphics[width=0.225\textwidth]{imagens/CaloRinger_Analysis_ElectronVsJet/CorrelationPlots_BC/Singlepart_e_noreq/CaloRinger_Analysis_ElectronVsJet_width2xnnoutput_electron_bc_all.pdf} &
\includegraphics[width=0.225\textwidth]{imagens/CaloRinger_Analysis_ElectronVsJet/CorrelationPlots_BC/Singlepart_e_loose/CaloRinger_Analysis_ElectronVsJet_width2xnnoutput_electron_bc_loose.pdf} &
\includegraphics[width=0.225\textwidth]{imagens/CaloRinger_Analysis_ElectronVsJet/CorrelationPlots_BC/Singlepart_e_medium/CaloRinger_Analysis_ElectronVsJet_width2xnnoutput_electron_bc_medium.pdf} &
\includegraphics[width=0.225\textwidth]{imagens/CaloRinger_Analysis_ElectronVsJet/CorrelationPlots_BC/Singlepart_e_tight/CaloRinger_Analysis_ElectronVsJet_width2xnnoutput_electron_bc_tight.pdf} \\
\textbf{J2} \linebreak (Barril)&  
\includegraphics[width=0.225\textwidth]{imagens/CaloRinger_Analysis_ElectronVsJet/CorrelationPlots_BC/J2_noreq/CaloRinger_Analysis_ElectronVsJet_width2xnnoutput_jet_bc_all.pdf} &
\includegraphics[width=0.225\textwidth]{imagens/CaloRinger_Analysis_ElectronVsJet/CorrelationPlots_BC/J2_loose/CaloRinger_Analysis_ElectronVsJet_width2xnnoutput_jet_bc_loose.pdf} &
\includegraphics[width=0.225\textwidth]{imagens/CaloRinger_Analysis_ElectronVsJet/CorrelationPlots_BC/J2_medium/CaloRinger_Analysis_ElectronVsJet_width2xnnoutput_jet_bc_medium.pdf} &
\includegraphics[width=0.225\textwidth]{imagens/CaloRinger_Analysis_ElectronVsJet/CorrelationPlots_BC/J2_tight/CaloRinger_Analysis_ElectronVsJet_width2xnnoutput_jet_bc_tight.pdf} \\
\textbf{Single\linebreak part\_e}\linebreak (Tampa) &  
\includegraphics[width=0.225\textwidth]{imagens/CaloRinger_Analysis_ElectronVsJet/CorrelationPlots_AC/Singlepart_e_noreq/CaloRinger_Analysis_ElectronVsJet_width2xnnoutput_electron_ac_all.pdf} &
\includegraphics[width=0.225\textwidth]{imagens/CaloRinger_Analysis_ElectronVsJet/CorrelationPlots_AC/Singlepart_e_loose/CaloRinger_Analysis_ElectronVsJet_width2xnnoutput_electron_ac_loose.pdf} &
\includegraphics[width=0.225\textwidth]{imagens/CaloRinger_Analysis_ElectronVsJet/CorrelationPlots_AC/Singlepart_e_medium/CaloRinger_Analysis_ElectronVsJet_width2xnnoutput_electron_ac_medium.pdf} &
\includegraphics[width=0.225\textwidth]{imagens/CaloRinger_Analysis_ElectronVsJet/CorrelationPlots_AC/Singlepart_e_tight/CaloRinger_Analysis_ElectronVsJet_width2xnnoutput_electron_ac_tight.pdf}
\\
\textbf{J2} \linebreak (Tampa)&  
\includegraphics[width=0.225\textwidth]{imagens/CaloRinger_Analysis_ElectronVsJet/CorrelationPlots_AC/J2_noreq/CaloRinger_Analysis_ElectronVsJet_width2xnnoutput_jet_ac_all.pdf} &
\includegraphics[width=0.225\textwidth]{imagens/CaloRinger_Analysis_ElectronVsJet/CorrelationPlots_AC/J2_loose/CaloRinger_Analysis_ElectronVsJet_width2xnnoutput_jet_ac_loose.pdf} &
\includegraphics[width=0.225\textwidth]{imagens/CaloRinger_Analysis_ElectronVsJet/CorrelationPlots_AC/J2_medium/CaloRinger_Analysis_ElectronVsJet_width2xnnoutput_jet_ac_medium.pdf} &
\includegraphics[width=0.225\textwidth]{imagens/CaloRinger_Analysis_ElectronVsJet/CorrelationPlots_AC/J2_tight/CaloRinger_Analysis_ElectronVsJet_width2xnnoutput_jet_ac_tight.pdf}
\\
\end{tabular}
}
\label{fig:singlexj2_width2}
\caption{Correlações da saída neural para o conjunto Single\_e x J2 com: wEta2}
\end{sidewaysfigure}




\subsection{\texorpdfstring{$\text{J}/\Psi \times$ Minbias}{JPsi x Minbias}}
\label{ssec:jpsi}


\subsection{\texorpdfstring{Z$\rightarrow$ee $\times$ JetTauEtmiss}{Zee x
JetTauEtMiss}}
\label{ssec:Zee}




