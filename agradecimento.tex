\clearpage

\begin{center}
\textbf{\Large Agradecimento}
\end{center}

Agradeço em primeiro lugar aos meus pais, Ronaldo e Cristina, 
pela educação e apoio que me deram. Aos meus irmãos que sempre
estiveram comigo, enriquecendo os bons momentos e me divertindo. Aos meus 
avôs Ronaldo e Gisela, por sempre terem me acolhido.

Aos meus orientadores R. Torres, D. Damazio, J. Seixas pelo conhecimento
transmitido, pelo incentivo que me deram, e pela paciência. Obrigado 
por terem acreditado e por tudo que fizeram por mim. Muito mais do que
orientadores, a presença de vocês sempre é agradável. Agradeço ao E. Simas pela
ajuda dada com o material e pela compreensão durante a época de sua tese com a
dificuldade de acertar os dados, nem sempre as coisas são tão simples quanto nós
queremos.

Também aos bons professores da elétrica que me ensinaram e trouxeram 
verdadeiros desafios a serem superados. Graças a vocês me sinto preparado para
qualquer dificuldade que venha a aparecer no futuro.
À Kátia, secretária da elétrica, que tem uma paciência sem fim para lidar com as
perguntas um tanto ou quanto repetitivas dos alunos, desde sua época de calouros
até a colação de grau. 

\newacronym[type=Abrev]{ufrj}{UFRJ}{Universidade Federal do Rio de Janeiro}
\newacronym[type=Abrev]{coppe}{COPPE}{instituto alberto luiz COimbra de
  Pós-graduação e Pesquisa de Engenharia}
\newacronym[type=Abrev]{lps}{LPS}{Laboratório de Processamento de Sinais -
 \acrshort{coppe}/\acrshort{ufrj}} 

Aos meus companheiros do \acrshort{lps}, D. Deva, D. Lima, 
dentre outros, pela contribuição e ajuda que me deram. Valhe destacar aqui que a
implementação inicial do código no Sistema de Reconstrução do Ringer foi feita pelo D. Lima
e ele em muito me auxiliou no entendimento do código complexo desse ambiente. 
Ainda, agradeço aos profissionais que nesse 
laboratório trabalham, principalmente a Talia, cuja dedicação se destaca.

Finalmente, agradeço a todos meus amigos que sempre me ajudaram quando precisei, 
eu lhes devo muito.

Muito obrigado a todos vocês!
