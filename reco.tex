\chapter{Reconstrução da Física do Experimento ATLAS}
\label{cap:reco}

A alta energia e luminosidade do \gls{lhc} (Seção~\ref{sec:lhc}) oferecem um alto alcance para
explorar física, desde a medição precisa de propriedades de objetos conhecidos, a
ultrapassar a fronteira de energias onde a física ainda não havia sido experimentada.
Por ser um detector de propósito geral, o \gls{atlas} (Seção~\ref{sec:ATLAS}), deve ser capaz de
explorar diversos dos objetivos do \gls{lhc} (Subseção~\ref{ssec:obj_lhc}), e
para isso ele precisar ter a capacidade de explorar um largo espectro de assinaturas
físicas. Esse fato guiou a otimização do projeto do \gls{atlas}, dando as especificações de sensibilidade e precisão do
detector para que seja possível através das assinaturas realizar a reconstrução
com precisão da física ocorrida nas colisões. O foco principal é na busca pelo
bóssom de Higgs \cite{ATLAS_TDR2}, com o objetivo de provar a origem da escala
eletrofraca\footnote{A escala eletrofraca existe somente no estado primitivo descrito na
Subseção~\ref{ssec:bossoms}.}. 

A física se apoia em algoritmos
para realizar a busca pelas assinaturas, sendo este capítulo dedicado aos
algoritmos que fazem a reconstrução do Canal \gls{eg} (Seção~\ref{sec:egamma}),
onde se deseja encontrar as assinaturas dessas partículas.
Outras assinaturas experimentais são as de hádrons e jatos, que serão tratadas
quando abordando o Canal \gls{eg}, por serem o ruído de fundo do qual se deseja
filtrar nesse canal. Entretanto existem setores de física que se
interessam por esses decaimentos, como o já citado decaimento do bóssom W em jatos
duplos\footnote{A citação foi realizada no Subtópico~\ref{par:cal_part_id}.}. 
Além disso, múons constituem de outras assinaturas, uma vez que interagem
pouco com os calorímetros, tendo seu próprio subsistema, o Espectômetro de
Múons. Existem outras três assinaturas "indiretas": 
táons, sabores pesados e léptons neutrinos. Os táons decaem
rapidamente em múons (ou elétrons) e mais dois neutrinos, ou ainda, em um ou três
píons carregados acompanhados por traços isolados, de forma que sua assinatura é
feita nos espectometros de múons (ou calorímetro), ou de chuveiros \gls{had} com um ou três
traços isolados apontando para o chuveiro. Os sabores pesados procuram
assinaturas de mésons pesados, que geralmente decaem a menos de 1 mm do ponto de
colisão, sendo necessário criar um vértice secundário que será identificado com
a alta precisão do Detector de Pixel. Finalmente, léptons neutrinos escapam da
deteção, para sua assinatura é utilizada a propriedade de conservação de
\gls{pt}, de modo que, quando há quantidade suficiente de \gls{ptmiss}, há indicios
de que algo escapou do processo de deteção.

Os algoritmos são implementados em dois ambientes diferentes, um ambiente mais
restrito, necessário devido ao alto nível de ruído físico gerado nas condições 
impostas do \gls{lhc} (citadas no Tópico~\ref{sssec:minb_ue_pileup}): o Sistema de 
Filtragem (Seção~\ref{sec:sf}). Nesse ambiente o tempo de discriminação 
é limitado a latência suportada pela capacidade computacional das máquinas localizadas 
no \gls{ip}1. O trabalho segue com o desenvolvimento do algoritmo HLT\_EgammaCaloRinger 
(Subseção~\ref{ssec:hlt_ringer}), um algoritmo alternativo proposto para o
Segundo Nível de Filtragem desse sistema. A versão implementada pela colaboração, o T2Calo (Subseção~\ref{ssec:t2calo}), 
também é descrita em detalhes. Por sua vez o outro ambiente, o Sistema de Reconstrução
(\ref{sec:sr}), visa fazer uma análise mais precisa dos eventos de colisão e da 
física gerada, sendo o Sistema de Filtragem, nos níveis mais altos de filtragem, 
nada mais que uma versão restringida baseada no mesmo. Para que a Colaboração tenha 
um melhor contato com o algoritmo proposto e entenda o seu funcionamento, se fez
necessário a implementação (Tópico~\ref{sssec:egringer_impl}) de uma versão no Sistema 
de Reconstrução (Subseção~\ref{ssec:egringer}) para que os físicos o utilizassem nas suas 
análises a posteriori. A versão padrão (Subseção~\ref{ssec:egamma}) utilizada
pelos físicos também foi abordada.


\section{\texorpdfstring{O Canal Elétron/Fóton (e/$\gamma$)}{O Canal
Elétron/Fóton (Egamma)}}
\label{sec:egamma}

As assinaturas experimentais de partículas \gls{em} e \gls{had} foram 
discutidas em detalhes quando se referindo a calorímetria no \gls{em}
Tópico~\ref{sssec:intro_cal}. Fazendo um breve resumo, quando em contato com o
calorímetro, essas partículas criam chuveiros de particulas que terão sua energia 
absorvida pelo mesmo. Os chuveiros eletromagnéticos serão completamente absorvidos pelo
\gls{ecal}, já os chuveiros de partículas \gls{had}



%\subsection{Procura pelo bóssom de Higgs}
%\label{ssec:proc_higgs}

% Explicar aqui os decaimentos do bóssom de Higgs e a relação sinal ruído,
% utilizando o TDR2

%\subsection{A partícula $\text{J}/\Psi$}

% Falar brevemente do Jpsi

\section{O Sistema de Filtragem}
\label{sec:sf}


\subsection{T2Calo}
\label{ssec:t2calo}

% Falar sobre Fex 
% Colocar rcore, eratio etc, fazendo referencia ao processo de geracao dos
% chuveiros

% Falar sobre Hypo
% cortes lineares, estudados a partir de Monte Carlo

\subsection{HLT\_EgammaCaloRinger (Ringer\_HLT)}
\label{ssec:hlt_ringer}

% Explicar o processo de anelamento

% Falar sobre redes neurais

% Pre-processamento: normalizacao, ica etc

\section{O Sistema de Reconstrução}
\label{sec:sr}

% Falar aqui que o Sistema de Filtragem é uma versão mais simples do Sistema de
% Reconstrução

\subsection{\texorpdfstring{$e/\gamma$ Regular}{Egamma Regular}}
\label{ssec:egamma}


% Falar que tem mais opções que 
% Colocar os requerimentos, tight, loose, medium.


\subsection{\texorpdfstring{$e/\gamma$ \emph{Calorimeter Ringer}
(EgCaloRinger)}{Egamma Calorimeter Ringer (EgCaloRinger)}}
\label{ssec:egringer}

% Falar aqui da request feito pela colaboracao

\subsubsection{Implementação}
\label{sssec:egringer_impl}
% Implementação



