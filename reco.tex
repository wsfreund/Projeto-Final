\chapter[Reconstrução da Física do Canal eGamma do Experimento ATLAS]
{Reconstrução da Física do Canal e/$\gamma$ do Experimento ATLAS}
\label{cap:reco}

\newacronym[type=Abrev]{t2calo}{\emph{T2Calo}}{Algoritmo para a Reconstrução do
Calorímetro no Segundo Nível de Filtragem} 
\newacronym[type=Abrev]{hltringer}{\emph{HLT\_Ringer}}{HLT\_EgammaCaloRinger} 
\glsunset{t2calo}

A alta energia e luminosidade do \gls{lhc} (Seção~\ref{sec:lhc}) oferecem um alto alcance para
explorar física, desde a medição precisa de propriedades de objetos conhecidos, a
ultrapassar a fronteira de energias onde a física ainda não havia sido experimentada.
Por ser um detector de propósito geral, o \gls{atlas} (Seção~\ref{sec:ATLAS}), deve ser capaz de
explorar diversos dos objetivos do \gls{lhc} (Subseção~\ref{ssec:obj_lhc}), e
para isso ele precisar ter a capacidade de explorar um largo espectro de assinaturas
físicas. Esse fato guiou a otimização do projeto do \gls{atlas}, dando as especificações de sensibilidade e precisão do
detector para que seja possível através das assinaturas realizar a reconstrução
com precisão da física ocorrida nas colisões. O foco principal é na busca pelo
bóssom de Higgs \cite{ATLAS_TDR2}, com o objetivo de provar a origem da escala
eletrofraca\footnote{A escala eletrofraca existe somente no estado primitivo descrito na
Subseção~\ref{ssec:bossoms}.}. 

A física se apoia em algoritmos
para realizar a busca pelas assinaturas, sendo este capítulo dedicado aos
algoritmos que fazem a reconstrução do Canal \gls{eg},
onde se deseja encontrar as assinaturas dessas partículas (e dos pósitrons, que
tem a mesma assinatura de elétrons, excluindo a direção de deflexão de seus
traços, oposta àquela de elétrons devido a sua carga positiva). A identificação
dessas partículas é de fundamental importância por formarem os canais de
decaimentos mais limpos do bóssom de Higgs (Sessão~\ref{sec:busca_higgs}). Além
disso, outras partículas como o bóssom Z e partícula $\text{J}/\Psi$, podem ser
utilizadas para melhorar a escala de medida de energia dos calorímetros e
estudar sua linearidade de resposta por terem suas massas bem conhecidas, 
em especial os decaimentos de elétrons do
$\text{J}/\Psi$ para baixas energias ($\gls{Et} > 5$ GeV), onde se esperam encontrar 
decaimentos leptônicos do bóssom de Higgs de menor energia ($\gls{Et} > 7$ GeV).
Outras assinaturas experimentais são as de hádrons e jatos, que serão tratadas
quando abordando o Canal \gls{eg}, por serem o ruído de fundo do qual se deseja
filtrar nesse canal. Entretanto existem setores de física que se
interessam por esses decaimentos, como o já citado decaimento do bóssom W em jatos
duplos\footnote{A citação foi realizada no Subtópico~\ref{par:cal_part_id}.}. 
Além disso, múons constituem de outras assinaturas, uma vez que interagem
pouco com os calorímetros, tendo seu próprio subsistema, o Espectômetro de
Múons. Existem outras três assinaturas:
táons, sabores pesados e léptons neutrinos. Os táons decaem
rapidamente em múons (ou elétrons) e mais dois neutrinos, ou ainda, em um ou três
píons carregados, de forma que sua assinatura é
feita nos espectometros de múons (ou calorímetro), ou de chuveiros \gls{had} com um ou três
traços isolados apontando para o chuveiro. Os sabores pesados procuram
assinaturas de mésons pesados, que geralmente decaem a menos de 1 mm do ponto de
colisão, sendo necessário criar um vértice secundário que será identificado com
a alta precisão do Detector de Pixel. Finalmente, léptons neutrinos escapam da
deteção, para sua assinatura é utilizada a propriedade de conservação de
\gls{pt}, de modo que, quando há quantidade suficiente de \gls{ptmiss}, há indicios
de que algo escapou do processo de deteção.

Os algoritmos são implementados em dois ambientes diferentes, um ambiente mais
restrito, necessário devido ao alto nível de ruído físico gerado nas condições 
impostas do \gls{lhc} (citadas no Tópico~\ref{sssec:minb_ue_pileup}): o
\glsdesc{sf} (Seção~\ref{sec:sf}). Nesse ambiente o tempo de discriminação 
é limitado a latência suportada pela capacidade computacional das máquinas localizadas 
no \gls{ip}1. O trabalho segue com o desenvolvimento do algoritmo
\acrlong{hltringer} (Subseção~\ref{ssec:hlt_ringer}), um algoritmo alternativo proposto para o
Segundo Nível de Filtragem desse sistema. A versão implementada pela
colaboração, o \gls{t2calo} (Subseção~\ref{ssec:cadeia_egamma}), 
também é descrita em detalhes. Por sua vez o outro ambiente, o \glsdesc{sr}
(Seção~\ref{sec:sr}), visa fazer uma análise mais precisa dos eventos de colisão e da 
física gerada, sendo o \glsdesc{sf}, nos níveis mais altos de filtragem, 
nada mais que uma versão restringida baseada no mesmo. Para que a Colaboração tenha 
um melhor contato com o algoritmo proposto e entenda o seu funcionamento, se fez
necessário a implementação (Tópico~\ref{sssec:egringer_impl}) de uma versão no Sistema 
de \glsdesc{sr} (Subseção~\ref{ssec:egringer}) para que os físicos o utilizassem nas suas 
análises a posteriori. A versão padrão (Subseção~\ref{ssec:egamma}) utilizada
pelos físicos também foi abordada.


\section{Busca pelo bóssom de Higgs}
\label{sec:busca_higgs}

\newacronym[type=Simb]{mh}{\ensuremath{m_H}}{massa do bóssom de Higgs} 
\newacronym[type=Abrev]{lep}{LEP}{\emph{Large Electron Positron Collider}} 
\glsunset{mh}

O bóssom de Higgs no setor do \gls{mp} é amplamente irrestrito, de forma que sua
massa, \acrshort{mh}, não é prevista pela teoria. Um limite superior na ordem de
$\sim$1 TeV teórico, e um limite inferior experimental\footnote{As exclusões
citadas foram feitas para um nível de confiança de 95\%.\label{fn:95cl}} de 95,2
GeV \cite{lep_higgs_1999} por um experimento anterior do \gls{cern}, o \gls{lep}, determinam a região para a qual
se esperava encontrar o bóssom de Higgs durante a época das últimas propostas técnicas
do projeto do \gls{atlas} \cite{ATLAS_TDR,ATLAS_TDR2}, por volta de 1999. 

Um estudo do potêncial de descoberta do bóssom de Higgs através de seus
possíveis canais de decaimentos explorados pelo programa de física do
\gls{atlas} foi revisado na propostas citadas. Esses canais são:

\begin{itemize}
\item $H\rightarrow\gamma\gamma$ (110 a 150 GeV/$c^2$), produção direta;
\item $H\rightarrow\gamma\gamma$ (110 a 150 GeV/$c^2$), da produção associada $WH$, $ZH$ e
$t\bar{t}H$, usando um lépton (e, $\mu$) rotulado do decaimento dos bóssom W, Z ou
do quark \emph{top};
\item $H\rightarrow b\bar{b}$ (110 a 130 GeV/$c^2$), da produção associada $WH$, $ZH$ e
$t\bar{t}H$, usando um lépton (e, $\mu$) rotulado do decaimento dos bóssom W, Z ou
do quark \emph{top};
\item $H\rightarrow ZZ^*\rightarrow4l$ (110 a 600 GeV/$c^2$. O símbolo $^*$ representa uma partícula 
que está fora de sua fronteira de massa\footnote {Na literatura em inglês 
\emph{off the mass shell}, ou simplesmente \emph{off
shell}.} na relação massa-momento dada pela equação~\ref{eq:rel_mom}. Essas
partículas só existem como estados intermediários, constituindo partículas
virtuais, e estão de acordo com o princípio da incerteza, em que as partículas
só podem existir nesse estado durante um tempo menor que uma fração da
constante 
de Plank pela diferença de energia, ou $\Delta E \Delta t \gtrsim h$), onde l
representa estes léptons: $e,\mu$; 
\item $H\rightarrow ZZ\rightarrow4l$ e $H\rightarrow ZZ\rightarrow ll\nu\nu$i
(200 a 600 GeV/$c^2$), aqui $\nu$ são os léptons neutrinos;
\item $H\rightarrow WW\rightarrow l\nu jj$ e $H\rightarrow ZZ\rightarrow ll jj$
($\sim$600 a 1 TeV),
onde $jj$ representa o decaimento em jatos duplos;
\item $H\rightarrow WW^*\rightarrow l\nu l\nu$ (110 a 300 GeV/$c^2$). 
\end{itemize}


Nesse estudo se utilizou simulações
de Monte Carlo do \emph{Pythia}, através da significância desses 
canais numa região de 80-1000 GeV/$c^2$ (lembrando que na época só havia sido excluído
o bóssom de Higgs para \gls{mh} < 95,2), utilizando os valores de resposta dos
subdectores, resolução e capacidade esperadas de identificação de sinais e rejeição 
de ruídos redutivéis. 
Os resultados desse estudo estão resumidos na Figura~\ref{fig:higgs_sim_atlas},
mostrando a significância do sinal para uma luminosidade integrada\footnote{A 
luminosidade integrada, como seu nome indica, não passa da integral da luminosidade 
instantânea no tempo de operação do \gls{lhc}: $\int{L_{inst} dt}$.} de
100 $\text{fb}^{-1}$. Para efeito de comparação na
Figura~\ref{fig:higgs_branch_ratio} há as relações entre os canais
e seus alcânces em massa. A Figura~\ref{fig:higgs_production} contêm a seção
de choque esperada para as possíveis massas.

\begin{figure}[ht!]
    \label{fig:higgs_expected}
    \begin{center}
%
        \subfigure[]{%
            \label{fig:higgs_branch_ratio}
            \includegraphics[width=0.47\textwidth,height=6cm]{imagens/higgs_branch_ratio.pdf}
        }%\hspace{0.1\textwidth}
        \subfigure[]{
            \label{fig:higgs_production}
            \includegraphics[width=0.47\textwidth,height=6cm]{imagens/higgs_production.pdf}
        }\\
        \subfigure[]{%
            \label{fig:higgs_sim_atlas}
            \includegraphics[width=.6\textwidth]{imagens/higgs_sim_atlas.pdf}
        }
    \end{center}
\caption[Simulação da significância dos canais de decaimento do bóssom
de Higgs explorados pelos ATLAS, as relações de seus canais e alcances em massa,
e a seção de choque do mesmo.]
{Os decaimentos do bóssom de Higgs do MP e suas relações de
seus canais e alcances em massa,~\ref{fig:higgs_branch_ratio}, e os seus valores de seção de choque para suas
repectivas massas,~\ref{fig:higgs_production}, ambos extraídas de \cite{lhc_higgs_group}. 
Em \ref{fig:higgs_sim_atlas} está uma simulação da significancia dos possíveis canais de decaimento do
bóssom de Higgs, explorados pelo ATLAS, para uma luminosidade integrada de
$100fb^{-1}$, utilizando as eficiências esperadas para a deteção das
assinaturas dos estados finais e resoluções dos subdetectores, extraído de
\cite{ATLAS_TDR2}.}
\end{figure}

Observe que decaimentos do bóssom de Higgs mais frequentes não são,
necessariamente, os canais com melhor significância esperada. Aqueles canais em
que se tem mais facilidade de observar os decaimentos finais e poder associá-los
a produção de um bóssom de Higgs podem ter mais significância do que canais com
maior produção para uma dada massa. Isso reflete, por outro lado, na importância
dos algoritmos de reconstrução, obtendo melhor eficiência na reconstrução dos
estados finais facilitariam a encontrar o bóssom com menor quantidade de dados.
Os canais mais importantes nas regiões de massa intermediária, $\gls{mh}<2m_Z$,
onde se espera que os decaimentos reconstruam um pico de massa, são 
o canal de quatro léptons, $H\rightarrow ZZ^* \rightarrow 4l$, o canal de
produção direta em dois fótons, $H\rightarrow \gamma\gamma$, junto com os
canais de produção associada com o mesmo estados finais, . Para massas por volta
de 170 GeV/$c^2$, onde a produção de $ZZ^*$ é suprimida, o potêncial da descoberta do
Higgs pode ser elevada pelo decaimento $H\rightarrow WW^*\rightarrow l\nu l\nu$,
nesse caso, o sinal deverá ser observado como um excesso de eventos. O canal de
quatro léptons com dois bóssoms Z reais é o dominante na região de
$\gls{mh}>2m_Z$, cobrindo uma vasta região de massa\footnote{Por isso chegou a
ser referênciado como canal banhado a ouro.}. Em massas mais elevadas, por volta de 600 GeV/$c^2$ até 1
TeV, outros canais como $H\rightarrow WW\rightarrow l\nu jj$, $H\rightarrow
WW\rightarrow ll\nu\nu$ são utilizados para a busca do bóssom de Higgs.

Posteriormente, outro canal foi adicionado, por outros estudos
\cite{atlas_tautau,atlas_tautau2}:

\begin{itemize} 
\item $H\rightarrow \tau \tau \rightarrow ll4\nu$ e $H\rightarrow \tau \tau
\rightarrow l\tau_{had}3\nu$ (110 a 150 GeV/$c^2$), onde $\tau_{had}$ seria o
decaimento do táon em píons especificados no início deste capítulo.
\end{itemize} 

Assim, a reconstrução de elétrons e múons fazem parte de grande
parte dos canais explorados para a vasta região de massa em que se espera
encontrar o bóssom de Higgs, inclusive um dos canais
mais importante por cobrir uma vasta região de massa. Se o Higgs existir com
uma massa mais baixa, os fótons serão necessários para poder encontrar o bóssom,
mostrando a importância do Canal \gls{eg}. 

\begin{figure}[t!]
    \label{fig:higgs_exclusion}
    \begin{center}
    \includegraphics[width=0.8\textwidth]{imagens/exclusion_cms_atlas.pdf}
%
%        \subfigure[LEP, extraído de \cite{lep_higgs_2008}.]{%
%            \label{fig:lep_exclusion}
%            \includegraphics[width=0.47\textwidth]{imagens/lep_exclusion.pdf}
%        }%\hspace{0.1\textwidth}
%        \subfigure[Tevatron, extraído de \cite{tevatron_higgs}.]{%
%            \label{fig:tevatron_exclusion}
%            \includegraphics[width=0.47\textwidth]{imagens/tevatron_exclusion.pdf}
%        }\\%\hspace{0.1\textwidth}
%        \subfigure[ATLAS, extraído de \cite{atlas_higgs}.]{
%            \label{fig:atlas_exclusion}
%            \includegraphics[width=0.47\textwidth]{imagens/atlas_exclusion.pdf}
%        }
%        \subfigure[CMS, extraído de \cite{cms_higgs}.]{%
%            \label{fig:cms_exclusion}
%            \includegraphics[width=0.47\textwidth]{imagens/cms_exclusion.pdf}
%        }
    \end{center}
\caption[Exclusões de massa do Higgs realizada pela combinação dos dados do
ATLAS e CMS. As exclusões de massa do Tevatron e LEP também estão indicadas.]{As exclusões de massa do bóssom de Higgs. Os valores de massa
observados, no caso para os resultados combinados do ATLAS e CMS, quando
abaixo da linha horizonal vermelha indicada são excluídos com um nível de
confiança de 95\%. A linha pontilhada indica o ruído de fundo esperado gerado
por física na ausência do bóssom de Higgs. As faixas verdes e amarelas indicam
um nível de certeza de 68\% e 95\% de certeza de que algo fora do comum está
ocorrendo. Caso a linha observada ultrapasse o limite superior da faixa amarela
em regiões acima da linha vermelha, então há 95\% de certeza de que há indicios
do bóssom de Higgs ou algum processo de física (talvez ruído físico) não bem
conhecido. Extraído de \cite{atlas_cms_higgs}.}
\end{figure}

Resultados mais recentes excluiram outras regiões de massa do bóssom de Higgs. Em 2008, 
o legado deixado pelo \gls{lep} elevou o limite inferior para 114 GeV/$c^2$
\cite{lep_higgs_2008}, enquanto o experimento Tevatron realizado pelo Fermilab, excluiu as
regiões de $156 < \gls{mh} < 177$ GeV/$c^2$, em Setembro de 2011
\cite{tevatron_higgs}. Também no final de Setembro, resultados preliminares do
\gls{atlas} excluiram o bóssom de Higgs para as regiões de 146-230 GeV/$c^2$, 256-282 GeV/$c^2$ 
e 296-459 GeV/$c^2$ \cite{atlas_higgs}. Um mês antes o \gls{cms} havia excluido o bóssom para
as regiões parecidas: de 145-216 GeV/$c^2$, 226-288 GeV/$c^2$ e 310-400 GeV/$c^2$
\cite{cms_higgs}. Em meados de Novembro o \gls{atlas} e \gls{cms} uniram seus
dados para excluir a região de 141-476 GeV/$c^2$ \cite{atlas_cms_higgs}. Esses
resultados estão indicados na Figura~\ref{fig:higgs_exclusion}, onde estão os
resultados mais recentes do \gls{lhc}, incluidas das regiões anteriormente
excluídas por outros experimentos.


\section{O Sistema de Filtragem (SF)}
\label{sec:sf}

\newacronym[type=Abrev]{asic}{ASIC}{Circuitos Integrados de Aplicação Específica}

O \gls{atlas}, para a luminosidade na qual tem operado: $\sim10^{33}cm^{-2}s^{-1}$,
gera cerca de 19 colisões inelásticas para cada cruzamento entre
feixes, que irão se cruzar numa frequência máxima de 40 MHz. O propósito do
\gls{sf} \cite{trigger_perf_2010,trigger_tdr} é de reduzir essa taxa a cerca
de 200 Hz para seu armazenamento e processamento a posteriori. Esse
limite, correspondendo uma média de $\sim$300 MB/s, é determinado pelos recursos
computacionais para armazenamento e processamento dos dados. É possível gravar
dados com taxas relativamentes mais altas por períodos mais curtos, como foi
realizado durante 2010 quando a física foi beneficiada utilizando taxas de saída
de até $\sim$600 Hz. Nessa época, a luminosidade instantânea máxima atingida foi de
$\sim10^{32}cm^{-2}s^{-1}$, gerando cerca de $\sim$ 1,3 MB, e o \gls{lhc} não 
operava de forma continua -- o que é comum para um acelerador novo -- permitindo 
elevar a taxa de armazenamento.

\subsection{A estrutura do Sistema de Filtragem}
\label{ssec:estru_sf}
\newacronym[type=Abrev]{rob}{ROB}{\emph{Buffer} de Leitura}
\newacronym[type=Abrev]{ros}{ROS}{Sistemas de Leitura}

O \gls{sf}, Figura~\ref{fig:sf_esboco}, é dividido em três níveis sequenciais,
sendo descrito aqui resumidamente uma vez que já foi explicada em maiores detalhes em
trabalhos anteriores como \cite{tese_eduardo,tese_torres}, também podendo ser encontrada 
em detalhes nas referências \cite{trigger_tdr,l1_trigger_tdr,l2_ef_daq_dcs_tdr,trigger_perfomance}.

Os sinais do detector são armazenados em memórias \emph{pipeline}, com pendência
à decisão do \acrshort{l1}. Para obter uma latência de menos de 2,5 $\mu$s, 
o \gls{l1} é implementado em rápida eletrônica constituída de \gls{fpga} e
\gls{asic}, de forma a reduzir a taxa para uma valor máximo de 75 kHz. Para
aumentar a velocidade, esse nível utiliza apenas a informação dos subsistemas de
calorimetria e câmaras de mûons com granularidade reduzida. Além do
primeiro passo de seleção, o \gls{l1} identifica as \glspl{roi} do detector a
serem investigadas pelo \acrshort{hlt}. 

\begin{figure}[ht!]
\label{fig:sf_esboco}
\centering
\includegraphics[width=0.6\textwidth]{imagens/sf_resumo.pdf}
\caption[Esboço do Sistema de Filtragem.]{Esboço do Sistema de Filtragem. Extraído de \cite{tese_eduardo}.}
\end{figure}

O \gls{hlt} se consiste de fazendas de
processadores comerciais conectados a rede dedicadas (\emph{Ethernet} de \emph{Gigabit} 
e 10 \emph{Gigabit}).  O \gls{sf} foi projetado para conter cerca de 500 nós para o
\acrshort{l2} e 1800 nós para o \acrshort{ef}\footnote{Em 2010, a fazenda de processamento 
se consistia de 800 nós configuraveis como tanto \acrshort{l2} ou \acrshort{ef}, este 
contendo mais 300 nós dedicados.}. Quando um evento é aceito pelo
\gls{l1}, os dados na memória de cada um dos detectores são então transferidos
para \gls{rob} específicos para cada um dos subdetectores, que armazenam o
evento em fragmentos pendendo à decisão do \gls{l2}. Um ou mais \glspl{rob} são
agrupados para formar os \gls{ros} que são conectados a rede do \gls{hlt}. O
seleção do \gls{l2} é baseada em algoritmos personalizados e rápidos que
processam os dados dos eventos de forma parcial nas regiões especificadas pelas
\glspl{roi} identificadas pelo \gls{l1}. Os processores do \gls{l2} solicitam os
dados dos \gls{ros} correspondentes aelementos do subdetector dentro de cada
\gls{roi}, reduzindo a quantidade de dados a serem transferidas e processadas no
mesmo a cerca de 2-6\% do volume total dos dados. O \gls{l2} reduz a taxa de
dados para cerca de $\sim$3 kHz com uma média de processamento de $\sim$40
ms/evento. Qualquer evento com tempo excedendo a 5 s nesse nível é perdido a
rotulado como evento excedente ao tempo limite.   

O \gls{ef} reune todos os framentos dos eventos aceitos pelo \gls{l2} das
\glspl{rob}, tendo assim acesso a toda sua informação. O \gls{ef} é, em grande
maioria, baseado nos algoritmos do \acrlong{sr} rodando em interfaces
personalizadas para o ambiente do \gls{sf}. Esse nível é projetado para reduzir
os dados para $\sim$200 Hz com uma média de processamento de $\sim$4 s/evento.
Qualquer evento com um tempo excedente a 180 s também é rotulado e perdido como
no \gls{l2}. 

Os eventos de dados selecionados pelo \gls{sf} são escritos nos fluxos de dados
inclusivos, baseados no canal de filtragem. Existem quatro canais de dados
primários: Egamma, Muons, JetTauEtMiss, Minbias, adicionados de diversos canais
de calibração. Cerca de 10\% dos eventos são escritos para um canal expresso
aonde a reconstrução pronta dos eventos é realizada, como foi dito na
Subseção~\ref{ssec:lcg}. Ainda, além da gravação de evnetos completos em um
canal, é possível escrever informação parcial de um ou mais subdetectores em um
canal, o que é normalmente realizado para a calibração do detector nos canais
destinados a esse propósito.

\subsection{Topologia dos algoritmos e sua configuração}
\label{ssec:alg_topo}

\newacronym[type=Abrev,\glslongpluralkey={Algoritmos de Extração de
Características}]{fex}{FEX}{Algoritmo de Extração de Característica}
\newacronym[type=Abrev,\glslongpluralkey={Algoritmos de Hipótese}]
{hypo}{HYPO}{Algoritmo de Hipótese}

A configuração do \gls{sf} é realizada por um \emph{menu}, que define as
\emph{cadeias} de filtragem que especificam uma sequência de passos de reconstrução e
seleção para as especificas assinaturas de filtragems requeridas pela filtragem.
A cadeia de filtragem de elétrons é ilustrada na Figura~\ref{fig:electron_chain}.
Cada \emph{cadeia} é composta por \gls{fex} que criam objetos,
como os aglomerados de células e variáveis que permitam obter uma informação
sobre a descrição do evento, e \gls{hypo} que fazem a discriminação através dos
critérios de seleção nos objetos gerados, como por exemplo a exigência de um
$\gls{pt} > 20$ GeV. As extrações de características realizadas por uma
\emph{cadeia} podem ser reutilizadas por outro cadeia, reduzindo tanto o acesso
a dados como o tempo de processamento do \gls{sf}.

\begin{figure}[ht!]
\label{fig:electron_chain}
\centering
\includegraphics[width=0.9\textwidth]{imagens/cadeia_eletron.png}
\caption[A \emph{cadeia} de filtragem de elétrons.]{A \emph{cadeia} de filtragem
de elétrons. A direita de cada nível de filtragem estão colocados exemplos de
nomes de filtros de elétrons e fótons, e o nome de sua respectiva cadeia a
direita da seta indicando a evolução da \emph{cadeia}. Adaptado de \cite{trigger_perf_2010}.}
\end{figure}

Aproximadamente 500 filtros são definidos nos \emph{menus} atuais, sendo
composto por um número de diferentes classes de filtros:

\begin{enumerate}
\item \textbf{Objetos de filtro único}: usado para determinar estados com pelo
menos um objeto característico. Por exemplo, um filtro para elétron único com um
limiar nominal superior a 3 GeV seria referido como $e3$;
\item \textbf{Objetos de filtro multiplos}: usado para determinar estados finais
com dois ou mais objetos do mesmo tipo. Por exemplo, elétrons duplos decaindo da
partícula $\text{J}/\Psi$. Esses filtros são indicados dependendo de sua
multiplicidade, por exemplo: $2e3$;
\item \textbf{Objetos de filtro combinados}: usados por estados finais de dois
ou mais objetos característicos de tipos diferentes. Por exemplo, um múon de 13
GeV adicionado de 20 GeV de \gls{Etmiss} para selecionar decaimentos
$W\rightarrow \mu\nu$, que seriam denotados de $mu13\_xe20$;
\item \textbf{Filtros Topológicos}: usados para estados finais selecionados de
duas ou mais \glspl{roi}, como no caso do decaimento da partícula
$\text{J}/\Psi$, que combina os traços das duas \glspl{roi} provenientes dos
elétrons.
\end{enumerate}

Quando se referindo a um nível em partícular, o nível (\gls{l1}, \gls{l2},
\gls{ef}) aparece como um prefixo, como por exemplo L1\_EM3 (aqui elétrons e
fótons não são diferenciados, então se utiliza o prefixo EM em ambos os casos)
para denotar um filtro de elétrons e fótons com um limiar nominal superior a 3
GeV para o \gls{l1} e L2\_e3 no caso de elétrons para um filtro com o mesmo
limiar mas para elétrons apenas no \gls{l2}. Um nome sem o prefixo de nível
refere-se a toda a \emph{cadeia}.

O controle das taxas de filtragem podem ser realizadas mudando os limiares ou
aplicando outros valores de seleção. A seletividade de um grupo de cortes
aplicados para um certo objeto de filtragem é representado pelos termos
\emph{Loose} (frouxo), \emph{Medium} (mediano), \emph{Tight} (apertado). Esses
critérios de seleção são colocados como sufixos ao nome do filtro, por exemplo,
e10\_medium. Requerimentos adicionais como isolamento, podem ser adicionados
para reduzir as taxas dos filtros. O isolamento mede a quantidade de energia, ou
o número de partículas próximos a uma assinatura, se esse valor estiver acima de
um limiar, então a partícula não está isolada. Nesse caso se adiciona a letra
'i' ao nome do filtro, como por exemplo L1\_EM20I ou e20i\_tight.

Fatores de \emph{pré-escala} podem ser adicionados a cada filtro da 
\emph{cadeia} do \gls{hlt}, de tal forma que apenas 1 em cada N eventos passando
o filtro causam que o evento seja aceito por aquele nível. A \emph{pré-escala}
controla a taxa e composição dos canais expressos. Uma série de
\emph{pré-escalas} são utilizadas baseadas em diversas regras que levam em
consideração a prioridade dos filtros em relação com as seguintes categorias:

\begin{enumerate}
\item \textbf{Filtros Primários}: filtros princípais de física, que não deveriam
ser \emph{pré-escalados};
\item \textbf{Filtros de Suporte}: filtros importantes de suporte a filtros
primários, como filtros que permitam estudar a eficiência utilizando 
seletividades mais baixas e limiar mais baixo de \gls{Et} \emph{pré-escalados}
para permitir sua gravação em disco, uma vez que a quantidade de dados que
passam esse filtro será maior;
\item \textbf{Filtros de Monitoração e Calibração}: permitindo a coleta de dados
para garantir a operação correta do \gls{sf} e dos subdetetores do \gls{atlas},
incluindo a calibração dos mesmos. 
\end{enumerate}

\newacronym[type=Abrev]{lb}{LB}{Bloco de Luminosidade} 

Esses fatores de \emph{pré-escala} devem ser modificados conforme a queda da
luminosidade durante um preenchimento do \gls{lhc} de forma a garantir a
maximização das taxas utilizadas pelos filtros, enquanto garantindo uma taxa
constante para a monitoração e calibração. Assim, as mesmas podem ser alteradas
em quaisquer momentos de uma temporada, no início de um novo \gls{lb}. Um
\gls{lb} é a unidade fundamental para a medição da luminosidade, cerca de 120 s
em 2010, na qual se considera que a mesma permaneceu constante com o valor
medido.

Flexibilidade adicional é oferecida ao definir \emph{grupos de pacotes}, separando os
filtros para colisões com pacotes emparelhados (compõe as colisões de física
regulares, contendo pacotes de feixes opostos que se encontram no ponto de
colisão simultaneamente), pacotes vazios para estudo de pedestal criado por ruído 
no detector e raios cósmicos, ou até mesmo configurações mais inusitadas, como 
exigindo pacotes desparelhados separados por no mínimo 75 ns de quaisquer outro 
pacote no feixe oposto.

\subsection{Algoritmos Padrões da \emph{Cadeia} de Filtragem de Elétrons e Fótons}
\label{ssec:cadeia_egamma}

\newacronym[type=Abrev,\glslongpluralkey={Torres de Filtragem}]
{tt}{TT}{Torre de Filtragem}

O sistema de calorimetria do \gls{atlas} cobre uma região de $|\gls{eta}| <
4,9$, enquanto o \gls{id} fornece reconstrução precisa de traços dentro de
$|\gls{eta}| < 2,5$. Os chuveiros \gls{em} são reconstruídos com melhor
performance para essa última região, a região de precisão do \gls{atlas}, 
contendo calorímetros com maior granularidade como descrito na
Subseção~\ref{ssec:calorimetria}. Assim, os filtros de elétrons e fótons
\cite{expected_perf_2011,perf_2011} do \gls{sf} atuam somente para essa região.
A cadeia de elétrons no \gls{sf} está esboçada na Figura~\ref{fig:electron_chain}, entretanto
a cadeia de fótons pode ser facilmente extrapolada ao se remover os algoritmos
referentes ao \gls{id}.

No \gls{l1}, os aglomerados de informação\footnote{No caso, um conjunto de
células identificados como \emph{clusters} na literatura em inglês.} \gls{eg} do calorímetro são retirados
utilizando granularidade reduzida, as chamadas \glspl{tt}, que cobrem uma região
de aproximadamente $\Delta\eta\times\Delta\phi\approx0,1\times0,1$, 
a Figura~\ref{fig:cal_lar_camadas} contem um esboço da \gls{tt} para o
barril do calorimetro \gls{em}. A vantagem de se utilizar \glspl{tt} se deve ao
fato de reduzir a quantidade de fluxo de informação, reduzindo o custo e
complexidade do sistema, e, ao mesmo tempo, conter o chuveiro \gls{em} inteiramente 
em cerca de uma ou duas dessas torres, podendo
assim identificar a região em que foi formado o chuveiro não prejudicando a sua
eficiência. Para cada \gls{tt}, as células do calorímetro \gls{em} e \gls{had} são somadas, com a
excessão da quarta camada da tampa do calorímetro \gls{had} e dos cintiladores.
Um algoritmo de janela deslizante formada por uma dimensão de $4\times4$
\gls{tt}, ilustrada na Figura~\ref{fig:sliding_window_l1}, 
busca pela região com a melhor deposição de energia do chuveiro por
todo o calorímetro com um passo de uma \gls{tt}.
O \gls{l1} seleciona o evento como um candidato a \gls{eg} 
quando os seguintes critérios forem satisfeitos:

\begin{figure}[ht!]
\label{fig:l1_alg}
\centering
        \subfigure[A janela deslizante, seu núcleo e as regiões de isolamento.]{
            \label{fig:sliding_window_l1}
            \includegraphics[width=0.6\textwidth]{imagens/sliding_window_l1.pdf}
        }\hspace{0.01\textwidth}
        \subfigure[Requerimento para o núcleo ser máximo local.]{%
            \label{fig:local_et}
            \includegraphics[width=0.3\textwidth]{imagens/roi_local_max.pdf}
        }
\caption[O Primeiro Nível de Filtragem para a Cadeia de Elétrons e Fótons.]
{O Primeiro Nível de Filtragem para a Cadeia de Elétrons e Fótons. Extraído de
\cite{l1_trigger_tdr}.}
\end{figure}

\begin{itemize}
\item A região deve ser um máximo local. Essa condição é importante para se
evitar multiplicidade de \glspl{roi} a serem analisadas pelo \gls{l2}.
Assim, a energia contida no núcleo deve ser
maior, ou ao menos igual, como na lógica da Figura~\ref{fig:local_et}, 
que em todos as outras regiões de $2\times2$ que podem ser formados na
janela;
\item O aglomerado formado pela dupla de torre mais energética na região
(indicados na Figura~\ref{fig:sliding_window_l1} como somas verticais ou horizontais) deve ultrapassar o limiar
\gls{em} exigido pelo filtro do \gls{l1}. A posição de \gls{eta} e \gls{phi} desse aglomerado é passado
para o \gls{l2}, junto com outros \emph{bits} indicando os critérios que foram
satisfeitos, que formarão a palavra informada para o \gls{l2} para a qual será
utilizada para a formação da \gls{roi};
\item Se isolamento for exigido: 
\begin{itemize}
\item A \gls{Et} total da região \gls{em} de
isolamento não deve ultrapassar o limiar de isolamento \gls{em};
\item A \gls{Et} total da região \gls{had} de isolamento não deve ultrapassar o
limiar de isolamento \gls{had}.
\end{itemize}
\end{itemize}


O \gls{l2}, alimentado pela posição dos
aglomerados de torres do \gls{l1}, realiza uma rápida reconstrução do
calorímetro, e no caso de elétrons, uma rápida reconstrução dos traços no \gls{id}. A
reconstrução do calorímetro trabalha de maneira semelhante aos algoritmos do
\gls{sr}, entretanto apenas a região da janela de $\Delta\eta\times\Delta\phi =
0,4 \times 0,4$, região chamada de \gls{roi}, em torno da posição alimenda pelo \gls{l1} 
é utilizada, o que reduz o tráfico de dados e aumenta a velocidade de processamento no \gls{l2}.
Algumas diferenças entre o algoritmo do \gls{l2} e o \gls{sr} se dão as
limitações no tempo de latência. O algoritmo de reconstrução do aglomerado de
células do calorímetro no \gls{l2} começa utilizando a célula mais energética da
segunda camada \gls{em} dentro da região central de $0,2 \times 0,2$, enquanto o
algoritmo de análise a posteriori utiliza uma janela deslizante para encontrar
sua semente.

No \gls{l2}, o tamanho do aglomerado é fixado para $3
\times 7$ células em $\gls{eta} \times \gls{phi}$ para o barril ($|\gls{eta}|<1,5$) e 
$5 \times 5$ na tampa ($1,5<|\gls{eta}|<2,5$). O
algoritmo a posteriori contêm tamanhos de aglomerados diferentes, usando
$3\times5$ e $3\times7$ para fótons não convertidos e convertidos,
respectivamente, em $|\gls{eta}|<1,5$, e $5\times5$ até $|\gls{eta}|<2,5$.

As energias das células podem ser corrigidas na análise a posteriori, para
problemas transientes no \emph{hardware}, como falhas de energia, etc., o que não
pode ser realizado em tempo real, o que se aplica para tanto o \gls{l2} e
\gls{ef}. Para a reconstrução de traços no \gls{l2}, um rápido reconhecimento de
padrões é utilizado ao se determinar primeiro a posição z de interação ao longo
do eixo do feixe e então realizando a combinação de pontos do traço apenas para o grupo
de pontos que apontam para a posição determinada.

Como no \gls{sr}, a reconstrução do aglomerado no \gls{ef} é realizado
utilizando um algoritmo de janela deslizante atuando nas torres contendo energia
de toda a profundidade do calorímetro somada. Após encontrada
a semente, um aglomerado é construido iniciando da segunda camada \gls{em},
com o mesmo tamanho que aqueles descritos para o algoritmo a posteriori. O
centro de distribuição de energia em \gls{eta} e \gls{phi} é calculado
utilizando o aglomerado construido na segunda camada, e então o valor dessa
posição é então propagado de forma a incluir as camadas faltantes. O algoritmo
de reconstrução de traços é feito como no algoritmo a posteriori, com uma
combinaçao dos traços começando dos pontos dos Detectores de Silicone e
\gls{trt}. 

Os algoritmos de seleção são aplicados na reconstrução do
\gls{l2} e no \gls{ef} com o objetivo de identificar bons candidatos a \gls{eg}
e rejeitar falsos alarmes provenientes de jatos. As seleções são baseadas no
formato do chuveiro nas aglomerações, tentando identificar as diferenças já
citadas entre os chuveiros \gls{em} e \gls{had} no
Subtópico~\ref{par:cal_part_id}, como: a largura dos chuveiros, onde os \gls{em}
devem ser mais estreitos; sua profundidade, no qual os \gls{had} são mais
profundos que os \gls{em} para uma partícula de mesma energia e geralmente
somente os \gls{had} deverão alcançar o calorímetro \gls{had}, ou deverão
depositar a maior parte de sua energia no calorímetro específico para a
contenção de sua energia. Para elétrons, também se utiliza critérios de seleção 
baseados em informação do traço e qualidade de casamento entre o aglomerado e o
traço. A parte do algoritmo responsável pela reconstrução do calorímetro no
\gls{l2} é chamada de \emph{T2Calo}.

\newacronym[type=Simb]{Rhad1}{\ensuremath{R_{had1}}}{Variável de corte dos
algoritmos \acrshort{eg} padrões. Ver tabela \ref{tab:cortes_em}} 
\newacronym[type=Simb]{Rhad}{\ensuremath{R_{had}}}{Variável de corte dos
algoritmos \acrshort{eg} padrões. Ver tabela \ref{tab:cortes_em}} 
\newacronym[type=Simb]{reta}{\ensuremath{R_{\eta}}}{Variável de corte dos
algoritmos \acrshort{eg} padrões. Ver tabela \ref{tab:cortes_em}} 
\newacronym[type=Simb]{eratio}{\ensuremath{E_{ratio}}}{Variável de corte dos
algoritmos \acrshort{eg} padrões. Ver tabela \ref{tab:cortes_em}} 
\newacronym[type=Simb]{weta2}{\ensuremath{w_{\eta2}}}{Variável de corte dos
algoritmos \acrshort{eg} padrões. Ver tabela \ref{tab:cortes_em}} 
\newacronym[type=Simb]{weta}{\ensuremath{w_{\eta}}}{Variável de corte dos
algoritmos \acrshort{eg} padrões. Ver tabela \ref{tab:cortes_em}} 
\newacronym[type=Simb]{deta1}{\ensuremath{\Delta\eta_1}}{Variável de corte dos
algoritmos \acrshort{eg} padrões. Ver tabela \ref{tab:cortes_em}} 
\newacronym[type=Simb]{dphi2}{\ensuremath{\Delta\phi_2}}{Variável de corte dos
algoritmos \acrshort{eg} padrões. Ver tabela \ref{tab:cortes_em}} 
\newacronym[type=Simb]{Ep}{\ensuremath{\frac{E}{p}}}{Variável de corte dos
algoritmos \acrshort{eg} padrões. Ver tabela \ref{tab:cortes_em}} 

\glsunset{Rhad1}
\glsunset{Rhad}
\glsunset{eratio}
\glsunset{reta}
\glsunset{weta}
\glsunset{weta2}
\glsunset{deta1}
\glsunset{dphi2}
\glsunset{Ep}

Os três conjuntos de referências citados em~\ref{ssec:alg_topo} são utilizados para 
seleção de partículas \gls{eg}. Eles são definidos ao se elevar a potência de rejeição 
de ruído físico nos dados finais: \emph{Loose}, com menor seletividade, 
evitando a perda de dados prematura, mas ao mesmo tempo elevando a taxa de dados 
a serem gravados; \emph{Tight}, com grande seletividade, reduzindo o
ruído físico e a taxa de dados, mas podendo haver perda de eventos de interesse;
\emph{Medium}, que tenta conciliar a escolha da seletividade de forma
a reduzir a taxa de armazenamento ao mesmo tempo que evita a perda de eventos
de forma prematura. As variáveis utilizadas no \gls{ef}, dispostas na
Tabela~\ref{tab:cortes_em}, são as mesmas que as
utilizadas na versão a posteriori, mas com limiares tipicamente mais relaxados
que no último para se evitar a perda de eventos interessantes, enquanto o
\gls{l2} utiliza apenas algumas dessas variáveis, novamente com valores mais relaxados que no 
\gls{ef} pelo mesmo motivo. No caso do \gls{t2calo} essas variáveis são: \gls{reta}, 
\gls{eratio}, \gls{Et} e \gls{Rhad1}, com algumas condições específicas 
dependendo da região em que a partícula está incidindo, em especial nesse 
caso para a região de fenda do calorímetro, e para partículas de altas energias onde se aceita um maior
vazamento hadrônico. Os valores dos limiares são escolhidos através de estudos
de \gls{mc}, e, ainda que os mesmos variem conforme \gls{eta} e a energia da
partícula, eles são compostos por diversos cortes lineares nessas variáveis
físicas. Vale ressaltar que as variáveis utilizando o \gls{id} 
não se aplicam para fótons, independente do algoritmo em questão.


\begin{table}
\centering
\resizebox{\textwidth}{!}{
\begin{tabular}{p{4cm}p{9cm}c}
\hline
\hline
\hline
\textbf{Tipo} & \textbf{Descrição} & \textbf{Símbolo (se aplicável)} \\
\hline
\hline
 & \centering Cortes \emph{Loose} & \\
\hline
\hline
Vazamento Hadrônico & Razão de \gls{Et} da primeira camada \gls{had} com a
depositada no calorímetro \gls{em} (usado para $|\gls{eta}|<0,8$ e
$|\gls{eta}|>1,37$). & \gls{Rhad1} \\
 & Razão de \gls{Et} da energia depositada no calorímetro \gls{had} com a
depositada no calorímetro \gls{em} (usado para $|\gls{eta}|>0,8$ e
$|\gls{eta}|<1,37$). & \gls{Rhad} \\
\hline
Segunda camada do calorímetro \gls{em} & Razão em \gls{eta} entre a enegia contida nas
células numa região de $3\times7$ por uma região $7\times7$. & \acrshort{reta}\\
 & Largura lateral do chuveiro. & \gls{weta2} \\
\hline
\hline
 & \centering Cortes \emph{Medium} (Incluem o \emph{Loose}) & \\
\hline
\hline
Primeira camada do calorímetro \gls{em} & Largura lateral total do chuveiro. & \gls{weta} \\
 & Diferença entre o primeiro e o segundo depósito de maiores
energias normalizadas por sua soma. & \gls{eratio} \\
\hline
Qualidade do Traço & Número de pontos no Detector de Píxel ($\ge1$). & \\
 & Número de pontos no Detector de Píxel e \gls{sct} ($\ge7$). & \\
 & Parâmetro de impacto transverso ($<5$ mm). & \gls{d0} \\
\hline
Casamento de Traço & $\Delta\eta$ entre o aglomerado de células e traço (<0,01)
& \\
\hline
\hline
 & \centering Cortes \emph{Tight} (Incluem o \emph{Medium}) & \\
\hline
\hline
1 camada do Detector de Pixel (camada B) & Número de pontos na camada B
($\ge1$).  & \\
\hline
Casamento de Traço & $\Delta\phi$ entre o aglomerado de células e traço
(<~0,02). & \gls{dphi2} \\
 & Razão entre a energia do aglomerado com o momento medido pelo \gls{id}. & \gls{Ep} \\
 & Corte mais restritivo em $\Delta\eta$ (<~0,005). & \gls{deta1} \\
\hline
Qualidade do Traço & Parâmetro de impacto transvero mais restritivo (<~1 mm).  & \gls{d0} \\
\hline
\gls{trt} & Número de pontos no \gls{trt}. & \\
 & Razão entre o número de pontos de alta precisão com o número de pontos no
\gls{trt}. & \\
\hline
Conversões & Candidatos a elétron provenientes de fótons convertidos são
rejeitados. & \\
\hline
\hline
\end{tabular}
}
\caption[Definições dos cortes utilizados para os critérios de identificação de elétrons \emph{loose}, \emph{medium}
e \emph{tight}, na região de $|\eta|<2,47$]
{Definições dos cortes utilizados para os critérios de identificação de elétrons \emph{loose}, \emph{medium}
e \emph{tight}, na região de $|\gls{eta}|<2,47$. Adaptado de \cite{expected_perf_2011}.}
\label{tab:cortes_em}
\end{table}

\subsection{HLT\_EgammaCaloRinger (Ringer\_HLT)}
\label{ssec:hlt_ringer}

\newacronym[type=Abrev]{rna}{RNA}{Rede Neural Artificial}
\newacronym[type=Abrev]{ica}{ICA}{Análise de Componentes Independentes}
\newacronym[type=Abrev]{nlica}{NLICA}{Análise de Componentes Independentes
Não-Linear}
\newacronym[type=Abrev]{pca}{PCA}{Análise de Componentes Principais}
\newacronym[type=Abrev]{pcd}{PCD}{Componentes Principáis de Discriminação}
\newacronym[type=Abrev]{som}{SOM}{Mapas Auto-Organizáveis}

Uma outra proposta para se identificar a evolução do chuveiro no calorímetro é 
abordada pelo algoritmo alternativo proposto, afim de se identificar as partículas \gls{em}. 
Inicialmente esse algoritmo, chamado de \gls{hltringer}, foi planejado para operar no
\acrlong{l2} como uma alternativa ao \gls{t2calo}, o ambiente mais restritivo 
em questões de tempo. Ao invés de gerar diversas variáveis de interpretação física 
para a compreensão da interação da partícula com o calorímetro, como os já
citados: \gls{eratio}, \gls{reta} e \gls{Rhad1}; o algoritmo utiliza a informação anelada
de calorimetria como o seu \gls{fex} (Tópico~\ref{sssec:anelamento}) 
e um processo de discriminação para o seu \gls{hypo}. Qualquer método estatístico 
pode ser aplicado para realizar o processo de discriminação. 
Em partícular, os estudos deste trabalho utilizaram \gls{rna}
(Tópico~\ref{sssec:rna}), pois resultados anteriores indicam melhor perfomance.

Diversas técnicas de pré-processamento podem ser combinadas com redes neurais de
forma a melhorar a sua eficiência de discriminação. Estudos anteriores
\cite{tese_eduardo,tese_torres} utilizaram métodos como \gls{ica}, \gls{pca},
\gls{nlica}, \gls{som} e \gls{pcd}, obtendo resultados ainda melhores que uma
abordagem utilizando diretamente \gls{rna}. Ainda, um caso especial de pré-processamento 
de dados, necessaria quando utilizando redes neurais, é a normalização dos dados. Esse 
pré-processamento ajusta o alcânce de energia dos anéis ao alcânce dinâmico das
\gls{rna}. Uma grande gama de métodos de normalização são utilizados, podendo
gerar alterações no espaço de representação de anéis, de forma que a
interpretação da nova representação pode ter uma melhor, ou pior, interpretação
pela \gls{rna}. As normalizações testadas para otimizar a eficiência do
\gls{hltringer} estão explicadas no Tópico~\ref{sssec:preproc_norm}.


\subsubsection{O processo de anelamento}
\label{sssec:anelamento}

O processo de anelamento, esboçado na Figura~\ref{fig:cons_aneis}, é realizado
para todas as camadas do Sistema de Calorimetria do \gls{atlas}. Como foi dito,
no \gls{l2} a semente utilizada para a construção do aglomerado é a célula mais quente  -- ou a célula mais
energética -- da segunda camada \gls{em} na \gls{roi} estudada gerada a partir da posição 
fornecida pelo \gls{l1}. No \gls{hltringer}, ao invés da construção do
aglomerado, essa célula será o centro dos anéis e sua energia será a informação
contida no anel central. Os anéis posteriores contêm a soma da energia das células adjacentes
exteriores ao anel anterior, por exemplo, no caso do segundo anel, as células imediatamente
exteriores a célula quente. Esse processo irá se repetir até que uma região pré-determinada seja
completamente preenchida, no caso o valor atual utilizado é de $0,4\times0,4$ em
$\Delta\eta\times\Delta\phi$ centrados na semente. Nas outras camadas a posição da
semente é utilizada para encontrar a célula central e o processo é repetido, até
que a mesma região seja preenchida, entretanto, como a granularidade das células
do calorímetro variam conforme a segmentação longitudinal, o número anéis variam
conforme a camada em questão. Como o \gls{l2} calcula
um novo centro, anéis podem necessitar de células exteriores a \gls{roi}, essas não
estando disponíveis para o \gls{l2}, de forma que nem todos os anéis seram
completamente fechados, sendo assim o conceito de anéis abstrato. No caso de
nenhuma célula estiver disponível para um dado anel, é então atribuido valor
nulo ao mesmo para garantir que o processo complete o número necessário para o
preenchimento da região. Um total de 100 anéis foram especificados inicialmente,
divididos conforme as camadas da maneira indicada na Figura~\ref{fig:cons_aneis}, 
onde a quantidade de anéis é proporcional a granularidade de cada camada.


\begin{figure}[p]
\label{fig:cons_aneis}
\centering
\includegraphics[height=0.9\textheight]{imagens/cons_aneis.pdf}
\caption[Diagrama do processo de construção dos anéis.]
{Diagrama do processo de construção dos anéis. Extraído de \cite{tese_eduardo}.}
\end{figure}


\begin{figure}[ht!]
\label{fig:perfil_aneis}
\centering
\includegraphics[width=0.9\textwidth]{imagens/segunda_camada_celulas.pdf}
\caption[Perfil dos anéis na segunda camada para elétrons e jatos.]{Perfil dos
anéis na segunda camada para elétrons e jatos. Extraído de \cite{tese_eduardo}.}
\end{figure}

Esse processo reduz a quantidade de informação a ser analisada no calorímetro ao
agrupar diversas células em uma única informação, mas, ao mesmo tempo, mantêm a interpretação 
física da propagação do chuveiro, como a sua espessura
lateral e profundidade longitudinal. Ainda, o número total de anéis a serem
propagados para o método de discriminação pode ser reduzido através de estudo
com quantidade elevada de estatística do processo. Entretanto, deve-se enfatizar
que a redução de informação é utilizada para se referir a compactação da
informação das células do calorímetro nos anéis, não podendo ser utilizado quando se comparando 
com a ordem de informação armazenada nas variáveis físicas, que se resumem a cerca de 5 variáveis.

\subsubsection{Normalização}
\label{sssec:preproc_norm}

Neste tópico estão as descrições das normalizações testadas para a otimização do
\gls{hltringer}. Foram utilizadas normalizações presentes na literatura
utilizando tratamento estatístico dos dados, como a Esferização, MinMax, 
adicionadas de normalizações que fazem mão do conhecimento da topologia dos calorímetros, 
sua segmentação longitudinal e das diferenças entre os chuveiros \gls{em} e \gls{had}, 
buscando realçar suas características, citando entre elas a normalização
sequencial.

\paragraph{Energia Total (Norma 1)}
\label{par:norm_norm1}

Neste modo de normalização, também chamado de Norma~1, cada um dos anéis ($r$) produzidos é normalizado pela energia total 
(considerando-se todas as camadas) contida em uma região de $0,4 \times 0,4$ em 
$\eta \times \phi$ do evento fazendo-se

\begin{equation}
r_{i}' = \frac{r_i}{\overset{N}{\underset{j=1}{\sum}} r_j}~~\forall~~i=1,2,3,...,N
\end{equation}

\noindent onde $N$ é o número de anéis produzidos, considerando todas as camadas (i.e. 100). 
O objetivo desta normalização é reduzir a influência da energia de cada evento nas análises, 
mantendo, ainda assim, a proporção de energia contida em cada anel.


\paragraph{Energia da Camada}
\label{par:norm_camada}

Neste modo de normalização, os anéis ($r_c$) produzidos na c-ésima camada (PS, EM1, HD1, etc.) 
são normalizados pela energia contida na camada, em uma região de $0,4 \times 0,4$ em $\eta 
\times \phi$, fazendo-se

\begin{equation}
r_{c_{i}}' = \frac{r_{c_{i}}}{\overset{N_c}{\underset{j=1}{\sum}} r_{c_j}}~~\forall~~i=1,2,3,...,N_c
\end{equation}

\noindent onde $N_c$ é o número de anéis produzidos na c-ésima camada. Neste modo de normalização, 
objetiva-se equiparar, do ponto de vista energético, a informação contida em cada camada.


\paragraph{Energia da Seção}
\label{par:norm_secao}

Neste modo de normalização, os anéis ($r_s$) produzidos na s-ésima seção (eletromagnética 
ou hadrônica) são normalizados pela energia contida na seção, em uma região de $0,4 \times 0,4$ 
em $\eta \times \phi$, fazendo-se

\begin{equation}
r_{s_{i}}' = \frac{r_{s_{i}}}{\underset{j=1}{\overset{N_s}{ \sum}} r_{s_{j}}}~~\forall~~i=1,2,3,...,N_s
\end{equation}

\noindent onde $N_s$ é o número de anéis produzidos na s-ésima seção. Neste modo de normalização, 
objetiva-se equiparar, do ponto de vista energético, a informação contida em cada seção.


\paragraph{Sequencial}
\label{par:norm_seq}

Esta técnica de normalização visa amplificar as diferenças no perfil lateral dos chuveiros produzidos. 
Nesta técnica, os anéis ($r_c$) produzidos na c-ésima camada são normalizados fazendo-se

\begin{equation}
\label{eq:normalizacao_sequencial}
r_{c_{i}}' = \frac{r_{c_{i}}}{ E_{tot_{c}} - \underset{j=1}{\overset{i-1}{\sum}} r_{c_{j}} }
\end{equation}

\noindent onde $E_{tot_{c}}$ é a energia total da c-ésima camada. Esta normalização pode ser interpretada 
como uma otimização da normalização por energia da camada, amplificando a contribuição dos anéis mais 
externos ao centro da RoI, através da aplicação de fatores de normalização sucessivamente menores. Para 
evitar a amplificação de anéis contendo apenas ruído, quando o fator de normalização fica menor do que 
um dado limiar  ($E_{stop} = 100$~MeV), todos os anéis restantes passam a ser normalizados pela energia 
total da camada ($E_{tot_{c}}$). Adicionalmente, para evitar que camadas contendo apenas ruído sejam 
normalizadas, caso $E_{tot_{c}}$ fique abaixo de um dado limiar ($E_{thres} = 0,01$~MeV), nenhuma 
normalização é aplicada para a camada em questão.


\paragraph{Norma 2}
\label{par:norm2}

Cada anel foi normalizado por

\begin{equation}
r_{i}' = \frac{r_i}{||\mathbf{r}||}~~\forall~~i=1,2,3,...,N
\end{equation}

\noindent onde $||\mathbf{r}||$ é a norma 2 dos N anéis produzidos.


%\paragraph{Energia Transversa}
%\label{par:norm2}
%
%Cada anel foi normalizado por
%
%\begin{equation}
%r_{i}' = \frac{r_i}{E_T}~~\forall~~i=1,2,3,...,N
%\end{equation}
%
%\noindent onde $E_T$ é a energia transversa da RoI.


\paragraph{Esferização dos Anéis}

Cada anel foi normalizado por

\begin{equation}
r_{i}' = \frac{r_i - \bar{r_i}}{\sigma_{r_i }}~~\forall~~i=1,2,3,...,N
\end{equation}

\noindent onde $\bar{r_i}$ e $\sigma_{r_i }$ são, respectivamente, a média e o desvio 
padrão do i-ésimo anel. 


\subsubsection{Discriminação com Redes Neurais Artificiais}
\label{sssec:rna}

Após a extração dos anéis pode-se utilizar uma abordagem não-segmentada, na qual
se concatena os anéis e o propaga então para uma única \gls{rna} discriminadora,
ou então uma abordagem segmentada, onde cada camada longitudinal do Sistema de
Calorimetria é propagada para uma \gls{rna} 

\begin{figure}[ht!]
\label{fig:tipo_class}
\centering
\subfigure[]{
    \label{fig:class_nseg}
    \includegraphics[width=0.5\textwidth]{imagens/classificacao_nao_segmentada.pdf}
}\\
\subfigure[]{%
    \label{fig:class_seg}
    \includegraphics[width=0.7\textwidth]{imagens/classificacao_segmentada.pdf}
}
\caption[Abordagens de Discriminação.]{Abordagem de discriminação não-segmentada
em~\ref{fig:class_nseg} e segmentada em~\ref{fig:class_nseg}. Extraído de 
\cite{tese_eduardo}.}
\end{figure}

\section{O Sistema de Reconstrução (SR)}
\label{sec:sr}

% Falar aqui que o Sistema de Filtragem é uma versão mais simples do Sistema de
% Reconstrução

\subsection{\texorpdfstring{Algoritmo $e/\gamma$ Padrão}{Algoritmo eGamma Padrão}}
\label{ssec:egamma}


% Falar que tem mais opções que 
% Colocar os requerimentos, tight, loose, medium.


\subsection{\texorpdfstring{$e/\gamma$ \emph{Calorimeter Ringer}
(EgCaloRinger)}{eGamma Calorimeter Ringer (EgCaloRinger)}}
\label{ssec:egringer}

% Falar aqui da request feito pela colaboracao

\subsubsection{Implementação}
\label{sssec:egringer_impl}
% Implementação



