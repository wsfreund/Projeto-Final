\chapter{Publicações}

\begin{itemize}

%-------------------------------------------------------------------------------
\item Tile DCS Web System
MAIDANTCHIK, C. L. L., FEREIRA, F. G., GRAEL, Felipe F.
Computing in High Energy Physics, 2009, Prague.

The web system described here provides functionalities to monitor the Detector
Control System (DCS) acquired data. The DCS is responsible for overseeing the
coherent and safe operation of the ATLAS experiment hardware. In the context of
the Hadronic Tile Calorimeter Detector, it controls the power supplies of the
readout electronics acquiring voltages, currents, temperatures and coolant
pressure measurements. The physics data taking requires the stable operation of
the power sources.

The DCS Web System retrieves data automatically and processes it extracting the
statistics for given periods of time. The mean and standard deviation outcomes
are stored as XML files and are compared to preset thresholds. Further, a
graphical representation of the TileCal barrels indicates the state of the
supply system of each detector drawer. Colors are designated for each kind of
state. This way problems are easier to find and the collaboration members can
focus on them. The user can pick a module to see detailed information. It is
possible to check the statistics and generate charts of the parameters over the
time.

The DCS Web System also gives information about the power supplies latest
status. The barrel colors green whenever the system is on. Otherwise it is
colored red. Furthermore, it is possible to perform customized analyses. It
provides search interfaces where the user can set the module, parameters, and
the time period of interest. Moreover the system produces the output of the
retrieved data as charts, XML files, CSV and ROOT files according to the user's
choice.

%-------------------------------------------------------------------------------
\item Glance project: a database retrieval mechanism for the ATLAS detector
In: Computing in High Energy Physics, 2007, Victoria.
Maidantchik, C, Grael, F F, GALVÃO, K K, Pommès, K
 Journal of Physics: Conference Series. , 2008. v.119.

During the construction and commissioning phases of the ATLAS Collaboration, data
related to the installation, testing and performance of the equipment are stored in
relational databases. Each group acquires information and saves them in repositories
placed in different servers, using diverse technologies. Both data modeling and
terminology may vary among the storage areas. The development of retrieval systems
for each data set would require too much effort and high maintenance cost.
The goal of the Glance Project is to provide navigation mechanisms among the
databases, which is independent of both the technology used to build them and their
relationship. The browsing over the data sets results in hypertext tables and links
on the Web. The user chooses the database and the system shows its structure. After
selecting a partition of the repository, the system automatically creates a retrieval
interface that allows the specification of search parameters. The interface can be
customized for specific needs and saved in order to be easily accessed later. Glance
corresponds to a single system that handles distinctive recovery mechanisms among
diverse databases. Therefore, further knowledge about how data is organized and
labeled is not required to perform queries. Maintenance costs are also minimized.
This paper describes the Glance conception, its development and functionalities. The
system usage is illustrated with some examples. Current status and future work are
also discussed.

%-------------------------------------------------------------------------------
\item Web system to support analysis of the Tile Calorimeter commissioning
In: Computing in High Energy Physics, 2007, Victoria.
Maidantchik, C, Faria, A, Grael, F F, Ferreira, F G, GALVÃO, K K,
Dotti, A, Solans, C, Price, L
 Journal of Physics: Conference Series. , 2008. v.119.

During the ATLAS detector commissioning phase, installed readout electronics
must pass performance standards tests. The resulting data must be analyzed to
ensure correct operation. For the Tile Calorimeter, developers plug their code
into a specific framework for physics data-processing,. Collaboration members,
taking shifts on commissioning work, interpret the results, in thousands of
readout channels, to identify potential problems that may need correction
during commissioning. The Tile Commissioning Web System (TCWS) facilitates the
repetitive data analysis and quality control by encapsulating all necessary
steps to retrieve information, execute programs, access the outcomes, register
statements, and verify the equipment status. TCWS integrates different
applications, each presenting a particular view of the commissioning process.
The TileComm Analysis application stores plots and analysis results, provides
equipment-oriented visualization, collects information regarding equipment
performance, and summarizes its status. The Timeline application provides
equipment status history in a chronological way. The Web Interface for Shifters
application supports monitoring tasks by managing test parameters, graphical
views of the calorimeter performance, and information status of all equipment
that was used in each test. Finally, equipment quality control data can be
filled, stored, modified, and retrieved as hypertext forms through the
ATLASMonitor application. These applications are also connected with other
commissioning programs that allow an automatic gathering of the commissioning
data. This paper describes in detail the programs that compose the TCWS and how
they are integrated within the Tile Calorimeter commissioning. Current status
and future work are also discussed

%-------------------------------------------------------------------------------
\item Management of Equipment Databases at CERN for the ATLAS Experiment
In: 10th ICATPP Conference on Astroparticle, Particle, Space Physics,
Detectors and Medical Physics Applications, 2007, Villa Olmo.
POMMES, K., GALVÃO, K K, MOLINA-PEREZ, J., Grael, F F, MAIDANTCHIK, C. L. L.
 10th ICATPP Conference on Astroparticle, Particle, Space Physics,
Detectors and Medical Physics Applications. , 2007.

The ATLAS experiment apparatus is divided into major components, composed by
numerous single equipment devices and a back end electronics. A complex cabling
scheme connects the machinery. The detector has large dimensions and a complex
geometry with components in several different shapes. Displacements and
deformation are important in data analysis and calculus. Furthermore, nuclear
and radiation control rules state the traceability of all the apparatus
components. Graphical visualization systems were deployed to address and
identify the several parts and manage displacements and deformation. Equipment
Databases were designed to support the trace of objects composing the
structure. User interfaces designed for potable devices facilitate data
updates. A universal mechanism allows the management of the distinct
information stored, providing means of data access and update by users and
other applications. Various aspects of equipment installation and maintenance
are addressed and supported by distinct software components, with heterogeneous
information integrated by one universal tool.

%-------------------------------------------------------------------------------
\item Sistema Glance para recuperação e operação de dados de repositórios
heterogêneos para o ATLAS
MAIDANTCHIK, C. L. L., GRAEL, Felipe F., GALVÃO, K K, EVORA, L. H. R. A.
XXIX Encontro Nacional de Física de Partículas e Campos, 2008.

O detector ATLAS é construído por uma colaboração internacional heterogênea.
Durante as etapas de construção, teste e integração dos componentes, dados
sobre a construção e desempenho dos equipamentos foram gerados. No entanto, a
recuperação desses dados envolve conhecer detalhes técnicos das tecnologias
utilizadas para o armazenamento, a organização dos repositórios para que um
método de recuperação seja programado.

Frente às dificuldades de recuperação de informações, foi desenvolvido na UFRJ,
em colaboração com o ATLAS, o Glance, um sistema universal capaz de reconhecer
a estrutura dos repositórios, auxiliar o usuário na criação de uma interface de
recuperação e realizar buscas sobre os dados. As interfaces criadas são
descritas em uma linguagem intermediária independente da tecnologia do
repositório. Dessa forma um único sistema gerencia diversas interfaces
independentes.

Ao longo da fase de comissionamento surgiu o requisito de processar os dados
recuperados para serem melhor interpretados pelos colaboradores. Para suportar
essa nova necessidade, foi proposto e implementado um mecanismo capaz de
realizar operações tais como calcular médias ou aplicar fórmulas envolvendo
atributos de uma pesquisa.

Com essa funcionalidade, o sistema reconhece o tipo dos dado que serão
operados, que podem ser numéricos ou textuais, e apresenta operações adequadas.
Ao executar a busca, o Glance passa para um módulo intermediário os resultados,
descritos em um formato intermediário, e a operação definida pelo usuário. Pela
existência do módulo intermediário, a operação é independente do tipo de banco
de dados. Os resultados são apresentados como uma nova coluna ou como um
sumário contendo, por exemplo, a média de uma coluna. Além disso, os dados
podem ser retornado na forma de uma tabela hipertextual, como gráficos ou até
em outros formatos como CSV ou XML.

Esse esquema está sendo aplicado ao Detector Control System (DCS) do
calorímetro de telhas. O DCS monitora e registra em um banco de dados tensões,
correntes e temperaturas em determinados pontos da eletrônica do sub-detector.
Através da nova funcionalidade, o Glance calcula a diferença de voltagem entre
dois pontos e também a média e o RMS das medições sobre um determinado período
de tempo. Outra aplicação semelhante é junto ao grupo que mede o posicionamento
dos equipamentos no ambiente experimental e monitora desvios entre a posição
teórica e a observada em determinados pontos do detector.

Este trabalho apresenta a funcionalidade de operações sobre dados do Glance e
demonstra sua utilização. Também são discutidas em mais detalhes as aplicações
existentes desse mecanismo.

%-------------------------------------------------------------------------------
\item O sistema Glance para recuperaçao e inserçao de dados em repositórios
heterogêneos
MAIDANTCHIK, C. L. L., GRAEL, Felipe F., GALVÃO, K K
XXVIII Encontro Nacional de Física de Partículas e Campos, 2007.

Os componentes do detector de partículas ATLAS são construídos por uma
colaboração internacional geografi- camente dispersa. A montagem de suas
dezenas de milhares de componentes está sendo finalizada no CERN, onde entrará
em operação. Uma vez integrados, os equipamentos são testados em conjunto. As
informações de localização e conectividade de cada parte, bem como os
resultados dos testes são armazenados em diferentes repositórios, utilizando
diversas tecnologias, modelagens e terminologias.

Para solucionar os problemas de recuperação de dados heterogêneos, inerentes a
esse ambiente, foi desenvolvido um sistema universal capaz de realizar buscas
sobre os repositórios independentemente de suas tecnologias.  Ao serem
fornecidas as informações de como se conectar em um repositório, o Glance
reconhece sua estrutura interna, e permite a seleção do conjunto de dados de
interesse aumentando sucessivamente o nível de detalhes apresentados. O sistema
então reconhece os atributos e seus tipos e gera automaticamente a descrição de
uma interface de recuperação correspondente. A descrição é feita em uma
linguagem de marcação baseada em XML e contém o nome dos atributos, as
informações técnicas necessárias para localizá-los, seus tipos, descrições e
possíveis valores. Baseado nessa descrição, o sistema gera uma interface de
busca paramétrica, onde devem ser estabelecidos os atributos sobre o qual se
quer realizar a pesquisa, os operadores e os valores de referência. Os
operadores são sensíveis ao tipo do atributo, isto é, um atributo numérico
possui os operadores ``maior que'' e ``menor que'', enquanto um textual possui
``contém''. Os resultados da busca são apresentados na forma de uma tabela
hipertextual, ou como arquivos em formatos tais como CSV ou XML.

Ao instalar ou remover um componente do detector, a equipe de coordenação
técnica precisa atualizar as informações de conectividade e localização nos
bancos de dados. Para dar suporte a essa tarefa, o sistema permite também a
criação de interfaces para inserção e modificação de dados. Antes de realizar
qualquer alteração no repositório, o Glance verifica a integridade dos novos
dados, evitando inconsistências.  O sistema também é usado pelo grupo do
calorímetro hadrônico, para recuperar dados do DCS (Detector Control System)
que monitora tensões, correntes e temperaturas. Para isso, o Glance realiza
operações pre-definidas sobre os resultados da busca, de acordo com o tipo de
análise a ser realizada.

Este trabalho apresenta a concepção do Glance e demonstra sua utilização.
Também é mostrado em mais detalhes como o sistema é utilizado pela coordenação
técnica e pelo calorímetro hadrônico.

%-------------------------------------------------------------------------------
\item Sistema de Análise e Monitoração dos dados não-físicos do Calorímetro
Hadrônico do detector ATLAS
MAIDANTCHIK, C. L. L., FEREIRA, F. G., GRAEL, Felipe F.
XXVIII Encontro Nacional de Física de Partículas e Campos, 2007.

O calorímetro hadrônico  do detector ATLAS encontra-se em fase de
comissionamento. Durante este periodo são testados os equipamentos e sistemas
que o compõe a fim de garantir o seu futuro funcionamento. O Laboratório de
Processamento de Sinais (LPS-UFRJ) é um dos laboratórios participantes desta
colaboração internacional.

O DCS (\emph{Detector Control System}), responsável pela monitoração das fontes
de alta e baixa tensões que alimentam os componentes eletrônicos integrantes do
sub-detector e pelo sistema de refrigeração que garante o funcionamento dessas
fontes, é um desses sistemas. Para tal, adquire dados como tensões, correntes e
temperaturas utilizando um programa  desenvolvido dentro do framework do
software proprietário PVSS, que é uma plataforma destinada especialmente para
sistemas de controle.  Após a aquisição tais informações são armazenadas numa
base de dados relacional de tecnologia Oracle. Estes dados não são usados para
análises físicas, mas são importantes para garantir a consistência delas.

Para realizar o monitoramento e análise desses dados foi proposto e
implementado um sistema capaz de disponibilizar as informações via internet que
possam garantir que as fontes estejam funcionando corretamente, e a este foi
dado o nome de \emph{DCS Web System}. Primeiramente os dados são recuperados da
base de dados utilizando-se o sistema Glance, também desenvolvido pelo chamado
grupo do Rio.  Após a recuperação dos dados, o sistema \emph{Glance} formata-os
em arquivos XML, garantindo assim flexibilidade, e através desses gera médias e
os desvio padrão das amostragens, dados suficientes para estudar a estabilidade
dos equipamentos, para um dia ou um mês. Ainda são disponibilizados gráficos
temporais que mostram o progresso de determinada fonte. Ainda são previstos no
projeto distribuições estatísticas dos dados em estudo.

Atualmente, o \emph{DCS Web System} oferece o suporte à análise e ao
monitoramento das fontes de baixa tensão e ao sistema de refrigeração, sendo o
suporte às fontes de alta tensão uma atualização prevista e pelo sistema de
refrigeração que garante o funcionamento dessas fontes, é um desses sistemas.
Para tal, adquire dados como tensões, correntes e temperaturas utilizando um
programa  desenvolvido dentro do framework do software proprietário PVSS, que é
uma plataforma destinada especialmente para sistemas de controle.  Após a
aquisição tais informações são armazenadas numa base de dados relacional de
tecnologia \emph{Oracle}. Estes dados não são usados para análises físicas, mas
são importantes para garantir a consistência delas.

%-------------------------------------------------------------------------------
\item Glance: a retrieval system for construction, installation and
commissioning data for ATLAS equipments
MAIDANTCHIK, C. L. L., Grael, F F, SEIXAS, J. M.
XXVII Encontro Nacional de Física de Partículas e Campos, 2006.

The ATLAS detector is being constructed by an international collaboration for
the LHC experiment at CERN.  For each part a big amount of data related to its
construction, installation and commissioning is being generated.  Since each
part is being constructed by different institute, the data are being stored at
different places and modeled according to the specific needs of each part and
institute.

During the current phase of installation and commissioning each working group
needs to have access to the data related to their equipments, which can involve
parts coming from different institutes. Different groups may be interested in
different perspectives of the data. For example, the commissioning team of a
sub-detector is interested in the installation and production data of their
electronics and cables. Other group, like the cabling team, is interested in
the data about installation and connectivity for all subsystems, but not about
the construction.

In this context of distributed working groups it is very difficult to provide a
single interface that delivers the information suitable for every group. In the
other hand, requiring each group to develop their own solution would need a
great effort. This would also represent a high maintenance cost, since small
changes to the repository of an institute can reflect into adjustments of
multiple interfaces.

The goal of the project is to facilitate the retrieval of the data of
construction, installation and commissioning for the equipments in an
integrated way. The user selects the set of data of his interest. The system
then recognizes the characteristics of the data and automatically generates a
corresponding retrieval interface. This interface then can be stored in order
to be accessed by the collaborators from the same group. Through the retrieval
interface, the user can perform searches and filter the results successively to
reach the desired data.  The system stores the description of each interface,
with the information needed to perform the retrieval, in a way that they can be
added or modified as needed, without requiring changes to the main program.

The system is being actively used by the Technical Coordination group to
retrieve data about cabling, positioning and physical installation of
electronic crates and boards, and integrate these data with an inventory
database, and with the production database of some of the sub-detectors. It is
running on the CERN web servers, and is developed using C++ and XML
technologies.


%-------------------------------------------------------------------------------
\item Sistema WEB para Testes de Equipamentos em Física de Altas Energias
MAIDANTCHIK, C. L. L., SEIXAS, J. M., ALVES, A. M., FARIA, A., GRAEL,
Felipe F., FEREIRA, F. G., GALVÃO, K K
XXVII Encontro Nacional de Física de Partículas e Campos, 2006.

O detector ATLAS, acoplado ao acelerador de partículas LHC do CERN, encontra-se
atualmente em fase de comissionamento. Os testes realizados geram uma enorme
quantidade de dados, que são posteriormente analisados pelos colaboradores em
diferentes países. A cada execução dos programas de análise, uma série de
procedimentos e configurações deve ser realizada pelos pesquisadores. Os
gráficos e histogramas resultantes se referem aos níveis de energia durante uma
colisão de partículas, e dados para o controle de qualidade dos equipamentos
também são gerados a partir das análises.

Este projeto apresenta o sistema \emph{Tile Commissioning Web System} que apóia
a manipulação e análise dos dados provenientes dos testes realizados no
Calorímetro de Telhas (TileCal), um dos sub-detectores do ATLAS, e apresenta
ferramentas para a recuperação dos resultados obtidos. O sistema é composto por
três softwares com interface Web que possuem funções específicas: o \emph{Web
Interface for Offline Shifters}, o \emph{Tilecomm Analysis} e o
\emph{AtlasMonitor}.

O \emph{Web Interface for Offline Shifters} (WIS) automatiza o processo de
análise, apresentando ao usuário uma tabela com todos os testes realizados e os
tipos de análises que podem ser realizadas para cada um. Após a seleção do
usuário, o WIS recupera o arquivo correspondente no sistema de armazenamento
CASTOR (\emph{Cern Advanced STORage}) e executa remotamente o programa de
análise requerido, seguindo todos os procedimentos e configurações exigidos. Ao
final do processo de análise, os resultados são disponibilizados na interface
do sistema, apresentando gráficos de níveis de energia e dados de controle de
qualidade. Os gráficos e os dados são automaticamente armazenados nas
respectivas bases de dados dos sistemas \emph{Tilecomm Analysis} e \emph{Atlas
Monitor}. O primeiro software recupera os gráficos, através da associação com o
tipo e o identificador do teste e com a seção do sub-detector que foi testada.
O \emph{AtlasMonitor} insere automaticamente os dados resultantes nas folhas de
controle de qualidade. Posteriormente, o colaborador pode inserir informações
adicionais ou até mesmo criar uma nova folha de controle de qualidade.

O sistema \emph{Tile Commissioning Web Syste}m está instalado no servidor do
CERN e é utilizado pelos colaboradores do TileCal e seu desenvolvimento conta
com a participação dos responsáveis tanto pelos testes dos equipamentos quanto
pelo funcionamento do calorímetro.


\end{itemize}
