\chapter{Publicações}

\begin{itemize}

\item  FREUND, W. S.; DAMAZIO, D.; SEIXAS, J. M. . Ringer algorithm for offline
identification. 2011.

\item  FREUND, W. S.; DAMAZIO, D.. Update on L2 Calibration. 2011. 

\item  FREUND, W. S.; DAMAZIO, D.. Update on L2 calibration/clustering. 2011.

\item  FREUND, W. S.; DAMAZIO, D.; SEIXAS, J. M. . Egamma Ringer - Algoritmo para
discriminação elétrons/fótons para o detector ATLAS. 2011. 

\item   FREUND, W. S.; TORRES, R. C.; DAMAZIO, D.; SIMAS, E.; SEIXAS, J. M.;
DEVA, D.. Ringer Algorithm. 2010. (Apresentação de Trabalho/Outra).

\item FREUND, W. S.; DAMAZIO, D.; SEIXAS, J. M.. Física Experimental de Altas
Energias e Tecnologias Assossiadas. 2010.

\item FREUND, W. S.; DAMAZIO, D.. L2 Calorimeter Calibration. 2010.


\item FREUND, W. S.; DAMAZIO, D.; LIMA, D. E. F.; SEIXAS, J. M.. Calibração do
Sistema de Filtragem Online do ATLAS. 2010. 

\item FREUND, W. S.; SEIXAS, J. M.; TORRES, R. C.. Física Experimental de Altas
Energias e Tecnologias Assossiadas. 2009.

\item FREUND, W. S.; TORRES, R. C.; DAMAZIO, D.; SEIXAS, J. M. ; LIMA, D. E. F.
. Filtragem Online Neural para o ATLAS usando Informação Anelada das Seções do
Sistema de Calorimetria. 2009.

%%-------------------------------------------------------------------------------
%\item Sistema WEB para Testes de Equipamentos em Física de Altas Energias
%MAIDANTCHIK, C. L. L., SEIXAS, J. M., ALVES, A. M., FARIA, A., GRAEL,
%Felipe F., FEREIRA, F. G., GALVÃO, K K
%XXVII Encontro Nacional de Física de Partículas e Campos, 2006.
%
%O detector ATLAS, acoplado ao acelerador de partículas LHC do CERN, encontra-se
%atualmente em fase de comissionamento. Os testes realizados geram uma enorme
%quantidade de dados, que são posteriormente analisados pelos colaboradores em
%diferentes países. A cada execução dos programas de análise, uma série de
%procedimentos e configurações deve ser realizada pelos pesquisadores. Os
%gráficos e histogramas resultantes se referem aos níveis de energia durante uma
%colisão de partículas, e dados para o controle de qualidade dos equipamentos
%também são gerados a partir das análises.
%
%Este projeto apresenta o sistema \emph{Tile Commissioning Web System} que apóia
%a manipulação e análise dos dados provenientes dos testes realizados no
%Calorímetro de Telhas (TileCal), um dos sub-detectores do ATLAS, e apresenta
%ferramentas para a recuperação dos resultados obtidos. O sistema é composto por
%três softwares com interface Web que possuem funções específicas: o \emph{Web
%Interface for Offline Shifters}, o \emph{Tilecomm Analysis} e o
%\emph{AtlasMonitor}.
%
%O \emph{Web Interface for Offline Shifters} (WIS) automatiza o processo de
%análise, apresentando ao usuário uma tabela com todos os testes realizados e os
%tipos de análises que podem ser realizadas para cada um. Após a seleção do
%usuário, o WIS recupera o arquivo correspondente no sistema de armazenamento
%CASTOR (\emph{Cern Advanced STORage}) e executa remotamente o programa de
%análise requerido, seguindo todos os procedimentos e configurações exigidos. Ao
%final do processo de análise, os resultados são disponibilizados na interface
%do sistema, apresentando gráficos de níveis de energia e dados de controle de
%qualidade. Os gráficos e os dados são automaticamente armazenados nas
%respectivas bases de dados dos sistemas \emph{Tilecomm Analysis} e \emph{Atlas
%Monitor}. O primeiro software recupera os gráficos, através da associação com o
%tipo e o identificador do teste e com a seção do sub-detector que foi testada.
%O \emph{AtlasMonitor} insere automaticamente os dados resultantes nas folhas de
%controle de qualidade. Posteriormente, o colaborador pode inserir informações
%adicionais ou até mesmo criar uma nova folha de controle de qualidade.
%
%O sistema \emph{Tile Commissioning Web Syste}m está instalado no servidor do
%CERN e é utilizado pelos colaboradores do TileCal e seu desenvolvimento conta
%com a participação dos responsáveis tanto pelos testes dos equipamentos quanto
%pelo funcionamento do calorímetro.
%

\end{itemize}
