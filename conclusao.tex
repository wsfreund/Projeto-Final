\chapter{Conclusão}

O LHC, em operação no CERN, é o maior acelerador de partículas do mundo. O
ATLAS, o maior dos seus detectores, foi construído e é operado por uma
colaboração internacional envolvendo 172 institutos em 37 países.  Nesse
ambiente, dados referentes à instalação, resultados de testes ou monitoração
do desempenho dos milhares de equipamentos, são armazenados em bancos de dados
usando modelagens específicas para cada caso e até tecnologias de armazenamento
diferentes, tais como banco de dados Oracle ou MySQL.  Como a colaboração
envolve pessoas inseridas em diferentes culturas, pode também acontecer de haver
diferentes terminologias para um mesmo equipamento.  Além disso, se uma busca
é realizada sobre todos os dados da colaboração, é grande a probabilidade de não
trazer resultados úteis por causa de possíveis erros de digitação ou pelo grande
volume de dados retornados.  Por outro lado, se cada grupo da colaboração criar
um sistema de recuperação para os dados de interesse, será necessário um grande
esforço de manutenção para que os sistemas continuem funcionando durante toda a
operação do detector, que será de, pelo menos, 20 anos.

Para solucionar esses problemas de recuperação de dados, foi desenvolvido o
Glance. O sistema é baseado em interfaces de buscas independentes, de forma que
um único sistema pode manipular diversas interfaces.  O custo de manutenção é,
portanto,  diminuido pois há somente um sistema para ser mantido.  A descrição
das interfaces de recuperação não é dependente de uma determinada tecnologia de
armazenamento, e o sistema foi projetado de forma que possa recuperar dados de
diferentes repositórios. Além disso, diferentes perspectivas sobre o mesmo
conjunto de dados podem ser exibidas, criando interfaces para cada caso de
interesse.

O sistema Glance pode ser acessado através da Web, o que permite que seja usado
pelos colaboradores de qualquer parte do mundo.
%
O sistema auxilia a realização de buscas em bancos de dados. Para tal, apresenta
opções para o usuário escolher, minimizando os problemas devidos a erros de
digitação e terminologia dos atributos. Durante o processo de criação de uma
interface, o sistema apresenta a estrutura do repositório, refinando
sucessivamente os detalhes. Adicionalmente, as interfaces de
recuperação criadas são paramétricas.

O mecanismo de operações do Glance permite que os dados recuperados sejam
processados, e o resultado do processamento é exibido para o usuário final.
As transformações a serem realizadas podem genéricas, no sentido de poderem ser
usadas em diferentes contextos, tais como transformar uma série temporal de
valores numéricos em um gráfico, ou específicas a uma determinada necessidade,
tais como a realização do processo de \textit{unsmoothing} do DCS.
Dessa forma, os resultados da recuperação chegam ao usuário no formato adequado
para análise.

O sistema Glance está instalado nos servidores Web do CERN e vem sendo usado
ativamente pela colaboração.
%
Dentre suas aplicações a casos reais da colaboração, destacam-se a interface ao
banco de dados ATLASIntegration, o monitoramento do alinhamento dos componentes
com o ATLAS Survey e a monitoração de sensores para o DCS do TileCal.
%
Para o banco de dados ATLASIntegration, o sistema auxilia no processo de
instalação dos componentes do detector recuperando dados sobre posição dos
equipamentos, disposição dos cabos e os estados de conectividade entre cabos e
eletrônicos.
%
As aplicações para o ATLAS Survey e para o DCS do TileCal recuperam do banco de
dados amostras de sensores e, através do mecanismo de processamento, calculam
diferenças entre as leituras de sensores correlacionados, extraem médias e desvio
padrão dos valores e gera gráficos.
%
Os resultados são usados, no caso do ATLAS Survey, para detectar desníveis no
chão da caverna experimental, e, no caso do DCS, para diagnosticar problemas nas
fontes de alimentação do TileCal.

O sistema também foi aplicado ao banco de dados LHCb Integration, cujos
requisitos são muito parecidos com os do ATLAS Integration, mas registra
equipamentos do LHCb, que é outro detector de partículas do LHC.
%
Isso mostra que, devido ao fato do Glance ter sido projetado para ser genérico,
atende aos requisitos de recuperação de outras colaborações dentro do CERN, ou
até em contextos diferentes.

Atualmente existem no sistema mais de 400 interfaces de busca. Dessas, 150 são
destinadas a recuperação de informações de equipamentos para aplicações do
ATLAS Integration e LHCb Integration, enquanto 25 são utilizadas pelo sistema
DCS do TileCal, e 2 pelo o ATLAS Survey. As demais interfaces são utilizadas
por outros sistemas da Coordenação Técnica do ATLAS.

Dentre os possíveis futuros passos para o sistema, destacam-se: a integração
de dados correlacionados, mas armazenados em repositórios distintos; melhorias
no desempenho da recuperação e do processamento dos dados; criação de um
isolamento entre as aplicações do Glance, de forma que novas funcionalidades
possam ser introduzidas no sistema sem impactar as outras aplicações, caso haja
alguma falha de programação.


O Glance também recebeu contribuições de outros alunos participantes da
colaboração entre a UFRJ e o ATLAS, que desenvolveram funcionalidades além das
apresentadas neste documento.
%
Dentre elas, destacam-se a inserção de dados utilizando o conceito de interfaces
de inserção, análogo ao da interface de busca, realizado por Kaio Karam, e o
esquema de categorização hierárquica das interfaces armazenadas, para facilitar
o acesso, realizado por Cimar Massulo.
