\chapter{Conclusão}

%O LHC, em operação no CERN, é o maior acelerador de partículas do mundo. O
%ATLAS, o maior dos seus detectores, foi construído e é operado por uma
%colaboração internacional envolvendo 172 institutos em 37 países.  Nesse
%ambiente, dados referentes à instalação, resultados de testes ou monitoração
%do desempenho dos milhares de equipamentos, são armazenados em bancos de dados
%usando modelagens específicas para cada caso e até tecnologias de armazenamento
%diferentes, tais como banco de dados Oracle ou MySQL.  Como a colaboração
%envolve pessoas inseridas em diferentes culturas, pode também acontecer de haver
%diferentes terminologias para um mesmo equipamento.  Além disso, se uma busca
%é realizada sobre todos os dados da colaboração, é grande a probabilidade de não
%trazer resultados úteis por causa de possíveis erros de digitação ou pelo grande
%volume de dados retornados.  Por outro lado, se cada grupo da colaboração criar
%um sistema de recuperação para os dados de interesse, será necessário um grande
%esforço de manutenção para que os sistemas continuem funcionando durante toda a
%operação do detector, que será de, pelo menos, 20 anos.



\section{Perspectivas}


% Falar sobre a adição do eta e energia a rede neural como uma possível melhora
% para a eficiencia da rede neural, por dois motivos, o calorimetro nao eh
% uniforme em eta, e há correlação entre a produção de física e essa coordenada.

% Procurar nas conclusões e ver o que mais tinha que fazer.

% testar os canais de decaimento em fótons.

