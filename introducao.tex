\chapter{Introdução}
\label{cap:intro}
\glsresetall

\newacronym[type=Abrev]{t2calo}{\emph{T2Calo}}{Algoritmo para a Reconstrução do
Calorímetro no Segundo Nível de Filtragem} 
\newacronym[type=Abrev]{hltringer}{\emph{HLT\_Ringer}}{HLT\_EgammaCaloRinger} 

A engenharia tem auxiliado em diversas áreas através do emprego de
novas tecnologias e desenvolvimento de ferramentas que estão se tornando cada
vez mais necessárias, como na medicina, biologia, ciências financeiras, etc. Em 
alguns casos seu papel é ainda mais essencial, caso das áreas
multidisciplinares, quando ela e outras ciências se aliam para caminhar juntas no
desenvolvimento de ferramentas e tecnologias. Com isso, se faz necessário a
capacidade de sua adaptação à complexidade oferecida por esses novos ambientes, 
como a alta quantidade e diversidade de sensores, alta taxa e raridade de
eventos. Para realizar a interpretação dos diversos canais e a seleção dos eventos 
de interesse, se faz mão de uma ferramenta: o \glsdesc{sf}, que irá descartar a 
informação inútil ao problema em questão, remanescendo em sua maioria apenas
a parte necessária. É importante que o filtro seja eficiente para evitar a polarização
dos eventos, onde a porção interessante dos mesmos seria desprezada.
A tarefa se torna ainda mais difícil nos casos de raros
eventos de interesse, na qual a seleção errônea tem consequências piores,
sendo necessário o emprego de técnicas para sua otimização. 

No \glsdesc{sf} podem ser utilizados tanto métodos baseados no conhecimento 
especialista do assunto, quanto métodos baseados na informação estatística do processo, 
que geralmente filtram com maior qualidade os dados de interesse. Ainda, 
a combinação do conhecimento especialista ao processamento 
estatístico trás uma abordagem ainda mais poderosa, através da utilização das informações 
em técnicas não lineares, como redes neurais.

% TODO falar sobre aplicação de redes neurais em engenharia eletrica

A seleção de eventos pode ser realizada no ambiente em tempo real,
quando o processo de seleção ocorre durante a aquisição de dados. Nesse
ambiente, deseja-se reduzir o volume de dados a serem armazenados. A decisão deve ser realizada 
em curtos espaços de tempo, de maneira que atenda a taxa esperada de geração de
dados, e ainda ser eficiente, uma vez que os dados rejeitados serão perdidos. 
No ambiente em tempo real é possível o processamento dos dados em série
ou em paralelo. O processamento em série limita o tempo de processamento àquele
de aquisição dos dados, não sendo interessante quando há flutuação do
processamento dos eventos. Por outro lado, o processamento em
paralelo possibilita que eventos mais complexos ocorram durante um tempo maior
que a taxa de aquisição de dados. Nesse tipo de processamento, é necessário que
os eventos não ultrapassem o valor máximo de latência (tempo utilizado
para a tomada de decisão pelo \glsdesc{sf}), assim como obterem um valor 
médio de latência estipulado, determinados pela capacidade de processamento,
afim de evitar gargalos no \glsdesc{sf}.

Já no ambiente de análise a posteriori, o processo de seleção pode ser mais
complexo, uma vez que ele não realiza a decisão atendendo aos
requisitos de tempo quando em tempo real. Assim, é esperado 
uma melhor eficiência quando realizando a filtragem a posteriori.

\section{Motivação}

O \gls{cern} é o maior centro de
pesquisas em física de partículas do mundo, situado na fronteira da Suíça com a
França. O experimento de grande repercussão do \gls{cern} se
consiste no \gls{lhc}, um acelerador de partículas de 27 km de
circunferência que irá atingir energia de colisões nunca antes obtidas
experimentalmente. 
Nesse ambiente de colaboração internacional estarão presentes
todas as características de um sistema de alta taxa de dados e raros eventos de
interesse.

Serão acelerados pacotes\footnote{Um aglomerado de partículas.} de prótons em
feixes com sentidos opostos no \gls{lhc},
ocorrendo colisões em quatro pontos. Nesses pontos, são instalados 
detectores de partículas que observam o subproduto das colisões dos pacotes de prótons 
acelerados pelo \gls{lhc}. O \gls{atlas} é um dos detectores do
\gls{lhc}, sendo construído e operado por uma colaboração internacional envolvendo 174 institutos de 38
países. Ele foi projetado de modo a atender diversos dos requisitos da física 
experimental atual e por isso é dito de uso geral, contendo um total de $~140$ M 
de canais de leitura. Para atendê-los ele é dividido nos seguintes subdetectores:

\newacronym[type=Abrev]{id}{ID}{Detector Interno}
\newacronym[type=Abrev,user1={Eletromagnéticas}]{em}{EM}{Eletromagnético}
\newacronym[type=Abrev,user1={Hadrônicas}]{had}{HAD}{Hadrônico}
\newacronym[type=Abrev,\glslongpluralkey=Calorímetros Eletromagnéticos]
{ecal}{ECAL}{Calorímetro Eletromagnético}
\newacronym[type=Abrev,\glslongpluralkey=Calorímetros Hadrônicos]
{hcal}{HCAL}{Calorímetro Hadrônico}

\begin{itemize}
\item \gls{id}, responsável pela detecção da trajetória de partículas carregadas;
\item \gls{ecal}, cujo objetivo é realizar a absorção total da
energia de partículas eletromagnéticas;
\item \gls{hcal}, que de maneira similar ao eletromagnético realiza a
absorção da energia de partículas hadrônicas;
\item Espectrômetro de Múons, para a identificação e determinação da trajetória de
Múons.
\end{itemize}

A física estudada nesse experimento é muito rara, podendo ser observados eventos
de interesse algumas poucas vezes ao longo de vários dias de colisão. 
Um dos objetivos para o qual o \gls{atlas} foi 
designado é a identificação da partícula bóssom de Higgs, única partícula 
do \glslink{mp}{Modelo Padrão} ainda não observada. Estima-se que cerca de 17 k dessas
partículas\footnote{Considerando o decaimento esperado para a massa de 500 GeV.} deverão ser produzidas por ano, comparados
com o total de $1,7\times10^{16}$ eventos produzidos de interações inelásticas
não difrativas. 

%\newglossaryentry{byte}{type=Simb,name=B,
%  description={\emph{Byte}, unidade de informa{çã}o digital}
%\newglossaryentry{hertz}{type=Simb,name=Hz,
%  description={\emph{Hertz}, unidade de informa{çã}o digital}

É prevista uma média de cerca de 1,5 MB de espaço em disco rígido para cada evento de 
colisão. A taxa de cruzamento entre os pacotes de prótons é de 40 MHz, 
desta forma o fluxo de dados no decorrer do experimento será de 60 TB/s, impossibilitando o
armazenamento completo dos eventos ocorridos no experimento. Para tornar a
situação ainda mais complexa são esperadas cerca de 1 GHz de interações
inelásticas não difrativas quando operando nas condições nominais, onde apenas
uma pequena parte dessas interações irá gerar a física desejada como objeto de
estudo.

Assim é natural a adoção de um \glsdesc{sf} em tempo real para a
identificação dos eventos de interesse a serem armazenados, reduzindo a taxa de 1 GHz
para cerca de 100 Hz. O \glsdesc{sf} do \gls{atlas} realiza o processamento dos 
eventos em paralelo, estando dividido em três níveis sequenciais, cada um 
analisando o evento com maior complexidade:

\newacronym[type=Abrev]{fpga}{FPGA}{\emph{Field-Programmable Gate Array}} 
\newacronym[type=Abrev,\glslongpluralkey={Regiões de Interesse}]{roi}{RoI}{Região de Interesse} 
\newacronym[type=Abrev]{l1}{L1}{Primeiro N\'{i}vel de filtragem} 
\newacronym[type=Abrev]{l2}{L2}{Segundo N\'{i}vel de filtragem} 
\newacronym[type=Abrev]{ef}{EF}{Filtro de Eventos, ou terceiro n{í}vel de
filtragem} 
\newacronym[type=Abrev]{hlt}{HLT}{Alto Nível de Filtragem} 

\begin{itemize}
\item O \glslink{l1}{Primeiro Nível de filtragem (L1)} realiza a filtragem com
\gls{fpga}\footnote{Circuitos integrados com portas lógicas.}, 
utilizando resolução reduzida das células do detector
com um tempo fixo de 2 $\mu$s, reduzindo a taxa de eventos para
75 kHz. Ele também é responsável pela identificação de regiões no detector onde
há informação relevante, referidas como \gls{roi}. Somente a
informação contida nessa região é propagada para o segundo nível, de forma a
minimizar o fluxo de dados no sistema.
\item O \glslink{l2}{Segundo Nível (L2)} analisa somente os eventos que passaram pelas condições do
primeiro nível. Utiliza a resolução total das células do detector, 
tendo um tempo de latência de 10 ms, de maneira a reduzir a taxa de eventos para
2 kHz. É implementado em linguagens de alto nível como C++ e python para sua configuração. 
Para esse nível são disponíveis cerca de 500 processadores, com quatro
núcleos de processamento, conectados em rede.
\item O \glslink{ef}{Terceiro Nível (EF)}, além de avaliar com maior acurácia aqueles que foram
selecionados pelo segundo nível, realiza a procura por eventos não identificados pelo
primeiro nível por causa de sua menor resolução. Ele possui um tempo de 
latência de 10 s, reduzindo a taxa de eventos para 200 Hz. Utiliza a mesma
infraestrutura computacional do segundo nível, entretanto tem disponíveis 1900
processadores.
\end{itemize}

Uma vez o evento sendo aceito pelo terceiro nível, o mesmo é então armazenado
para futura analise pelos físicos que irão utilizar os algoritmos desenvolvidos 
para analise a posteriori. Os físicos dão o nome para esses algoritmos de Sistema de
Reconstrução, pois eles os permitem fazer a reconstrução da física que ocorreu
durante a colisão e sua interação com o detetor. Para facilitar a leitura irá se 
referir ao \glsdesc{sf} em tempo real simplesmente como Sistema de 
Filtragem, já a sua versão de analise a posteriori será referida como \glsdesc{sr}.

Um dos canais de interesse do experimento, o Canal \acrshort{eg},
deseja identificar elétrons \glslink{eg}{(e$^-$)}, pósitrons \glslink{eg}(e$^+$)
ou fótons \glslink{eg}{($\gamma$)}, 
partículas de componentes eletromagnéticas. Muitos dos decaimentos do bóssom de
Higgs serão nessas partículas, de forma que esse canal é de fundamental
importância para o experimento. Ainda, a partícula $J/\Psi$, partícula importante
para a determinação da resolução em baixa energia do detector, 
também decai em elétrons, sendo assim um objeto de estudo nesse canal.
Jatos, partículas onde componentes hadrônicas são predominantes, 
mascaram a assinatura das partículas desejadas pelo Canal \acrshort{eg}, fazendo 
com que a tarefa da identificação dessas partículas não seja trivial.

\section{Objetivo} % {{{

O objetivo deste trabalho é otimizar o \gls{egcaloringer},
um algoritmo alternativo que vem sido desenvolvido pelo \gls{lps}
para realizar a discriminação 
no canal \acrshort{eg}. Tal algoritmo foi idealizado para o \gls{l2}, 
sendo implementado na colaboração em 2005.
Nele a informação de calorimetria é tratada através do conhecimento especialista
e então propagada para um método estatístico 
de discriminação, onde atualmente é utilizado redes neurais. 
Para que tal melhoria seja alcançada no
algoritmo será realizado um estudo de preprocessamento dos dados,
através da escolha da melhor normalização a ser adotada.

Ainda, um dos requisitos para a adesão do algoritmo no \glsdesc{sf} é a
implementação de uma versão do mesmo para o \glsdesc{sr},
de maneira  que a colaboração possa utilizá-lo e entender o seu
funcionamento. Por esse motivo, também se faz necessário a implementação 
da nova versão para a complexa estrutura de código desse ambiente.

Por fim, é necessária uma analise da eficiência do algoritmo para ambos os
ambientes, utilizando como parâmetro de comparação seus equivalentes 
desenvolvidos pela colaboração.

 % }}}
\section{Organização do documento} {{{

O Capítulo~\ref{cap:cern} é dedicado ao \gls{cern} e ao contexto de colaboração internacional
no qual o projeto foi desenvolvido, dando a base necessária para o entendimento
do experimento \gls{atlas}, fazendo uma introdução sobre a Física de Partículas,
o acelerador \gls{lhc}, os componentes do \gls{atlas} e as ferramentas
utilizadas pela Colaboração.

No Capítulo 3 está a apresentação dos ambientes aonde atua o algoritmo proposto,
o Canal \acrshort{eg}. Nele será realizada uma apresentação do \gls{sr} e do
\gls{sf}, considerando o algoritmo padrão utilizado pela Colaboração para
realizar as tarefas necessárias do Canal \gls{eg} como a obtenção dos parâmetros
das partículas e discriminação. Também será apresentado o algoritmo proposto em
detalhes para suas duas versões, a versão de análise em tempo real no \gls{sf},
\gls{hltringer} e a versão de análise a posteriori, o \gls{egcaloringer}. A
descrição da implementação da versão em análise também será realizada nesse 

O Capítulo 4 é destinado a apresentação dos resultados e a sua discussão,
sendo dividido nos dois estudos realizados: a análise da normalização mais
indicada, e o estudo de eficiência do \gls{egcaloringer}. 
}}}
