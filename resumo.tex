\vfill
\begin{center}
\section*{Resumo\label{Resumo}}
\end{center}
\addcontentsline{toc}{chapter}{Resumo}

Ferramentas e tecnologias utilizadas em engenharia têm encontrado sua
aplicação em outras áreas do conhecimento.
Na Física, um dos ambientes que está no limiar da ciência atual é o maior acelerador 
de partículas já construído, o \acrshort{lhc}, que permitirá aos
cientistas validar e desenvolver teorias, como o \glslink{mp}{Modelo Padrão}. A
única partícula prevista ainda não observada por esse modelo é o bóson
de Higgs. Um dos detectores do \acrshort{lhc}, o \acrshort{atlas}, tem entre seus objetivos 
confirmar a existência de tal partícula. Todavia, o bóson de Higgs é altamente
instável, o que faz com que ele decaia rapidamente em outras partículas, como elétrons e fótons, de forma que é importante 
a detecção das mesmas para o sucesso do experimento. Por sua vez, essas
partículas têm sua assinatura mascarada por outras, como jatos hadrônicos, o que
torna o processo de sua identificação não trivial. 

O \gls{sr} do \acrshort{atlas} é o responsável por identificar as
partículas e seus parâmetros, e a capacidade de descoberta de novas físicas 
depende de sua eficiência. Outras dificuldades a serem superadas são a alta taxa de
eventos e a escassez dos eventos de interesse. 
O \gls{sf} foi desenvolvido para selecionar as informações relevantes para o experimento, 
atuando em tempo real, de forma a reduzir a grande quantidade de dados a ser armazenada.

Nesse contexto internacional, o presente trabalho realiza a continuação
do projeto \acrshort{egcaloringer}. O projeto consiste de um algoritmo 
para a identificação de elétrons e fótons, utilizando a informação especialista 
do detetor que é, então, propagada para um método estatístico de discriminação, 
atualmente formado por Redes Neurais, sendo uma das contribuições da
Engenharia para este experimento. O algoritmo de discriminação foi otimizado 
através do estudo do pré-processamento mais indicado para a rede neural. 
Não obstante, embora o algoritmo tenha sido idealizado para o \gls{sf}, 
o mesmo foi portado ao \gls{sr}, de modo a permitir sua utilização e 
entendimento pela colaboração do \acrshort{atlas}. Sua performance foi 
testada utilizando como referência o algoritmo padrão utilizado 
pela colaboração. O algoritmo proposto superou o algoritmo padrão nas três bases
de dados testadas em relação à sua capacidade de discriminação.

\paragraph*{}

\noindent Palavra-chave: Redes Neurais, Sistema de Filtragem, Calorimetria.

\vfill

\clearpage

% Abstract
\vfill
\begin{center}
\section*{Abstract\label{Abstract}}
\end{center}
\addcontentsline{toc}{chapter}{Abstract}

Tools and technologies used in engineering have found their
application in other areas of expertise assisting on the Science evolutionary
process. In physics, one environment on Science's edge is the biggest
particle accelerator in the world, the \acrshort{lhc}, which will enable
scientists to validate and develop new theories, as the Standard Model. The only
particle not yet observed provided by this model is the Higgs boson. One of the LHC 
detectors, the ATLAS, has among its goals to confirm the existence
of this particle. The Higgs boson is highly unstable and will rapidly
decay in others particles, like electrons and photons, such that the detection
of those particles is important to the experiment triumph. These particles will 
have their signature faked by others such as hadronic jets, 
which makes the identification process not trivial.

The ATLAS Reconstruction System is responsible to identify the particles and its
parameters, and, as consequence, the discoverability of new physics is related to its
efficiency. Other difficulties that need to be overcome are the high event rate together
with events of interest scarcity. The Trigger System was developed to
select the relevant information for the experiment as an on-line algorithm in
order to reduce data storage.

In this international context, the present work proceed with the Egamma
Calorimeter project. The project consists of one algorithm to the 
ATLAS Reconstruction System, acting on the $e/\gamma$ physics channel. In this algorithm,
particular information from the detector is propagated to a stochastic
discrimination method, currently composed of Artificial Neural Networks, which
is one of the Engeneering contributions for this experiment.
This document describes the algorithm efficiency optimization through a study
of the most indicated data pre-processing. Furthermore, it was developed an
offline version for this algorithm, which was officially added to the ATLAS framework. 
The algorithm performance was tested using as benchmark the
standard algorithm developed by the collaboration. Finally, the proposed
algorithm overcomes the standart algorithm at the three datasets tested with
respect to their discrimination capacity.

\paragraph*{}

\noindent Key-words: neural networks, Trigger, Calorimetry.

\vfill
\clearpage
