\vfill
\begin{center}
\section*{Resumo\label{Resumo}}
\end{center}
\addcontentsline{toc}{chapter}{Resumo}

O Centro Europeu de Pesquisa Nuclear (CERN) tem sido foco da mídia atual devido
ao maior acelerador de partículas \emph{Large Hadron Collider} (LHC). Esse
acelerador dará aos cientistas a oportunidade de explorar um novo universo da física
experimental de altas energias, permitindo-os validar e desenvolver teorias.

Uma das teorias que deseja-se validar, o Modelo Padrão, prevê a partícula bóssom
de Higgs, ainda não observada experimentalmente. Um dos detectores do LHC, o
ATLAS, tem dentre seus objetivos confirmar a existência de tal partícula.

O bóssom de Higgs é altamente instável e irá decair rapidamente em outras
partículas, como elétrons e fótons, de forma que é importante a deteção das
mesmas para o sucesso do experimento.

Entretanto os eventos de interesse são raros, de forma que grande parte da
física produzida já é bem conhecida. O Sistema de Filtragem foi desenvolvido
com o objetivo de selecionar as informações relevantes para o experimento.
Existem duas versões, uma \emph{online}, que reduz a massiva quantidade de dados
a serem armazenados, e outra \emph{offline}, utilizada por físicos nos dados
selecionados de forma a validar suas teorias.

Nesse contexto internacional, o presente trabalho realiza a continuação
do projeto \emph{Egamma Calorimeter Ringer}. O projeto se consiste de um algoritmo 
para o sistema de filtragem do ATLAS, atuando no canal físico de
elétrons/fótons. Nele a informação especialista do detetor é propagada para um
método estatístico de discriminação, atualmente composto de redes neurais.

No presente trabalho realizou-se a otimização de eficiência do algoritmo, 
através do estudo do pré-processamento dos dados mais indicado. 
Ainda, foi desenvolvida uma versão para a analise em 
\emph{offline} desse algoritmo, versão adicionada oficialmente ao
sistema de análise do ATLAS. Por fim, comparou a perfomance desse algoritmo 
utilizando como referência o algoritmo padrão implementado pela colaboração.

\paragraph{}

\noindent Palavras-Chave:  redes neurais, sistema de filtragem, CERN, ATLAS.

\vfill

\cleardoublepage

% Abstract
\vfill
\begin{center}
\section*{Abstract\label{Abstract}}
\end{center}
\addcontentsline{toc}{chapter}{Abstract}

The European Laboratory for Particle Physics (CERN) is one of the current
media highlights due to the biggest particle accelerator, the Large Hadron
Collider (LHC). This accelerator will give cientists the oportunity to explore 
one new experimental physics universe, allowing them to validate and develop
theories.

One of those theories, the Standart Model, foresee the Higgs boson particles,
yet not seen experimentally. One of the LHC detectors, the ATLAS, has between
its goals to corroborate this particle existence.

The Higgs boson is highly unstable and will decay rapidly into other particles,
such as electrons and photons, so that it is important the detection of those to
experiment triumph.

However the events of interest are rare, therefore a large physics ammount
produced is already well known. A Reconstruction System was developed in the
intent to select the relevant information for the experiment. There are two
versions, the online called as Trigger, which reduce the massive ammount of data
to be stored, and another offline, used by physicists on stored data to evaluate
their theories.

In this international context, the present work proceed with the Egamma
Calorimeter project. The project consists of one algorithm to the ATLAS
Reconstruction System, acting on the electron/foton physics channel. In this
algorithm particular information from the detector is propagated to a stochastic
discrimination method, currently composed of artificial neural networks.

This document describes the algorithm efficiency optimization, through a study
of the most indicated data preprocessing. Futhermore, is was developed an
offline version for this algorithm, which was officially added to the ATLAS
framework. Finally, this algorithm perfomance was compared using as benchmark
the standart algorithm developed by the colaboration.


\paragraph{}

\noindent Key-words: neural networks, trigger, CERN, ATLAS.


\vfill
