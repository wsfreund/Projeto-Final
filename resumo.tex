\vfill
\begin{center}
\section*{Resumo\label{Resumo}}
\end{center}
\addcontentsline{toc}{chapter}{Resumo}
O presente trabalho desenvolve o Glance, um sistema \emph{Web} de
recuperação e operação de grandes massas de dados armazenados em bancos de
dados dispersos.
%
A motivação para o sistema veio das necessidades da colaboração internacional
que construiu, e atualmente opera, o detector de partículas ATLAS, no CERN
(\textit{European Organization for Nuclear Research}).

O ATLAS foi construído por um dos maiores esforços colaborativos do meio
científico, envolvendo 172 institutos de 37 países e é composto por centenas de
milhares de componentes.
%
Nessa colaboração, cada equipamento foi construído por grupos de colaboradores
em seus respectivos institutos de origem e, em seguida, levado ao CERN para ser
montado e testado em sua posição definitiva.
%
Os dados que são gerados sobre os equipamentos são armazenados em repositórios
gerenciados pelos próprios colaboradores que constroem os equipamentos,
estando, portanto, geograficamente dispersos e não usam uma mesma modelagem,
tecnologia ou não necessariamente uma mesma terminologia.

O Glance é genérico pelo fato de não depender de uma modelagem ou
tecnologia específica.
%
Para isso o sistema usa o conceito de interfaces de recuperação, que são
descritas em uma linguagem independente da tecnologia de recuperação.
%
As interfaces de buscas apresentadas pelo Glance são paramétricas, ou seja, o
usuário pode especificar critérios de busca envolvendo os atributos da
interface e um valor fornecido pelo usuário.
%
Após recuperados, o sistema pode processar os dados, tais como a extração de
estatísticas ou geração de gráficos, a fim de exibi-lo em um formato que seja
útil aos colaboradores.

O sistema está instalado nos servidores do CERN e em uso pela colaboração.
%
Dentre as aplicações do sistema, podem-se destacar a recuperação, pela
Coordenação Técnica do ATLAS, do estado de instalação dos equipamentos e
conectividade deles com os cabos, e a monitoração dos sensores de nível, pelo
grupo que controla o alinhamento entre os componentes do ATLAS.
%
Além disso, o sistema é utilizado como fonte de dados para o sistema que
monitora temperaturas e voltagens das fontes de alimentação de um dos
sub-detectores.

%O CERN (\textit{European Organization for Nuclear Research}) é o maior centro de
%pesquisas em física de partículas do mundo.
%
%Atualmente seu principal projeto é o acelerador de partículas LHC (\textit{Large
%Hadron Collider}), que acelera dois feixes de prótons a velocidades muito
%próximas à da luz, e então os colide.
%
%Em torno de cada ponto de colisão há um detector de partículas que observa o
%subproduto das colisões, com o intuito de comprovar teorias que explicam
%questões ainda não respondidas a respeito da matéria, como a origem da massa.

%O ATLAS (\textit{A Toroidal LHC ApparatuS}), o maior dos detectores do LHC, foi
%construido por um dos maiores esforços colaborativos do meio científico,
%envolvendo 172 institutos em 37 países.
%
%O detector tem forma cilíndrica e mede cerca de 45 metros de comprimento por 25
%metros de diâmetro, sendo composto por dezenas de milhares de componentes.
%
%Nesse ambiente, os repositórios de dados estão dispersos geograficamente e usam
%diferentes tecnologias, modelagens e, até mesmo, terminologias.

\paragraph{}

\noindent Palavras-Chave: banco de dados, web, colaboração internacional, CERN,
ATLAS.

\vfill

\cleardoublepage

% Abstract
\vfill
\begin{center}
\section*{Abstract\label{Abstract}}
\end{center}
\addcontentsline{toc}{chapter}{Abstract}

This document describes the development of Glance, a Web system for retrieval
and processing of big amounts of data stored in different databases.
%
The motivation for this system has come from the necessities of the
international collaboration that built, and currently operates, the ATLAS
detector, at CERN (European Organization for Nuclear Research).

Built by one of the largest collaborative efforts in the scientific medium,
ATLAS involves 172 institutes of 37 countries, and is made of hundreds of
thousands of components.
%
Each equipment was designed and built by a group of collaborators in their home
institute, and then taken to CERN to be assembled in its final position, and
then tested.
%
The data generated for each equipment during this process is stored in
repositories managed by the same group that built it.
%
The information for the detector equipments, therefore, is geographically
scattered and is not stored in a standard technology, modeling and might even
use different terminologies.

Glance is a generic system for data retrieval, as it does not depends the
modeling or technology of the repository.
%
In order to achieve this, the system uses the concept of search interfaces,
described in a technology independent language.
%
The search interfaces handled by Glance are parametric, i.e., the user specifies
search criteria by choosing an attribute, a comparison operator and a value.
%
Once retrieved from the database, the data can be processed in order to be
presented to the user in a useful format. Examples of processing are the
generation of plots, and the calculation of means and standard deviation.

The Glance system is currently installed at CERN's Web servers and being used by
the collaboration.
%
Among the applications of the system are the retrieval of
equipment installation status by the ATLAS Technical Coordination, and the
monitoring of hydrostatic level sensors by the ATLAS Alignment Survey group.
%
Glance is also used as a data source for the DCS Web System, which monitors
temperature, voltage and electric current sensors of the Detector Control System
from one of the sub-detectors.

\paragraph{}

\noindent Key-words: database, web, international collaboration, CERN, ATLAS.



\vfill
