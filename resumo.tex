\vfill
\begin{center}
\section*{Resumo\label{Resumo}}
\end{center}
\addcontentsline{toc}{chapter}{Resumo}

\newacronym[type=Abrev]{cern}{CERN}{\emph{Centre Européene pour la Rech{è}rche Nucleaire}} 
\newacronym[type=Abrev]{lhc}{LHC}{\emph{Large Hadron Collider}} 
\newacronym[type=Abrev]{mp}{MP}{Modelo Padr{ã}o de intera{çã}o entre as part{í}culas
elementares} 
\newacronym[type=Abrev]{atlas}{ATLAS}{\textit{A Toroidal LHC ApparatuS}} 
\newacronym[type=Abrev]{sf}{SF}{Sistema de Filtragem}
\newacronym[type=Abrev]{sr}{SR}{Sistema de Reconstrução}

A complexidade e quantidade de requisitos exigidos tem se elevado conforme o avanço da ciência, 
de forma que novas técnicas precisam ser aplicadas para a continuação do processo evolutivo.
A engenharia tem auxiliado em diversas áreas, apresentando ferramentas e novas tecnologias. 
Na física, um dos ambientes que proporciona todos esses desafios é o maior acelerador 
de partículas já construído no mundo, o \acrshort{lhc}, que permitirá aos
cientistas validar e desenvolver teorias, como o \glslink{mp}{Modelo Padrão}. A
única partícula ainda não observada prevista por esse modelo é o bóssom
de Higgs, onde um dos detectores do \acrshort{lhc}, o \acrshort{atlas}, tem dentre seus objetivos 
confirmar a existência de tal partícula. Todavia, o bóssom de Higgs é altamente instável 
e irá decair rapidamente em outras partículas, como elétrons e fótons, de forma que é importante 
a detecção das mesmas para o sucesso do experimento. Essas partículas terão sua
assinatura mascarada por outras, de modo que o processo de identificação não é
trivial. 

O \gls{sr} do \acrshort{atlas} é o responsável por identificar as
partículas e seus parâmetros, onde a capacidade de descoberta de novas físicas 
está relacionada com sua eficiência. Outras dificuldades a serem superadas são a alta taxa de
eventos acompanhado da escassez de eventos de interesse. 
O \gls{sf} foi desenvolvido para selecionar as informações relevantes para o experimento 
atuando em tempo real, de forma a reduzir a quantidade massiça de dados a ser armazenada.

\newacronym[type=Simb]{ej}{\ensuremath{e/J}}{El{é}tron / Jato} 
\newacronym[type=Simb]{eg}{\ensuremath{e/\gamma}}{El{é}tron, Pósitron e Fóton} 
\newacronym[type=Abrev,sort=egcaloringer]{egcaloringer}{EgCaloRinger}{\acrshort{eg} \emph{
Calorimeter Ringer}} 

Nesse contexto internacional, o presente trabalho realiza a continuação
do projeto \acrshort{egcaloringer}. O projeto consiste de um algoritmo 
para a identificação de elétrons e fótons utilizando a informação especialista 
do detetor que é, então, propagada para um método estatístico de discriminação, 
atualmente composto de redes neurais. O algoritmo de discriminação foi otimizado 
através do estudo do pré-processamento mais indicado para a rede neural. 
Não obstante, embora o algoritmo tenha sido idealizado para o \gls{sf}, 
o mesmo foi portado ao \gls{sr}, de modo a permitir sua utilização e 
entendimento pela Colaboração do \acrshort{atlas}. A sua performance foi 
testada utilizando como referência o algoritmo padrão utilizado 
pela colaboração.

\paragraph*{}

\noindent Palavras-Chave: Redes Neurais, Sistema de Filtragem, Calorimetria.

\vfill

\cleardoublepage

% Abstract
\vfill
\begin{center}
\section*{Abstract\label{Abstract}}
\end{center}
\addcontentsline{toc}{chapter}{Abstract}

The amount of requirements and its complexity needed are enhanced while science
develops, so new techniques are requested to maintain the evolutive proccess.
The engeneering has supported in many areas of knowledge, introducing tools and
new technologies.
In physics, one enviromment which provides all this challenges is the biggest
particle accelerator in the world, the \acrshort{lhc}, which will enable
scientists validate and develop new theories, aas the Standart Model. The only
particle yet not observated, provided by this model is the Higgs boson, where
one of the LHC detectors, the ATLAS, has among its goals to confirm this
particle existance. However, the Higgs boson is highly unstable and will rapidly
decay in others particle, like electrons and photons, so it is important the
detection of those to the experiment triumph. This particles will have their
signature faked by others, so the identification process it not trivial.

The ATLAS Reconstruction System is requested to identify the particle and its
parameters, while the discoverability of new physics is related to its
efficiency. Other difficulties to be overcome are the high event rate together
with the short rate for events of interest. The Trigger System was developed to
select the relevant information for the experiment as an online algorithm in
order to reduce data storage.

In this international context, the present work proceed with the Egamma
Calorimeter project. The project consists of one algorithm to the 
ATLAS Reconstruction System, 
acting on the $e/\gamma$ physics channel. In this algorithm,
particular information from the detector is propagated to a stochastic
discrimination method, currently composed of artificial neural networks.
This document describes the algorithm efficiency optimization, through a study
of the most indicated data preprocessing. Furthermore, it was developed an
offline version for this algorithm, which was officially added to the ATLAS fra-
mework. Finally, the algorithm performance was compared using as benchmark the
standard algorithm developed by the collaboration.

\paragraph*{}

\noindent Key-words: neural networks, Trigger, Calorimetry.


\vfill
