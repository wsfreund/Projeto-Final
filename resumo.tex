\vfill
\begin{center}
\section*{Resumo\label{Resumo}}
\end{center}
\addcontentsline{toc}{chapter}{Resumo}

\newacronym[type=Abrev]{cern}{CERN}{\emph{Centre Européene pour la Rech{è}rche Nucleaire}} 
\newacronym[type=Abrev]{lhc}{LHC}{\emph{Large Hadron Collider}} 
O \gls{cern} tem sido foco da mídia atual devido
ao maior acelerador de partículas já construído, o \gls{lhc}. Esse
acelerador dará aos cientistas a oportunidade de explorar um novo universo da física
experimental de altas energias, permitindo-os validar e desenvolver teorias.

\newacronym[type=Abrev]{mp}{MP}{Modelo Padr{ã}o de intera{çã}o entre as part{í}culas
  elementares} 
\newacronym[type=Abrev]{atlas}{ATLAS}{\textit{A Toroidal LHC ApparatuS}} 

Uma das teorias que deseja-se validar, o \glslink{mp}{Modelo Padrão}, prevê a partícula bóssom
de Higgs, ainda não observada experimentalmente. Um dos detectores do \gls{lhc}, o
\gls{atlas}, tem dentre seus objetivos confirmar a existência de tal partícula.

O bóssom de Higgs é altamente instável e irá decair rapidamente em outras
partículas, como elétrons e fótons, de forma que é importante a detecção das
mesmas para o sucesso do experimento.

Entretanto, grande parte da física produzida já é bem conhecida, de forma que 
os eventos de interesse são raros. O Sistema de Filtragem foi desenvolvido
com o objetivo de selecionar as informações relevantes para o experimento.
Existem duas versões, uma em tempo real, que reduz a massiva quantidade de dados
a serem armazenados, e outra a posteriori, normalmente referida como Sistema de
Reconstrução, utilizada por físicos nos dados selecionados de forma a validar suas teorias.

\newacronym[type=Simb]{ej}{\ensuremath{e/J}}{El{é}tron (e p{ó}sitron)/Jato} 
\newacronym[type=Simb]{eg}{\ensuremath{e/\gamma}}{El{é}tron (e p{ó}sitron)/Fóton} 
\newacronym[type=Abrev,sort=egcaloringer]{egcaloringer}{EgCaloRinger}{\acrshort{eg} \emph{
Calorimeter Ringer}} 

Nesse contexto internacional, o presente trabalho realiza a continuação
do projeto \acrshort{egcaloringer}. 
O projeto se consiste de um algoritmo 
para o Sistema de Filtragem do \gls{atlas}, atuando no canal físico de
\acrshort{eg}. Nele a informação especialista do detetor é propagada para um
método estatístico de discriminação, atualmente composto de redes neurais.

Assim, realizou-se a otimização da eficiência do algoritmo, 
através do estudo do preprocessamento mais indicado. 
Ainda, foi implementada uma versão para a analise 
a posteriori desse algoritmo, versão adicionada oficialmente ao
sistema de análise do \gls{atlas}. Por fim, comparou a performance desse algoritmo 
utilizando como referência o algoritmo padrão implementado pela colaboração.

\paragraph{}

\noindent Palavras-Chave:  redes neurais, Sistema de Filtragem, \acrshort{cern},
  \acrshort{atlas}.

\vfill

\cleardoublepage

% Abstract
\vfill
\begin{center}
\section*{Abstract\label{Abstract}}
\end{center}
\addcontentsline{toc}{chapter}{Abstract}

The \glslink{cern}{European Laboratory for Particle Physics (CERN)} is one of the current
media highlights due to the biggest particle accelerator ever built, \glslink{lhc}{the Large 
Hadron Collider (LHC)}. This accelerator will give scientists the opportunity to explore 
one new experimental physics universe, allowing them to validate and develop
theories.

One of those theories, the Standard Model, foresees the Higgs boson particles,
yet not seen experimentally. One of the \acrshort{lhc} detectors, the
\gls{atlas}, has between its goals to corroborate this particle existence.

The Higgs boson is highly unstable and will decay rapidly into other particles,
such as electrons and photons, so that it is important the detection of those to
experiment to corroborate the existence, or not, of this particle.

However the events of interest are rare, therefore a large amount amount physics
production is already well known. A Trigger System was developed in the
intent to select the relevant information for the experiment. There are two
versions, the on-line, which reduce the massive amount of data
to be stored, and another offline, refereed as Reconstruction System, used by 
physicists on stored data to evaluate their theories.

In this international context, the present work proceed with the Egamma
Calorimeter project. The project consists of one algorithm to the
\gls{atlas} Reconstruction System, acting on the \acrshort{eg} physics channel. In this
algorithm, particular information from the detector is propagated to a stochastic
discrimination method, currently composed of artificial neural networks.

This document describes the algorithm efficiency optimization, through a study
of the most indicated data preprocessing. Furthermore, it was developed an
offline version for this algorithm, which was officially added to the
\gls{atlas} framework. Finally, the algorithm performance was compared using as benchmark
the standard algorithm developed by the collaboration.


\paragraph{}

\noindent Key-words: neural networks, Trigger, \acrshort{cern}, \acrshort{atlas}.


\vfill
